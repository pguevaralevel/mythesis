\documentclass[12pt,a4paper]{book}
\usepackage[french]{babel}
\usepackage[utf8]{inputenc}
\usepackage[T1]{fontenc}
\usepackage[a4paper, margin=2cm]{geometry}
\usepackage{amsmath, amsfonts, amssymb}
\usepackage[pdftex]{graphicx}
\usepackage{ragged2e}
\usepackage{setspace}
\usepackage[T1]{fontenc} %para escurecer a cor da fonte
\usepackage{lmodern}
\usepackage[mathcal]{eucal}
\usepackage{pdfpages}
\usepackage{latexsym}
\usepackage{minitoc}
\usepackage{pdftexcmds}
\usepackage{tikz}
\usepackage{bm}
\usepackage[all]{xy}
\usepackage{makeidx}
\usepackage{geometry}
\usepackage[multiple]{footmisc} %permite colocar virgulas entre footnotes
\usepackage{relsize} %permite aumentar o tamanho das integrais
\usepackage{pbox} %permite quebrar a linha numa tabela
\usepackage{paralist}
\usepackage[toc,page]{appendix}
\bibliographystyle{unsrt}

\newcommand{\ket}[1]{\ensuremath{|#1\rangle}\xspace} %mecanica quantica
\newcommand{\bra}[1]{\ensuremath{\langle #1|}\xspace} %mecanica quantica
\newcommand{\braket}[2]{\ensuremath{\langle #1|#2\rangle}\xspace} %mecanica quantica
\newcommand{\cent}[1]{$#1\,^{\circ}{\rm C}$} %degrees
% % % % % % % % % % % % % % % % % % % % % % % % % % % % % % % % % % % % %
\usepackage{fancyhdr} %para customizar o cabecalho
\pagestyle{fancy} %para customizar o cabecalho
\fancyhf{} %para customizar o cabecalho
\fancyhead[EL]{\nouppercase\leftmark} %para customizar o cabecalho
\fancyhead[OR]{\nouppercase\rightmark} %para customizar o cabecalho
\fancyhead[ER,OL]{\thepage} 

\usepackage{titlesec} %para centralizar o capitulo e o ti�tulo
\titleformat{\chapter}[display]%para centralizar o capitulo e o ti�tulo
{\normalfont\huge\bfseries\centering}{\chaptertitlename\ \thechapter}{20pt}{\Huge}%para centralizar o capitulo e o ti�tulo
\usepackage[shortcuts]{extdash} 
\begin{document}
\newenvironment{romanpages}
{\setcounter{page}{1}
	\renewcommand{\thepage}{\roman{page}}}
{\newpage\renewcommand{\thepage}{\arabic{page}}\setcounter{page}{1}}
\newcommand{\PRESn}{Dr Brigitte PEPIN-DONAT}
\newcommand{\PRESp}{CNRS, France}
\newcommand{\PRESq}{Président}

\newcommand{\INVAn}{Dr }
\newcommand{\INVAp}{xxxxxxx, Univ.}
\newcommand{\INVAq}{Invit\'e}
\newcommand{\RAPAn}{Dr Christophe RAYNAUD}
\newcommand{\RAPAp}{Universit\'e de Montpellier, France}
\newcommand{\RAPAq}{Rapporteur}
\newcommand{\RAPBn}{Dr Johannes GIERSCHNER}
\newcommand{\RAPBp}{IMDEA, Spain}
\newcommand{\RAPBq}{Rapporteur}

\newcommand{\EXAAn}{Dr Paul TOPHAM}
\newcommand{\EXAAp}{Aston University, UK}
\newcommand{\EXAAq}{Examinateur}

\newcommand{\EXABn}{Jury}
\newcommand{\EXABp}{UPPA}
\newcommand{\EXABq}{Examinateur}

\newcommand{\EXACn}{M.}
\newcommand{\EXACp}{xxxxxxxxxx}
\newcommand{\EXACq}{Examinateur}

\newcommand{\JURY}{
	\textbf{\large{\textbf{JURY}}}\\[\baselineskip]
	\begin{tabular}{l@{\protect\hspace{0.5cm}}l@{\protect\hspace{0.5cm}}l}
		%\textbf{\large{\textbf{Composition du jury :}}}
		\RAPAn &\RAPAp &\RAPAq\\
		\RAPBn &\RAPBp &\RAPBq\\
		\EXAAn &\EXAAp &\EXAAq\\
		%\EXABn &\EXABp &\EXABq\\
		\PRESn &\PRESp &\PRESq\\
		%\EXACn &\EXACp &\EXACq\\
		%\INVAn &\INVAp &\INVAq\\
	\end{tabular}
}

%\begin{minipage}[c]{15cm}
%\addtolength{\hoffset}{-0.5cm}
%\addtolength{\textwidth}{0.6cm} 
\thispagestyle{empty}
\newgeometry{left=2cm,bottom=2cm,right=2cm,top=1cm}
%\baselineskip=13pt
%\vspace*{-4cm}
\begin{center}
\includegraphics[height=2.2cm]{logo/uppa.png}
\end{center}
\begin{center}
	\hbox{\raisebox{0.4em}{\vrule depth 0pt height 1pt width 15cm}}\setlength{\baselineskip}{13pt}~\\
	{\Large{\textbf{THESES}}}\\[\baselineskip]
	préparée et présentée à \\[\baselineskip]
	L'UNIVERSIT\'E DE PAU ET DES PAYS DE L'ADOUR \\[\baselineskip]
	pour obtenir le grade de\\[\baselineskip]
		{\Large\textit{{\textbf{Docteur}}}}\\[\baselineskip]
	Spécialité : CHIMIE PHYSIQUE \\[\baselineskip]
	par\\[\baselineskip]
  	\baselineskip=20pt
	{\LARGE{\textbf{Patricia Guevara Level}}}\\
	Sujet de la thèse :\\
	\hbox{\raisebox{0.2em}{\vrule depth 0pt height 3.5pt width 17cm}}
	\setlength{\baselineskip}{4pt}
	\hbox{\raisebox{0.4em}{\vrule depth 0pt height 1pt width 17cm}}\setlength{\baselineskip}{10pt}~\\
	\vspace*{-5pt}
	{\Large\textbf{Title in English \\ \textit{Titre en français}}}~\\[\baselineskip]
	\hbox{\raisebox{0.4em}{\vrule depth 0pt height 1pt width 17cm}}~\\
	
	Directeur de Thèse: \textbf{Pr Didier B\'EGU\'E}\\[\baselineskip]
	Soutenue le  XXth, 2016\\[\baselineskip]
	devant la commission d'examen composée de :\\[\baselineskip]~\\[\baselineskip] 
	
	\JURY
\end{center}

% Nom profession Qualite du president du jury
\newcommand{\President}[3]{%
	\renewcommand{\PRESn}{#1}%
	\renewcommand{\PRESp}{#2}%
	\renewcommand{\PRESq}{#3}%
}

% Nom Profession Qualité de l'invité
\newcommand{\Invite}[3]{%
	\renewcommand{\INVAn}{#1}%
	\renewcommand{\INVAp}{#2}%
	\renewcommand{\INVAq}{#3}%
}

% Nom Profession Qualite d'un rapporteur A
\newcommand{\RapporteurA}[3]{%
	\renewcommand{\RAPAn}{#1}%
	\renewcommand{\RAPAp}{#2}%
	\renewcommand{\RAPAq}{#3}%
}
% Nom Profession Qualite d'un rapporteur B
\newcommand{\RapporteurB}[3]{%
	\renewcommand{\RAPBn}{#1}%
	\renewcommand{\RAPBp}{#2}%
	\renewcommand{\RAPBq}{#3}%
}

%Prenom Nom Qualite d'un examinateur A
\newcommand{\ExaminateurA}[3]{%
	\renewcommand{\EXAAn}{#1}%
	\renewcommand{\EXAAp}{#2}%
	\renewcommand{\EXAAq}{#3}%
}
%Prenom Nom Qualite d'un examinateur B
\newcommand{\ExaminateurB}[3]{%
	\renewcommand{\EXABn}{#1}%
	\renewcommand{\EXABp}{#2}%
	\renewcommand{\EXABq}{#3}%
}

%Prenom Nom Qualite d'un examinateur C
\newcommand{\ExaminateurC}[3]{%
	\renewcommand{\EXACn}{#1}%
	\renewcommand{\EXACp}{#2}%
	\renewcommand{\EXACq}{#3}%
}



% une page blanche (deuxième de couverture)

\newpage\thispagestyle{empty}\addtocounter{page}{-5}
\null\newpage\thispagestyle{empty}

%\newpage\thispagestyle{empty}
%~\newpage\thispagestyle{empty}
%\newpage

\setcounter{secnumdepth}{3}
\setcounter{tocdepth}{3}

\newenvironment{bottompar}{\par\vspace*{\fill}}{\clearpage}
\restoregeometry
\include{remerciements/remerciements}
\include{epigrafe/epigrafe}
\include{dedic/dedic}
\include{resumes/ingles}
\include{resumes/frances}
\renewcommand*\contentsname{Table of Contents} %troca o nome do sumario para o que quiser
\tableofcontents \thispagestyle{empty} 
\listoffigures \thispagestyle{empty} 
\listoftables \thispagestyle{empty}
\newpage
\include{abrev/abrev}
\newpage
\newgeometry{textwidth=16cm}
	\chapter*{Introduction}
	\minitoc
	\restoregeometry
	
	
	Les asphaltènes représentent une fraction du pétrole brut, définie sur la seule base de ses propriétés de solubilité. Ainsi, appartiennent à la famille des asphaltènes les composés chimiques du pétrole solubles dans les solvants aromatiques, tels que le toluène, mais insolubles dans les $n$-alcanes, tels le $n$-pentane ou le $n$-hexane. Cette définition, limitée à une unique propriété physique, laisse comprendre implicitement que la nature chimique des asphaltènes, c'est-à-dire leur composition aussi bien que le type et la forme des différentes structures moléculaires qui les constituent, reste inconnue. En somme, ces systèmes chimiques sont pareils à un puzzle dont on ne possèderait que l'emballage. Les propriétés macroscopiques -- la solubilité -- correspondent à l'image du puzzle présentée sur la boîte, mais le nombre de pièces, de même que leurs formes et la manière dont elles s'assemblent, nous sont inconnus. Tel est le défi auquel font face depuis de nombreuses années les chercheurs, aidés de leur palette de techniques expérimentales et de méthodes théoriques. L'enjeu est de taille, tant cette famille de composés pose de problèmes, aux divers stades de l'extraction, de la production et du raffinage du pétrole. Dans l'industrie pétrolière, le seul mot \og asphaltène \fg{} est invariablement associé aux phénomènes d'agrégation caractéristiques de ces systèmes, responsables entre autres de l'obstruction des pipelines. \\
	
	Au fil des années, diverses techniques de caractérisation ont permis de lever le voile sur une partie des propriétés chimiques de ces systèmes. Il est désormais admis que les asphaltènes sont formés d'un ensemble de structures moléculaires composées de noyaux aromatiques poly-condensés, ramifiés par des chaînes alkyles. La majeure partie des propriétés physiques macroscopiques de ces systèmes, notamment leur propension à l'agrégation et leur comportement en solubilité, découlent toutefois de ce qu'ils présentent un taux significatif d'hétéroatomes (N, S et O) et de métaux (V et Ni), à l'origine d'interactions intermoléculaires caractéristiques, comme les liaisons hydrogène, les interactions acido-basiques, la formation de complexes métalliques, les phénomènes de \og $\pi-\pi$ stacking \fg{} et de \og $H-\pi$ stacking \fg{}. Comme il sera détaillé dans le \textbf{premier chapitre} du présent travail, l'ensemble des techniques de caractérisation potentiellement utilisables sur ces fractions les plus lourdes du pétrole brut ont été employées à maintes reprises, au gré des améliorations technologiques, pour tenter de percer à jour la structure chimique intime des asphaltènes. Bien que ces travaux aient pour certains offert des avancées remarquables, nombreuses sont les études -- expérimentales comme théoriques -- dont les conclusions \textit{a priori} contradictoires suscitent la controverse. Si l'on a pu jusqu'à présent identifier quelques-uns des atomes à l'origine des interactions responsables des phénomènes parasites observés aux différentes étapes de la production du pétrole, l'état actuel des connaissances ne nous permet pas de donner une définition structurale précise pour ces systèmes. \\
	
	Le présent travail de thèse prend naissance dans l'idée originale proposée en 2011 par J. Shaw et K. Michaelian, d'employer les spectroscopies infra-rouge (IR) et Raman -- en tant que techniques de caractérisation courantes -- à la base d'un \og protocole \fg{} innovant de caractérisation moléculaire des asphaltènes. Dans le projet proposé par ces chercheurs, la spectroscopie IR sert de pivot, et sa simplicité de mise en œuvre, en même temps que les principes physiques sur lesquels elle repose, permettent de la coupler aisément à une large variété de techniques expérimentales.\\
	
	En première étape à ce \og protocole \fg, J. Shaw et K. Michaelian proposent d'employer des analyses IR et Raman à la constitution d'une large base de données, dont le but serait de faciliter l'identification des motifs moléculaires constitutifs des asphaltènes. 
	Dans une seconde étape, le couplage des techniques d'IR lointain et de photo-acoustique permettrait de caractériser la présence -- ou non -- d'interactions de type $\pi-\pi$. Cet approfondissement présenterait alors le double intérêt de vérifier la validité des familles de molécules précédemment identifiées, et d'associer à ces interactions une fréquence et une intensité, deux paramètres physiques caractéristiques. \\
	
	Pour répondre à ces attentes, le soutien des méthodes de la chimie quantique pour la caractérisation des propriétés vibrationnelles apparaît crucial, et c'est dans ce cadre que s'inscrit ce travail de thèse. En aidant à l'identification des signatures caractéristiques des différentes familles de composés et des modes inter-moléculaires qui leurs sont associés, la chimie calculatoire offre un guide précieux aux expérimentateurs, et contribue à la construction des deux étapes du schéma proposé par J. Shaw et K. Michaelian.\\
	
	Au vu de la diversité des asphaltènes, du point de vue de la nature chimique et de la taille, la première partie de ce travail de thèse a consisté en une réflexion en termes de stratégie calculatoire et méthodologique, le but étant d'accéder aux données vibrationnelles calculées dans les hypothèses harmonique et anharmonique. Ce travail préliminaire, véritable clé de voûte de notre investigation théorique, sera détaillé dans le \textbf{second chapitre} du présent manuscrit.\\
	
	Au-delà de la stratégie calculatoire, la nature méconnue des asphaltènes et leur mélange au sein des fractions lourdes du pétrole posent de nombreux problèmes de mise en œuvre. 
	Tout d'abord, la diversité des composés chimiques constitutifs de ces systèmes nous a contraint à faire le choix de molécules modèles judicieuses, aptes à représenter les différentes classes de composés postulées comme présentes au sein des asphaltènes. 
	Par ailleurs, le comportement en mélange de ces différentes familles de composés induit, dans le cadre des modélisations, deux problématiques majeures, que sont : 
	\begin{itemize}
	\item la prise en compte des effets de l'environnement chimique d'un système, de sorte à apporter une description théorique aussi proche que possible des conditions de mesure
	\item la traduction, en termes de description quantique, de l'ensemble des interactions identifiées entre ces systèmes, et notamment les interactions longue portée.
	\end{itemize}
	
	La taille des systèmes à l'étude a imposé le recours aux méthodes de la théorie de la fonctionnelle de la densité (DFT). Toutefois, il est bien connu que le talon d'Achille de ces méthodes demeure la description des forces de dispersion, pour laquelle les avancées méthodologiques sont encore lentes et parfois mises en défaut. La nécessité, dans la présente étude, de traduire les effets d'interactions de type longue portée, nous impose de présenter, dans le \textbf{troisième chapitre}, les solutions proposées jusqu'à ce jour pour palier certains des défauts de la DFT dans ce domaine. Cette partie abordera notamment la notion de fonctionnelle hybride et détaillera le principe des approches employées dans ce travail, qui ajoutent \textit{ad hoc} à un calcul Kohn-Sham (KS) usuel des fonctions semi-empiriques. Ces approches sont reconnues aujourd'hui comme étant les plus aptes à rendre compte des effets de dispersion dans le cadre de l'utilisation des méthodes de la DFT. \\
	
	À la suite de ces trois chapitres, indispensables pour établir la méthodologie suivie pour répondre à notre problématique, nous reviendrons dans une quatrième partie sur la mise en œuvre pratique de notre stratégie calculatoire. Celle-ci présentera donc les calculs préliminaires menés sur des molécules test (à savoir des composés thiophéniques), pour lesquelles nous disposions d'une grande quantité d'informations moléculaires de référence, afin d'évaluer la fiabilité des approximations auxquelles nous avons eu recours. En effet, comme il aura été explicité plus avant, le traitement quantique de ces systèmes de grande dimension, qui repose sur la résolution de l'équation vibrationnelle de Schrödinger (REVS) dans la double hypothèse des anharmonicités électrique et mécanique, requiert de réduire la taille de l'espace actif des configurations vibrationnelles. De fait, dans l'objectif de parvenir à un ratio acceptable temps \textit{versus} précision des calculs réalisés, nous nous sommes employés à la construction de bases réduites de modes normaux. L'emploi de ces bases réduites, déduites des couplages observées entre les modes inter-moléculaires et les modes de torsion angulaire et de stretching des liaisons C-H, sera donc corroboré et justifié par les résultats présentés au \textbf{chapitre 4}.  \\
	
	Les trois derniers chapitres de ce manuscrit présentent l'ensemble des résultats obtenus par le biais des stratégies calculatoires mises en place, en gardant à l'esprit que notre travail se veut traiter des différentes problématiques inhérentes aux asphaltènes, que sont : leur nature chimique, et particulièrement le rôle des hétéroéléments, l'influence de l'environnement et les interactions à l'origine de l'agrégation caractéristique de ces systèmes.\\ 
	
	C'est pourquoi le \textbf{chapitre 5} se propose de décrire les résultats des calculs moléculaires réalisés dans le but d'élucider le rôle singulier des hétéroatomes dans les signatures vibrationnelles de ces systèmes. Ces calculs, qui emploient les outils méthodologiques développés pour la REVS, ont pour double objectif : 
	\begin{itemize}
	\item de déterminer la signature propre de molécules modèles isolées
	\item d'analyser les modifications respectives de ces signatures sous l'effet des interactions inter-moléculaires, simulées sur des dimères. 
	\end{itemize}

	Afin d'assurer une analyse complète, nous avons mené nos calculs sur une collection de six motifs moléculaires, pour les trois hétéroéléments (N, S et O), ainsi que sur des motifs identiques pris pour références, ne comportant que des atomes de carbone.\\


	Le \textbf{chapitre 6} se concentre quant à lui sur l'effet de l'environnement, simulé par des calculs à l'état solide. Ce dernier constitue en effet l'état le plus proche des conditions réelles, pour lequel nous possédons des données expérimentales de référence obtenues en photo-acoustique sur des systèmes cristallins. Le passage de l'état moléculaire à l'état solide induit bien évidemment de profondes modifications dans la description des modes de vibration intermoléculaires. Nos calculs visent ainsi à caractériser explicitement ces modes en identifiant des singularités dans les signatures spectrales, avec pour objectif final de fournir aux expérimentateurs des caractéristiques précises leur permettant d'identifier ou non la présence de systèmes associés. \\ 
	
	Enfin, le \textbf{chapitre 7} de ce manuscrit développe les conclusions de travaux menés dans le sens d'une meilleure compréhension des phénomènes d'agrégation des asphaltènes. Les calculs réalisés dans cet objectif se placent donc à une toute autre échelle et permettent d'inscrire le présent travail dans une étude complète de type \og bottom-up \fg menée au laboratoire. Seront ainsi présentées dans cette partie des simulations en dynamique moléculaire, dont l'objectif est de comprendre le rôle des interactions intermoléculaires dans la nano-agrégation (les échelles supérieurs étant la micro agrégation et  la floculation toutes deux responsables des désagréments observés par les industriels à chaque fois que des asphaltènes sont présents dans les matrices de pétrole). Ces simulations seront de deux types, chacune présentant un objectif distinct : 
	\begin{itemize}
	\item dans un premier temps, nous considèrerons des systèmes modèles des asphaltènes, dans le but de comprendre l'influence de la position des hétéroéléments au sein de la moléculle. En effet, suivant leur position, différentes sortes d'interactions spécifiques se créent (liaisons hydrogène, interactions $\pi-\pi$ ou $\pi-H$, \textit{etc.})
	\item dans un second temps, nous nous baserons sur des systèmes moléculaires mis en évidence très récemment par le biais de la spectroscopie XPS, et pouvant être considérés \textit{a priori} comme plus réalistes. L'intérêt est ici d'étudier le comportement des agrégats suivant les proportions dans lesquelles sont mélangés ces systèmes et suivant la nature et la teneur des hétéroatomes qui les constituent.   
	\end{itemize} 




 









 	
	
	
	
	

		
	
	
	
\newgeometry{textwidth=16cm}
\chapter{Asphaltènes}
\minitoc
\restoregeometry

\newpage	
	\section*{Introduction}
	\spacing{1.5}
	
	Les enjeux de la compréhension des propriétés physiques et chimiques d'un système reposent sur l'arrangement des atomes qui constituent sa structure. Dans cette optique, chimistes et physiciens ont développé au cours des années diverses méthodes visant à caractériser les systèmes chimiques, en tenant compte de leur environnement. Ces avancées méthodologiques restent toutefois limitées par les moyens technologiques, et la modélisation des systèmes de grande dimension représente un défi de taille pour les chercheurs. Entrant dans cette catégorie, les asphaltènes demeurent des constituants mal connus du pétrole brut et leur nature exacte comme leur taille restent, aujourd'hui encore, autant de freins à la caractérisation précise, en termes de composition chimique, des huiles brutes. L'étude et la compréhension approfondie de ces systèmes représente pourtant un enjeu scientifique et économique majeur, la présence de ces composés dans toutes les étapes, de la production au raffinage du pétrole, impliquant à l'heure actuelle l'emploi de procédés spécifiques particulièrement onéreux \cite{akbarzadeh2007asphaltenes}. Plus précisément, les asphaltènes, présents sous forme de clusters ou de nanoparticules au sein des huiles brutes, tendent à augmenter la viscosité du pétrole brut et nécessitent des extractions chimiques successives, certaines étapes requérant de surcroît un traitement à haute température. \\
	
	Si les informations expérimentales (obtenues par spectrométrie de masse, pyrolyse, etc.) recueillies au fil des années sur les asphaltènes n'ont pas permis, à elles seules, d'élucider leur composition chimique ou leur arrangement, éléments clés qui ouvriraient la voie à une caractérisation approfondie, des études conjointes expérience-théorie ont révélé des éléments intéressants \cite{neurock1994molecular}. Notamment, l'étude des huiles lourdes en tant que systèmes biphasiques complexes requiert d'appréhender en premier lieu les phénomènes d'agrégation conduisant à la formation de micelles d'asphaltènes. L'organisation de ces micelles permet ensuite de mettre en lumière l'effet de l'addition de pentane, employé comme solvant dans l'extraction des asphaltènes.   \\
	
	
	En définitive, bien que la modélisation des asphaltènes sur la base de ces éléments ouvre des pistes à la compréhension des processus d'agrégation, de floculation et de précipitation de ces systèmes, il est devenu indispensable d'augmenter le niveau de précision des représentations chimiques des asphaltènes pour espérer comprendre leurs propriétés et prédire, sinon prévenir, les conséquences des phénomènes d'agrégation. 
	Dans cette optique, il convient de revenir à la composition moléculaire de base des asphaltènes. Les asphaltènes sont constitués d'un mélange hétérogène de structures polycondensées et d'hétéroatomes (azote, oxygène, soufre). La présence de ces hétéroélements induisant la formation de nombreuses interactions, notamment de type van der Waals, ces atomes constituent la clé de voûte de l'assemblage moléculaire \cite{zhang2013speciation, coelho2012elucidation}. Elucider les interactions intermoléculaires impliquant ces hétéroatomes est le premier enjeu dans la représentation approfondie de la structure des asphaltènes.  \\
	
	En reprenant l'ensemble des études réalisées jusqu'à ce jour sur la structure des asphaltènes, ce chapitre se veut un état des connaissances actuelles de ces systèmes. En pointant du doigt les zones d'ombre qui persistent dans la représentation de ces molécules, nous poserons le cadre de travail dans lequel s'inscrit la présente étude. 
	
	
	
	\newpage	
	
	\section{Définition}
	Historiquement, le premier emploi du terme "asphaltène" est attribué au français Boussingault, qui l'utilise dès 1837 pour caractériser certains des produits de distillation de l'asphalte \cite{goual2012petroleum}. Les "asphaltènes" sont alors les solides insolubles dans l'alcool mais solubles dans l'essence de térébenthine, par opposition aux "pétrolènes", constituants volatils solubles dans l'éther. \\
	A l'heure actuelle, le terme "asphaltène" désigne la fraction lourde du pétrole, insoluble dans les n-alcanes (tels que le n-pentane ou le n-heptane) mais soluble dans les solvants aromatiques tels que le toluène, la pyridine ou le benzène. D'un point de vue structural, les asphaltènes sont constitués de noyaux aromatiques condensés, substitués par des groupements aliphatiques et napthéniques et faisant intervenir des hétéroatomes (azote, soufre et oxygène) au sein d'arrangements de type hétérocycles \cite{strausz1992molecular}. La proportion globale d'hétéroatomes, comme la fraction précise d'atomes d'azote, de soufre ou d'oxygène, peut être quelque peu variable suivant les procédés d'extraction employés pour isoler les asphaltènes de l'huile brute \cite{calles2007properties}.
	
	\bigskip
	\section{Composition élémentaire}
	
	Les méthodes d'analyse principales appliquées pour déterminer la composition élémentaire d'une fraction lourde du pétrole ont été décrites par Barbelet \textit{et al.}\cite{barbelet1979analyses}. La composition élémentaire des asphaltènes en atomes de carbone et d'hydrogène semble varier faiblement suivant leur origine, et a été évaluée respectivement à $82\% \pm 3\%$ et $8.1\% \pm 0.7\%$ par Speight et Moschopedis \cite{speight1979some}. En outre, contrairement aux fractions plus légères du pétrole, au sein desquelles les hydrocarbures sont principalement de nature aliphatique (paraffine cyclique de type mono ou di-naphtène) ou monoaromatiques, les fractions lourdes incluent des structures naphténiques et aromatiques comportant plus de six cycles alkyliques.
	
	\subsection{Composition en hétéroatomes}
	
	
	\subsubsection{Soufre}
	Les dérivés soufrés contenus dans les asphaltènes sont semblables aux espèces que l'on retrouve dans les fractions plus légères du pétrole, mais leur proportion varie sensiblement. En règle générale, ces dérivés se divisent en cinq grandes classes chimiques : thiols, sulfures, disulfides, sulfoxydes et thiophènes. Thiols, sulfures, sulfoxydes et disulfides peuvent être classés plus précisément suivant leur nature cyclique ou acyclique, soit suivant l'organisation du squelette carboné (alkyle, aryle ou alkyl-aryle). Les dérivés thiophéniques, quant à eux, sont des structures polyaromatiques condensées autour d'un noyau thiophénique (benzo, dibenzo, naphtobenzo-thiophènes, etc.). Au sein des fractions lourdes, la majorité des espèces soufrées sont des dérivés thiophéniques, suivi par des dérivés sulfures (cycliques et acycliques). Des espèces de type sulfoxydes ont également pu être mises en évidence, dans des proportions très variables, leur teneur variant entre 0.3\% et 10.3\% \cite{merdrignac2007physicochemical, speight2004petroleum}.
	
	\subsubsection{Azote}
	Dans les fractions brutes, l'azote est présent en quantité inférieure aux autres hétéroéléments. On distingue néanmoins deux types de composés en fonction de la teneur en azote de l'asphaltène considéré. C'est ainsi que l'on distingue les composés "bases", dont la proportion d'azote est inférieure à l'unité, des composés "neutres". \\
	Les familles basiques qui ont pu être caractérisées présentent des structures dérivées de la quinoléine, contenant entre deux et quatre cycles aromatiques dont les configurations varient (péri ou catacondensées, avec différents degrés d'alkylation). Plus particulièrement, des benzo, dibenzo et tétrahydroquinoléines, ainsi que des azapyrènes ont été mis en évidence. \\ 
	De même, des structures neutres présentant divers degrés d'alkylation ont été détectées : carbazoles, benzo-carbazole et dibenzo-carbazoles. \\
	La proportion de structures neutres et basiques au sein d'une fraction brute est fortement liée aux paramètres géochimiques du site de provenance du pétrole. 
	Enfin, il est à noter que des structures dérivées de la porphyrine ont été mises en évidence, même si de tels composés se retrouvent en proportions plus anecdotiques. Leur présence, caractérisée au travers du complexe qu'elles ont tendance à former avec les ions nickel ou vanadium, dépend une fois de plus de l'origine du pétrole dont est extraite la fraction lourde. Le pourcentage d'occurence de ces structures varie communément entre 0.6 et 3.3\% \cite{merdrignac2007physicochemical, speight2004petroleum}.
	
	\subsubsection{Oxygène}
	Des structures oxygénées se retrouvent également dans les fractions lourdes du pétrole, leur teneur variant entre 0.3\% et 4.9\% \cite{speight2004petroleum}. Différentes familles ont pu être identifiées, notamment des dérivés du phénol, tandis que la caractérisation de groupements carboxyles révèle la présence d'ester, d'acides carboxyliques, de cétones, d'amides ou de sulfoxydes.
	
	
	\bigskip
	
	\section{Poids moléculaire des asphaltènes}
	
	Un rapide passage en revue de la bibliographie des asphaltènes révèle tout le débat autour de leur poids moléculaire. En effet, la distribution, en termes de poids moléculaire, varie entre 400 et 1500 Da pour les fractions de faible poids moléculaire, et peut atteindre $10^{6}$ Da pour les fractions de haut poids moléculaire \cite{mullins2008contrasting}. Le phénomène d'agrégation des asphaltènes est une des raisons avancées pour justifier qu'a l'heure actuelle aucune méthode n'ait permis de déterminer avec précision le véritable poids moléculaire de ces espèces. De nombreux procédés ont été employés pour tenter de résoudre ce problème, mais ceux-ci conduisent à des résultats ambigüs, qu'il serait hasardeux de généraliser. A titre d'exemple, des mesures par osméométrie à tension de vapeur font apparaître des résultats fortement dépendants du type de solvant employé ; en présence d'un solvant apolaire tel que le toluène, les résultats apparaissent surestimés, au contraire de l'utilisation d'un solvant polaire telle que la pyridine ou le nitrobenzène. 
	%S'agissant d'autres méthodes d'analyse, Groenzin et Mullins\cite{groenzin2000molecular} parviennent, par spectroscopie de fluorescence résolue en temps, à un poids moléculaire moyen de 750 Da environ, pour une distribution de 500 à 1000 Da, là où Pinkston et col.\cite{pinkston2009analysis} rapportent une distribution de 350 à 1050 Da pour une étude par spectrométrie de masse de résonance cyclotronique ionique à transformée de Fourier (FT-ICR). Ces derniers supposent l'existence de fragments de plus bas poids moléculaires, non détectés même à haute énergie électronique du fait de la stabilisation des intermédiaire obtenus par les chaînes alkyles des asphaltènes. 
	
	
	Merdrignac et col.\cite{merdrignac2006evolution} rapportent l'évolution du poids moléculaire des asphaltènes et comparer les changements causés par le procédé d'hydroconvertion. 
	
	\bigskip
	
	\singlespacing
	\section{Caractérisation de la structure moléculaire des asphaltènes}
	
	\spacing{1.5}
	%\subsection{Modèles structuraux }
	
	La description de la structure des asphaltènes repose à l'heure actuelle sur deux modèles construits sur la base des résultats de diverses caractérisations expérimentales.  
	Le premier modèle, baptisé "modèle continental", représente les asphaltènes comme de larges coeurs constitués de quatre à dix cycles aromatiques condensés, ramifiés par des chaînes alkyles courtes \cite{groenzin2000molecular}. Ce modèle a été proposé par Zhao \textit{et al.} \cite{zhao2001molecular}figure\ref{figZ1} après caractérisation des asphaltènes obtenus par fractionnement avec du pentane supercritique, puis remployé par Rogel et Carbognani \cite{rogel2003density} dans leur travail sur des asphaltènes stables et instables extraits de pétrole provenant du Vénézuela. 
	Ce type de représentations est corrélé par des analyses en spectroscopie RMN, diffraction des rayons X et spectroscopie de fluorescence résolue en temps (TRDF). 
	
	
	
	\begin{figure}
		\centering
		\includegraphics[scale=0.8]{image/Zhao}
		\caption[Structure moyenne du modèle continental des asphaltènes]{Structure moyenne du modèle continental des asphaltènes. Zhao et col. 2001}
		\label{figZ1}
	\end{figure}
	
	
	Le modèle "archipel" présente quant à lui ces systèmes comme un ensemble de régions à caractère aromatique, constituées de deux à trois cycles condensés, reliées par des chaînes carbonées. Ce modèle s'appuie sur des analyses par pyrolyse, oxydation, dégradation thermique et dispersion angulaire neutronique, telles que rapportées par Gawrys \textit{et al.} \cite{gawrys2003role}. Sheremata \textit{et al.}\cite{sheremata2004quantitative} ont notamment employé ce modèle pour la description des asphaltènes, comme le présente la figure reproduite ci-dessous (figure \ref{fig2}). 
	
	\begin{figure}
		\centering
		\includegraphics[scale=0.8]{image/Sher}
		\caption[Molecule d'asphaltènes type archipel]{Représentation d'asphaltènes à l'aide du modèle archipel, proposé par Sheramata \textit{et al.}}
		\label{fig2}
	\end{figure}
	
	Ces deux modèles, basés sur des structures moléculaires bien distinctes, conduisent néanmoins à des propriétés physico-chimiques différentes. La description des agrégats d'asphaltènes ainsi que leur solubilité dans le pétrole brut seront nettement impactées par l'emploi de l'un ou l'autre de ces modèles. Du fait de la planéité induite par la présence d'un grand nombre de cycles aromatiques condensés, la représentation d'un agrégat d'asphaltènes dans le cadre du modèle continental conduit à un empilement de plans. A l'inverse, les chaînes alkyles du modèle archipel confère une toute autre géométrie à ces systèmes : les asphaltènes peuvent se courber par le biais d'interactions moléculaires pour former des macro-agrégats globulaires susceptibles de piéger les molécules de solvant. 
	
	
	
	\subsection{Analyses par spectroscopie RMN}
	
	La spectroscopie de résonance magnétique nucléaire est couramment employée pour l'analyse structurale et la caractérisation de matériaux et de substances organiques. 
	%Elle est fondée sur les propriétés magnétiques de certains noyaux atomiques.Les noyaux les plus étudiés, dans le cadre de la chimie organique sont l'hydrogène $^{1}H$ et le carbone $^{13}C$. 
	Cette technique représente donc une méthode d'analyse de choix pour l'étude des asphaltènes et a, à ce titre, permis d'acquérir les premiers résultats visant à résoudre leur composition chimique. Dès 1982, les travaux de Murphy \textit{et al.} \cite{murphy1982determination} révèlent que la structure primaire des asphaltènes se compose de multiples cycles aromatiques condensés (l'indice de condensation relevé sur ces systèmes étant de 3 au moins) ainsi que d'une grande variété de chaînes aliphatiques, longues et ramifiées. Par la suite, les travaux de Yen \textit{et al.} en RMN $^{1}H$ ont donné une première idée de la proportion d'hydrocarbures saturés en regard de la fraction d'hydrocarbures aromatiques au sein de molécules d'asphaltènes  \cite{yen1984study}. Enfin, plus récemment, Durand \textit{et al.} ont employé les résultats de leurs analyses par RMN $^{13}C$ pour évaluer les indices de substitution et de condensation d'échantillons d'asphaltènes de différentes origines \cite{durand2010effect}. Leur étude tend à démontrer qu'au sein d'agrégats d'asphaltènes cohabitent les deux modèles structuraux évoqués ci-avant, à savoir le modèle continental et le modèle archipel. 
	
	
	\subsection{Analyses par diffraction des rayons X}  
	
	Une étude du caractère aromatique et des paramètres cristallins des asphaltènes, des résines et des gilsonites issus du pétrole a été présentée par Yen \textit{et al.} sur la base d'analyses par diffraction des rayons X \cite{yen1961investigation}. Leurs résultats, confortés par les travaux réalisés par Shirokoff \textit{et al.} sur quatre échantillons d'asphaltènes issus d'huiles brutes provenant d'Arabie Saoudite \cite{shirokoff1997characterization}, représentent les asphaltènes comme des feuillets de cycles aromatiques condensés portant des ramifications de natures naphténiques et aromatiques.
	Toutefois, Altget et Boduszynky rappellent judicieusement dans leur étude \cite{altgeltcomposition} qu'il convient de prendre avec précaution les résultats provenant d'analyses en diffraction, lorsque les données géométriques fournies par l'analyse servent de base pour construire un modèle structural du système étudié. Leurs travaux semblent en outre indiquer que la détermination du caractère aromatique des asphaltènes est, par ce biais, inappropriée sinon arbitraire. 
	
	
	
	\subsection{Analyses par spectroscopie de Fluorescence Résolue en Temps} 
	
	Cette méthode d'analyse a été employée à plusieurs reprises pour tenter d'élucider plus précisément la taille et la structure de molécules d'asphaltènes faiblement agrégées.  
	Les premiers travaux en ce sens, réalisés par Groenzin et Mullins  \cite{groenzin1999asphaltene}, tendent à montrer que les molécules d'asphaltènes ne seraient constituées que d'un seul groupe chromosphère. En effet, le faible poids moléculaire obtenu pour ces systèmes, comme la coloration des molécules, sont autant d'éléments recueillis incompatibles avec la représentation des asphaltènes sous forme de larges structures pseudo-polymériques constitués de deux ou trois cycles aromatiques reliés entre eux par de longues chaînes aliphatiques. En définitive, ces premières analyses semblent conforter la représentation des asphaltènes par le modèle continental. \\
	Par la suite, Souza \textit{et al.} ont rapporté l'agrégation persistante des asphaltènes à la faible concentration de 0.8 g/L, soit sous le seuil de concentration critique de nanoagrégation \cite{souza2009study}. Leurs travaux corroborent par ailleurs les résultats de Groenzin \textit{et al.} quant à la structure des molécules d'asphaltènes en cela qu'ils concluent à une structure primaire constituée par un anneau polyaromatique de quatre cycles ou plus. 
	
	
	\bigskip
	
	\subsection{Analyses par spectroscopie infrarouge}
	
	La caractérisation d'un système chimique en termes de composition passe nécessairement par des analyses en spectroscopie infrarouge, laquelle permet l'identification des groupements fonctionnels de ce système. Toutefois, l'étude des fractions lourdes du pétrole, la diversité chimique et la taille des espèces présentes rend complexe l'utilisation de la spectroscopie IR. C'est ce que soulignent Yuan \textit{et al.}, qui relèvent l'existence d'un faible nombre d'analyses employant l'infrarouge moyen pour la caractérisation physique et chimique des huiles lourdes \cite{hongfu2006determination}. 
	Parmi les travaux notables, l'étude couplée en spectroscopies IR et UV-sisible menée par El-Bassoussi \textit{et al.} sur deux échantillons d'asphaltènes provenant d'Egypte a permis de classifier les espèces présentes en mono, di et polyaromatiques. En accord avec les deux modèles de représentation des asphaltènes, les espèces prédominantes sont les espèces di ou polyaromatiques \cite{el2010characterization}. En 2007, Rodrigues Coelho \textit{et al.} \cite{coelho2007characterization} démontrent l'existence d'une corrélation linéaire entre les intensités des bandes infrarouges symétriques et antisymétriques associées aux atomes d'hydrogène aromatiques de type arènes méthyl-substitués dans les régions 2900-3100 $cm^{-1}$ et 700-900 $cm^{-1}$. Enfin, Laxalde \textit{et al.} \cite{laxalde2014combining} rapportent une interprétation ajustée sur les structures modèles des asphaltènes en repérant les vibrations de stretching des liaisons C-H (hors du plan) et C=C.
	
	\bigskip
	
	\section{Coexistence des structures moleculaires}
	
	La compilation de l'ensemble de ces résultats expérimentaux tend à suggérer que les deux modèles structuraux des asphaltènes peuvent coexister au sein des fractions lourdes. Cette idée est appuyée par les résultats d'Acevedo \textit{et al.} \cite{acevedo2004structural, gutierrez2001fractionation} qui ont montré que les asphaltènes pouvaient se diviser en deux fractions : une fraction insoluble dans le toluène, baptisée A1, qui serait rigide et plane, conformément au modèle continental, et une fraction soluble dans le toluène grace à de nombreuses interactions intermoléculaires, baptisée A2 et qui correspondrait au modèle archipel. Sur la base de ces éléments, Acevedo et son équipe ont mis en place une nouvelle représentation des agrégats d'asphaltènes. Le système colloïdal consisterait en une fraction d'asphaltènes A1, entourés par une couche d'asphaltènes A2, ces derniers aidant à la solubilisation de l'ensemble. 
	En définitive, une revue bibliographique des analyses menées à ce jour sur les asphaltènes montre qu'aucun modèle structural définitif n'a pu être validé pour ce type de systèmes. De nombreux travaux cherchent encore à clarifier l'architecture moléculaire des asphaltènes, à deux échelles distinctes. En premier lieu, l'étude de la microstructure correspond à des systèmes de poids moléculaires variant entre 500 et 10 000 Da, soit à des nano-agrégats d'asphaltènes. En second lieu, les travaux sur la macrostructure de ces entités correspondent à l'étude d'assemblages de ces nano-agrégats d'asphaltènes et sont intimement liés au milieu. 
	
	\bigskip
	
	\singlespacing
	\section{Construction de la structure moléculaire des asphaltènes}
	\spacing{1.5}
	
	La construction d'un modèle moléculaire utilisable en simulation se fait sur la bases des données expérimentales recueillies, suivant deux méthodes principales. 
	La première méthode repose sur une corrélation des résultats empiriques utilisés pour déduire des informations structurales. Dans le cas des asphaltènes, l'existence de nombreux cycles aromatiques et l'ensemble des données obtenues concernant la longueur des chaînes aliphatiques ont permis à Takanohashi \textit{et al.} d'étudier trois structures distinctes d'asphaltènes générées par la méthode de Sato par le biais de simulation en dynamique moléculaire \cite{takanohashi2004structural}. 
	La seconde méthode consiste à identifier des caractéristiques structurales propres au système ainsi que sa composition moléculaire en suivant un processus stochastique. Le modèle structural se construit par l'addition des données expérimentales et/ou par recoupement de ces résultats avec une base de données existantes. Cette méthode a été poursuivie aussi bien par Elyashberg \textit{et al.} \cite{elyashberg2008computer}, qui ont développé un algorithme permettant la construction d'une structure sur la base de données RMN, que par Todeschini \textit{et al.} \cite{todeschini1995weighted} qui ont quant à eux employé des propriétés physiques telles que l'hydrophobie, le point de fusion et le point d'ébullition. 
	La construction d'un modèle moléculaire des asphaltènes peut en dernier lieu se faire initialement de façon plus intuitive, les informations expérimentales servant ensuite à optimiser le modèle de départ \cite{faulon1996stochastic,al2012systematic,de2012monte}. 
	
	
	
	\bigskip
	
	\section{Modélisation moléculaire des asphaltènes}
	
	La chimie computationnelle a révolutionné notre façon d'appréhender la structure et la réactivité des molécules, de sorte que la simulation est désormais une des clés de voûte de l'avancée scientifique. 
	Comme nous l'avons souligné dans les paragraphes précédents, la nature et la complexité des asphaltènes rendent impossible une leur caractérisation exhaustive par le seul biais de données expérimentales. Dans ce cadre, l'intérêt des simulations est incontestable et nombreux sont les travaux qui se sont déjà penchés sur la configuration moléculaire, les interactions intermoléculaires, la stabilité ou les phénomènes d'agrégation de ces systèmes.  
	Concernant ce dernier point, la modélisation moléculaire a permis de justifier que l'agrégation des asphaltènes représentait la conformation la plus stable, sur la base d'une double approche structurale et thermodynamique révélant les interactions avec les résines, présentes au sein des fractions lourdes, et les solvant \cite{murgich1996molecular}. Selon ces travaux, l'interaction responsable de la formation comme de la stabilité des micelles formés d'asphaltènes et de résines serait la force d'attraction qui s'exerce entre les plans aromatiques. 
	D'autres études théoriques se fondent sur les modèles structuraux proposés pour les asphaltènes. Les orbitales moléculaires de dimères d'asphaltènes, envisagés successivement par les modèles continental et archipel, ont été étudiés dans une approche semi-empirique ZINDO après optimisation des structures en DFT (thérorie de la fonctionnelle de la densité). Du point de vue de la stabilité, la configuration continentale, représentées dans les simulations par des dimères empilés pour symboliser les plans d'asphaltènes propres à ce modèle, s'avère globalement plus stable que la configuration archipel \cite{alvarez2013island}. 
	
	
	\bigskip
	\singlespacing
	\section{Problèmes liés aux asphaltènes dans l'industrie pétrolière}
	\spacing{1.5}
	
	Comme évoqué en introduction du présent chapitre, les problèmes techniques induits par la présence d'asphaltènes au sein des fractions lourdes du pétrole se posent à tous les stades de la production pétrolière, de l'extraction au raffinage. 
	S'agissant de l'étape d'extraction, les asphaltènes, possédant des propriétés de sédimentation, viennent réduire la perméabilité de la roche et limiter le rendement d'extraction, lorsqu'ils ne provoquent pas le blocage de l'ensemble du pipeline et l'arrêt du procédé \cite{huang2011fundamental}. Par ailleurs, les asphaltènes ont tendance à précipiter lors du mélange de deux pétroles d'origine différente ; ce phénomène physique augmente la viscosité du fluide obtenu par mélange et le rend de fait difficile à manipuler et transporter. 
	Dans les phases de raffinage, et plus spécifiquement au cours des procédés d'hydrotraitement, la forte teneur en hétéroéléments des asphaltènes est susceptible de perturber l'action des catalyseurs, sinon de les désactiver. 
	De manière générale, les conditions physiques nécessaires aux différents stades de production pétrolière (hautes températures, pressions élevées) induisent des modifications chimiques structurales susceptibles de rendre le système instable. Dans ces conditions, les asphaltènes floculent et précipitent ; leur taille augmente et ces systèmes ont alors tendance à adhérer à la surface des pipelines \cite{broseta2000detection}. Enfin, les asphaltènes contribuent à la formation d'émulsions huile/eau stables qui affectent le procédé de séparation pétrole/eau dans la phase de récupération assistée du pétrole par l'eau \cite{kokal1999quantification}.  
	L'ensemble des problèmes induits par la présence des asphaltènes ont un impact économique majeur sur le procédé. Ces systèmes obligent à requérir à un traitement spécifique et onéreux pour le traitement des fractions lourdes du pétrole brut. 
	
	
	%Cette propriété de stabilisation d'émulsions constitue aussi un avantage dans certaines applications. 



\newgeometry{textwidth=16cm}
\chapter[Rappels théoriques : DFT]{Rappels théoriques : Méthodes de la fonctionnelle de la densité (DFT)}
\minitoc
\restoregeometry

\newpage

\section*{Introduction}
\markright{INTRODUCTION}{}

La chimie théorique est une science relativement récente, puisque ce n'est qu'en 1933 que le physicien autrichien Erwin Schr\"{o}dinger  a reçu le prix \textsc{Nobel} de physique, en commun avec Paul \textsc{Dirac}, pour ses travaux représentant, depuis, les fondements de la chimie quantique. En effet, l'équation de Schr\"{o}dinger nous prouve que la connaissance de la fonction d'onde du système donne accès, explicitement ou non, à toutes les valeurs caractéristiques du système chimique étudié. Dans l'article \textit{La situation actuelle en mécanique quantique}~\cite{schrocat} paru en 1935, la métaphore du \og \fg{chat de Schr\"{o}dinger} a largement contribué à la vulgarisation de cette science auprès du public scientifique. En effet, il y symbolise, à l'aide d'un exemple macroscopique régit par les lois de la physique classique, la philosophie de la mécanique quantique dévolue à l'étude des systèmes microscopiques. Le plus grand défi à relever ici était de faire comprendre qu'une \og mesure quantique \fg{} est une combinaison linéaire d'une somme d'états de probabilités non-nulles et non une valeur unique. L'expérience fictive consiste à placer un chat dans une boîte contenant un flacon de poison ainsi qu'une source radioactive. Lorsqu'un compteur Geiger détecte un certain seuil de radiation, un mécanisme vient briser le flacon et libère ainsi des vapeurs mortelles. Dans un raisonnement quantique, le chat est donc à la fois vivant \underline{et} mort dans la boîte tant qu'elle reste close, avec une probabilté de vie de plus en plus faible au cours du temps.

En pratique, le nombre considérable de calculs à réaliser dans le cadre de la chimie théorique lie intrinsèquement cette science au développement de l'informatique, tout aussi récent. Il s'agit même de l'une des plus grandes limites encore rencontrée de nos jours. Même si la loi de \textsc{Moore}~\cite{moore}~\footnote{La loi de \textsc{Moore} (1965), renommée première loi de \textsc{Moore} compte-tenu de l'ajustement ultérieur, énonce que la complexité des semi-conducteurs (et donc de la puissance de calcul) suit une loi exponentielle au cours du temps.}, empirique de son état, a commencé à perdre en véracité dès lors que la fréquence des processeurs (CPU) engendrait une déperdition de chaleur trop importante et commercialement non-maîtrisable ($\gtrsim$ 5 Ghz), l'apparition des processeurs multic\oe urs s'est vu être la solution la plus efficace. Après l'adaptation nécessaire des codes séquentiels en version parallèle, la démocratisation relative des clusters de calcul~\footnote{Un cluster de calcul, ou grappe de serveurs, est un ensemble de serveurs esclaves, les n\oe uds, contrôlés par un ou plusieurs serveurs maîtres, le(s) frontal(aux). Ce groupe de serveurs indépendants, fonctionnant comme un seul et même système, permet d'optimiser les ressources (processeur, mémoire vive, stockage\dots{}) et une meilleure répartition des tâches sur les différents n\oe uds.} permet à l'heure actuel le développement de codes hautement parallélisés (> 1000 processeurs). Ceci permet alors de faire moins d'approximations théoriques et donc de tendre vers des valeurs de plus en plus exactes, tout en gardant des temps de calcul raisonnables. Un engouement pour les clusters de cartes graphiques (GPU) est aussi à noter dans les domaines scientifiques où, tel que la dynamique moléculaire, les calculs sont extrêment fragmentables et peu interdépendants.\\

Dans ce chapitre, nous rappellerons dans un premier temps les fondements de la théorie de la fonctionnelle de la densité, notée DFT, par le biais de l'évolution des différents modèles qui ont été proposés. Nous verrons ensuite comment les théorèmes de \textsc{Hohenberg} et \textsc{Kohn} prouvent que la seule connaissance de la densité électronique permet de résoudre l'équation de Schr\"{o}dinger dans le cadre de la DFT. La fonctionnelle universelle $F_{HK}[\rho]$, qui permettrait une résolution exacte du problème, restant inconnue, nous aborderons dans une troisième partie l'approche KS qui contourne ce problème et légitime certaines approximations. Ces dernières donnant naissance à différents types de fonctionnelle, elles seront succintement présentées dans la pénultième partie de ce chapitre. Finalement, le développement de la méthode de calcul utilisée dans ce travail de thèse, sera présentée étape par étape en fin de chapitre. 

\newpage

\section{Les fondements de la DFT}

Contrairement aux méthodes HF, noté HF, (cf annexe~\ref{annexeHF}), et \textit{a fortiori} post-HF (cf annexe~\ref{annexemp2}) qui décrivent le système électronique par une fonction d'onde $\Psi_{(\vec{r})}$, la théorie de la fonctionnelle de la densité le décrit par la densité électronique, notée $\rho_{(\vec{r})}$, qui est liée à la fonction d'onde $\Psi_{(\vec{r})}$ par la relation suivante~:

\begin{align}
\rho_{(\vec{r})} &= \int \Psi_{(\vec{r})}^{*} \Psi_{(\vec{r})} \\
&= \int |\Psi_{(\vec{r})}^{2}| \notag
\end{align}

\begin{flushleft}
\begin{tabular}{@{}lrp{10cm}}
avec & $\vec{r}$ : & ensemble des coordonnées électroniques. 
\end{tabular}
\end{flushleft}


L'énergie de l'état fondamental est ainsi une fonctionnelle de la densité électronique, c'est-à-dire que $E_{0} = E_{(\rho)}$.

\subsection{Modèle de \textsc{Thomas-Fermi}}

Le terme d'énergie cinétique a été exprimé comme une fonctionnelle de la densité pour la première fois en 1927 par \textsc{Thomas} et \textsc{Fermi}~:

\begin{equation}
\hat{T}_{TF}[\rho] = \frac{3}{10} (3\pi)^{2/3} \int \rho_{(\vec{r})}^{5/3} .d\vec{r}
\label{ener_cin_thom_ferm}
\end{equation}

Celle fonctionnelle est alors combinée aux expressions classiques des interactions électrons-noyaux et électrons-électrons, exprimées elles aussi en fonction de la densité électronique :

\begin{equation}
E_{TH}[\rho] = T_{TH}[\rho] + V_{Ne}[\rho] + V_{ee}[\rho]
\end{equation}

\subsection{Modèle de \textsc{Thomas-Fermi-Dirac}}

Le terme d'échange, résultant du principe d'exclusion de \textsc{Pauli}, a ensuite été ajouté par Dirac en 1930 afin d'affiner le modèle :

\begin{align}
K[\rho] = E_{x}[\rho] &= \int \rho_{\vec{r}} \epsilon_{x}[\rho] .d\vec{r} \\
&= -\frac{3}{4} \left(\frac{3}{\pi}\right)^{1/3} \int \rho_{(\vec{r})}^{4/3} .d\vec{r} \notag
\end{align}

\begin{flushleft}
\begin{tabular}{@{}lrp{10cm}}
avec & $\epsilon_{X}[\rho]$ : & énergie d'échange par électron. 
\end{tabular}
\end{flushleft}

Le modèle de \textsc{Thomas-Fermi-Dirac} est défini par la combinaison de cette expression avec l'équation~\ref{ener_cin_thom_ferm} et le potentiel d'interaction électrons-noyaux $V_{Ne}[\rho]$. Notons que la corrélation électronique n'est toujours pas prise en compte dans ce modèle.

\subsection{Modèle de \textsc{Slater}}

Partant d'une approche basée sur la méthode HF, \textsc{Slater} proposa en 1951 de substituer le terme d'énergie d'échange par une fonctionnelle de la densité issue de l'énergie d'échange de Dirac. Ce terme d'échange dans le formalisme HF peut alors être généralisé en introduisant le paramètre $\alpha$ :

\begin{equation}
E_{x}[\rho] = - \frac{9\alpha}{8} \left(\frac{3}{\pi}\right)^{1/3} \int \rho_{(\vec{r})}^{4/3} .d\vec{r}
\end{equation}

Des analyses empiriques basées sur différents types de systèmes chimiques ont conduit à une valeur de $3/4$ pour $\alpha$, offrant une meilleure précision que la valeur originelle de l'expression de Dirac ($2/3$).

\section{Les théorèmes de \textsc{Hohenberg} et \textsc{Kohn}}

Tous ces modèles, qui constituent les fondements de la DFT, ne démontrent pas formellement que seul la connaisance de la densité est importante pour atteindre la valeur de l'énergie totale d'un système. C'est ainsi que \textsc{Hohenberg} et \textsc{Kohn} eurent l'idée en 1965 de démontrer, par le biais de deux théorèmes, que l'équation de Schr\"{o}dinger pouvait être résolue de façon exacte, dans le cadre de l'approximation de \textsc{Born-Oppenheimer} uniquement grâce à la densité électronique.

\subsection{Premier théorème : preuve d'existence}

Ce premier théorème énonce que l'ensemble des propriétés du système, notamment l'énérgie, peuvent être calculées à partir de la seule densité électronique de l'état fondamental. Elles peuvent donc être décrites comme une fonctionnelle de la densité électronique, et l'énergie totale s'écrit alors :

\begin{equation}
E[\rho] = F_{HK}[\rho] + \int \rho_{(\vec{r})} \nu_{ext} .d\vec{r}
\label{Hohen_Kohn}
\end{equation}
\noindent où :
\begin{align}
F_{HK}[\rho] &= T_{e}[\rho] + V_{ee}[\rho] \\
\nu_{ext} &= V_{Ne}[\rho] \notag
\end{align}

Notons que la fonctionnelle universelle $F_{HK}[\rho]$, qui regroupe les termes d'énergie cinétique des électrons et celui d'énergie potentielle d'interaction électron-électron, n'est pas liée au potentiel externe $\nu_{ext}$. L'énergie de l'état fondamental est \textit{a priori} accessible de manière exacte car cette fonctionnelle ne repose sur aucune approximation.

\subsection{Second théorème : théorème variationnel}

Basé sur l'équation~\ref{Hohen_Kohn}, \textsc{Hohenberg} et \textsc{Kohn} ont ensuite construit un principe variationnel pour déterminer la densité électronique de l'état fondamental :

\begin{equation}
E[\rho] \geq E[\rho_{0}]
\end{equation}

\begin{flushleft}
\begin{tabular}{@{}lrp{10cm}}
avec & $\rho_{0}$ : & densité électronique de l'état fondamental, \\
& $\rho$ : & densité électronique quelconque.
\end{tabular}
\end{flushleft}

Dans cette équation, à une densité d'essai $\rho$ correspond une seule énergie potentielle $\int \rho_{(\vec{r})} \nu_{ext} .d\vec{r}$ et une seule fonction d'onde $\Psi_{\rho}$. La méthode de double minimisation, ie sous contrainte de \textsc{Levy}, permet de différencier la fonction d'onde $\Psi_{\rho_{0}}$, correspondant à l'état fondamental, parmi le jeu infini des fonctions d'ondes $\Psi_{\rho}$ donnant la même densité. Ainsi, nous pouvons déterminer, parmi toutes les densités, celle qui minimisera l'énergie par la relation suivante :

\begin{equation}
E[\rho_{0}] = \min\limits_{\rho}\, (\min\limits_{\Psi\rightarrow\rho}\, (F[\rho] + \int \rho_{\vec{r}} \nu_{\vec{r}}\, .d\vec{r}\, ))
\end{equation}

Si les théorèmes de \textsc{Hohenberg} et \textsc{Kohn} démontrent une correspondance unique entre une densité $\rho_{\vec{r}}$ et la fonction d'onde $\Psi$ du système, la fonctionnelle universelle $F_{HK}[\rho]$ reste cependant inconnue.

\section{Approche KS}\label{Kohn-Sham}

Afin de contourner ce problème, \textsc{Kohn} et \textsc{Sham} substituèrent au Hamiltonien réel, décrivant un système de $n$ particules en interaction, un Hamiltonien de référence décrivant un système de $n$ particules sans interaction mais ayant la même densité que le système réel. Le problème est ainsi réduit à la résolution de $n$ équations monoélectronique couplées, analogues aux équations de HF. L'opérateur monoélectronique de Kohn-Sham $\hat{K}_{KS}$ s'exprime ainsi :

\begin{equation}
\hat{H}_{KS} = -\frac{1}{2} \nabla^{2} + \nu_{H}[\rho] + \nu_{xc}[\rho] + \nu_{ext}[\rho]
\end{equation}

\noindent où :
\begin{align}
\nu_{H}[\rho] &= \int \frac{\rho_{(\vec{r})} - \rho_{(\vec{r}')}}{|\vec{r} - \vec{r}'|} .d\vec{r}' \\
\nu_{xc}[\rho] &= \frac{\partial E_{xc}[\rho_{(\vec{r})}}{\partial\rho_{(\vec{r})}}
\end{align}

\noindent $\nu_{H}[\rho]$ et $\nu_{xc}[\rho]$ étant respectivement le potentiel de \textsc{Hartree} et le potentiel d'échange et de corrélation, dans lequel $E_{xc}[\rho_{(\vec{r})}]$ est l'énergie d'échange et de corrélation.

En définissant un potentiel fictif $\nu_{eff(\vec{r})}$ pouvant être appliqué à des systèmes sans interaction de densité $\rho$ :

\begin{equation}
\nu_{eff(\vec{r})} = \nu_{H}[\rho] + \nu_{xc}[\rho] + \nu_{ext}[\rho]
\end{equation}

\noindent nous introduisons alors un jeu d'orbitales $\psi_{(\vec{r})}$, appelées orbitales de Kohn-Sham, et nous obtenons un jeu d'équations aux valeurs propres :

\begin{equation}
\hat{H}_{KS} \psi_{i(\vec{r})} = \epsilon_{i} \psi_{i(\vec{r})}
\end{equation}

Comme dans le cas de la méthode HF, l'énergie du système peut être minimisée en résolvant ce jeu d'équations de façon auto-cohérente grâce à l'utilisation des orbitales de KS. L'énergie du système est alors donnée par :

\begin{equation}
E_{KS}^{tot}[\rho] = T_{s}[\rho] + J[\rho] + E_{xc}[\rho] + \int V_{ext(\vec{r})}\rho_{(\vec{r})} .d\vec{r}
\end{equation}

\begin{flushleft}
\begin{tabular}{@{}lrp{10cm}}
avec & $T_{s}[\rho]$ : & énergie cinétique des électrons sans interaction, \\
& $J[\rho]$ : & énergie d'interaction coulombienne entre les électrons, \\
& $E_{xc}[\rho]$ : & énergie d'échange et de corrélation, \\
& $\int V_{ext(\vec{r})}\rho_{(\vec{r})} .d\vec{r}$ : & énergie d'interaction avec le potentiel externe. 
\end{tabular}
\end{flushleft}

D'après les théorèmes de \textsc{Hohenberg} et \textsc{Kohn}, $E_{KS}^{tot}[\rho]$ doit être égale à l'énergie totale du système réel $E_{reel}^{tot}[\rho]$, qui peut être décrite comme suit :

\begin{equation}
E_{reel}^{tot}[\rho] = T[\rho] + V_{ee}[\rho] + \int V_{ext(\vec{r})}\rho_{(\vec{r})} .d\vec{r}
\end{equation}

Le terme d'échange corrélation peut ainsi être explicité comme étant la somme de la correction à l'énergie cinétique due à l'interaction entre électrons ($T[\rho] - T_{s}[\rho]$) et les corrections non classiques à la répulsion électron-électron ($V_{ee}[\rho] - J[\rho]$) :

\begin{equation}
E_{xc}[\rho] = T[\rho] - T_{s}[\rho] + V_{ee}[\rho] - J[\rho]
\end{equation}

La théorie de la fonctionnelle de la densité de KS a connu un grand succès parmi les méthodes de calcul appliquées aux grands systèmes dû à son ratio coût calculatoire/performance très intéressant. C'est pour cet avantage indéniable qu'elle sert de base à de nombreuses évolutions de la DFT.

\section{Les différentes classes de fonctionnelle}

Formellement, la DFT est donc une méthode exacte, dans la limite de la connaissance de la fonctionnelle universelle $F_{HK}[\rho]$ ou de sa fonctionnelle exacte d’échange et de corrélation $F_{xc}[\rho]$. Malheureusement, la forme exacte de l’énergie d’échange et de corrélation est inconnue, si bien qu’il est nécessaire de faire des approximations. Dans la pratique, l’énergie d’échange et de corrélation $E_{xc}[\rho]$ est calculée à l’aide de fonctionnelles d’échange et de corrélation, définies comme suit :

\begin{equation}
\int F_{xc}[\rho_{(\vec{r})}].d\vec{r} = E_{xc}[\rho_{(\vec{r})}]
\end{equation}

L’énergie d’échange et de corrélation est généralement séparée en deux termes distincts, l’un d’échange $E_{x}[\rho]$ et l’autre de corrélation $E_{c}[\rho]$ :

\begin{equation}
E_{xc}[\rho_{(\vec{r})}] = E_{x}[\rho_{(\vec{r})}] + E_{c}[\rho_{(\vec{r})}]
\end{equation}

Plusieurs fonctionnelles ont donc été développées pour traiter chacune de ces contributions, de façon simultanée ou indépendante. Nous allons ici donner un bref aperçu -- non exhaustif -- des différentes familles de fonctionnelles, en suivant le critère de classification donné par \textsc{Perdew} et couramment dit \og échelle de Jacob de \textsc{Perdew} \fg{}. Il a proposé de classer les fonctionnelles en fonction du degré d’information non local contenu dans leur forme analytique. Au premier échelon se trouvent les fonctionnelles qui dépendent uniquement de la densité électronique, dites fonctionnelles LDA (\og Local Density Approximation \fg{}). Viennent ensuite les fonctionnelles corrigées par gradient GGA (\og Generalized Gradient Approximation \fg{}) dans lesquelles la non-localité est introduite grâce à leur dépendance par rapport au gradient de la densité. A ce même niveau se trouvent également les fonctionnelles de type méta-GGA, dépendant aussi de l’énergie cinétique (calculée à partir des orbitales moléculaires remplies). Il faut noter que, d’un point de vue mathématique, toutes ces familles de fonctionnelles sont strictement locales. Pour parvenir à des fonctionnelles véritablement non locales il faut encore monter d’un cran dans l’échelle de \textsc{Perdew}, jusqu’aux fonctionnelles dites hybrides, où la présence d’un pourcentage (variable) d’échange HF calculé en utilisant les orbitales KS, permet d’introduire un véritable terme non local. Plus récemment une autre famille de fonctionnelles hybrides a été développée (dite à longue portée ou encore à séparation de portée) dans laquelle le pourcentage d’échange calculé de façon HF n’est pas constant mais dépend de la distance inter-électronique. Enfin, des fonctionnelles non locales montrant une dépendance explicite des orbitales KS occupées et vacantes représentent le dernier niveau dans l’échelle de \textsc{Perdew}.

\subsection{Local Density approximation (LDA)}\label{lda}

L’approximation locale de la densité est l’approximation la plus grossière, dans laquelle l’énergie d’échange et de corrélation $E_{xc}$ n’est fonction que de la seule densité électronique :

\begin{equation}
E_{xc}^{LDA}[\rho_{(\vec{r})}] = \int \rho_{(\vec{r})} \epsilon_{xc}[\rho_{(\vec{r})}].d\vec{r}
\end{equation}

\noindent où la valeur de $\epsilon_{xc}$ à une position $\vec{r}$ est calculée exclusivement à partir de la valeur de la densité électronique $\rho$ à cette position. En pratique, $\epsilon_{xc}$ décrit l’énergie d’échange et de corrélation par particule pour un gaz uniforme d’électrons de densité $\rho$. Le potentiel d’échange et de corrélation correspondant est alors :

\begin{equation}
\nu_{xc(\vec{r})}^{LDA} = \epsilon_{xc}[\rho_{(\vec{r})}] + \rho_{(\vec{r})} \frac{\partial \epsilon_{xc}[\rho_{(\vec{r})}]}{\partial \rho}
\end{equation}

Malgré le fait que les résultats obtenus soient généralement en bon accord avec les résultats expérimentaux, notamment au niveau de la géométrie et de la structure électronique, cette approximation reste une approximation locale, c'est-à-dire dans laquelle on ne tient pas compte de l’inhomogénéité de la densité électronique.

\subsection{Generalized Gradient Approximation (GGA)}

L’idée directrice de l’approximation du gradient généralisé est donc de mieux tenir compte de l’inhomogénéité de la densité du système, en introduisant une dépendance de la densité $\rho$ à son gradient $\nabla \rho$. L’expression générale des fonctionnelles de type GGA est la
suivante :

\begin{equation}
E_{xc}^{GGA}[\rho_{(\vec{r})}] = A_{x} \int \rho_{(\vec{r})}^{4/3} E^{GGA}(s) .d\vec{r}^{3}
\end{equation}

\noindent où $s$, gradient de la densité réduite, est tel que :

\begin{equation}
s = \frac{|\nabla \rho_{(\vec{r})}|}{2 k_{F} \rho_{(\vec{r})}}
\end{equation}

\noindent avec $k_{F} = (3 \pi^{2} \rho_{(\vec{r})})^{1/3}$. Ainsi, on fait apparaître avec $s$ un terme quasi-local, dépendant non seulement de la densité électronique mais également de son gradient au voisinage de $\vec{r}$.

Un exemple de fonctionnelle GGA est celle de \textsc{Perdew}, \textsc{Burke} et \textsc{Ernzherhof}, notée PBE~\cite{pbe}.

\subsection{Fonctionnelles hybrides}

La dernière grande famille de fonctionnelles est celle des fonctionnelles hybrides. L’idée consiste à introduire une fraction d’échange calculée de façon exacte (telle qu’utilisée dans la méthode HF) dans une fonctionnelle d’échange de type GGA. L’expression de $E_{xc}$ devient alors :

\begin{equation}
E_{xc}^{hybride}[\rho_{(\vec{r})}] = (1- \alpha) E_{xc}^{GGA}[\rho_{(\vec{r})}] + \alpha E_{xc}^{HF}[\rho_{(\vec{r})}]
\end{equation}

\noindent où le coefficient de la combinaison $\alpha$ donne le rapport HF/DFT.
PBE0~\cite{pbe0} Celle-ci présente 25\% d’échange HF~\cite{methodehf} dans une fonctionnelle GGA de type PBE~\cite{pbe} :

\begin{equation}
E_{xc}^{PBE0} = E_{xc}^{PBE} + \frac{1}{4} (E_{x}^{HF} - E_{x}^{PBE})
\end{equation}

Elle a l’avantage d’être non paramétrée (car le pourcentage d’HF inclus n’est pas empirique mais basé sur des arguments de théorie perturbationnelle), et de fournir des résultats très précis, que ce soit au niveau du calcul des structures moléculaires, des structures électroniques ou encore des propriétés spectroscopiques~\cite{pbe0}.

Nous mentionnerons également la fonctionnelle hybride la plus utilisée pour traiter des systèmes moléculaires, à savoir la fonctionnelle B3LYP~\cite{Bcxthermo}. Comme son nom l’indique, elle inclus trois paramètres et est basée sur les fonctionnelles GGA d’échange et de corrélation de \textsc{Becke} (B)~\cite{B88} et \textsc{Lee}, \textsc{Yang} et \textsc{Parr} (LYP)~\cite{lyp}, suivant l’expression :

\begin{equation}
E_{xc}^{B3LYP} = (1-a) E_{x}^{LSDA} + a E_{x}^{HF} + b \Delta E_{x}^{B} + (1-c) E_{c}^{LSDA} + c E_{c}^{LYP}
\label{B3LYP}
\end{equation}

\noindent avec $a$, $b$ et $c$ fixés respectivement à 0,20 , 0,72 et 0,81.


\subsection{Nécessité d'une fonctionnelle « longue portée »}

Bien que calculatoirement très pratiques, les méthodes basées sur la fonctionnelle de la densité connaissent des limites évidentes lorsqu'il s'agit de traduire correctement les interactions à longue distance. Ces interactions se révèlent pourtant être à l'origine de nombreuses propriétés chimiques et physiques des matériaux. En effet, elles sont par exemple responsables de la cohésion des cristaux liquides et moléculaires, des propriétés d'adhésion et de physisorption sur des surfaces, de la spécificité de site de l'ADN mais aussi de la cohésion dans les systèmes lamellaires, et plus particulièrement dans le cas des matériaux graphitiques qui nous intéresse présentement. La compréhension et la bonne modélisation de ce type d'interactions est donc naturellement devenu un axe important dans le domaine de la physique et de la chimie quantique. 

%!!!!!!!!!!!! tensioactif, température de fusion, ébullition, solvatation des ions, capillarité...


\subsubsection{Mise en évidence des forces de VdW}

L'existence de phases naturellement condensées comme les liquides et les solides moléculaires a permis de démontrer de façon empirique la présence de forces attractives entre les molécules. Avec un raisonnement analogue, des forces répulsives à faibles distances ont été mises en avant grâce aux propriétés d'incompressibilité infinie de ces mêmes systèmes chimiques.

Johannes Diderik VdW fut le premier en 1873 à intégrer cette notion de forces attractives et répulsives dans le modèle du \og gaz parfait \fg{} (\ref*{GP}) en introduisant comme suit deux termes correctifs dans sa loi des \og gaz réels \fg{} (\ref*{GR})~:

\begin{align}
PV &= nRT \label{GP} \\
\left(P+\frac{n^{2}a}{V^{2}}\right)(V-nb) &= nRT \label{GR}
\end{align}

\begin{flushleft}
\begin{tabular}{@{}lrp{10cm}}
avec & $P$ : & pression (Pa), \\
& $V$ : & volume du gaz ($m^{3}$), \\    % bien mettre le m cube en lettre droite
& $n$ : & quantité de matière de gaz (mol), \\
& $R$ : & constante des gaz parfaits, \\
& $T$ : & température (K), \\
& $a$ et $b$ : & constantes ajustables positives. 
\end{tabular}
\end{flushleft}

Cette équation d'état est beaucoup plus proche des valeurs expérimentales, notamment aux fortes pressions. En effet, le volume d'exclusion $nb$ permet de tenir compte du volume propre occupé par les molécules. Nous rappelerons que dans le modèle des gaz parfaits, dans lequel les molécules sont assimilées à des points ponctuels, ce volume est négligé. Notons que la pression est aussi corrigée par le terme $\frac{n^{2}a}{V^{2}}$ afin de tenir compte de la diminution de pression au voisinage des parois, du fait de l'existence d'interactions attractives entre les molécules.

L'écriture de la loi des \og \fg{gaz réels} par VdW a réellement initié la volonté de comprendre la nature de ces interactions et a naturellement conduit à de nombreuses études concernant les différents systèmes chimiques. Des efforts sont particulièrement portés sur la modélisation de forces de cohésions universelles, et non pas liées à une seule problèmatique.

\subsubsection{Natures des interactions à modéliser}

Avant de pouvoir établir une théorie, il faut avant tout comprendre la nature du phénomène. Les atomes et molécules étant composés de particules naturellement chargées, ie de noyaux et d'électrons, ils interagissent donc par des forces de \textsc{Coulomb}. Celles-ci se décomposent en plusieurs éléments selon leur nature~: l'électrostatique, l'induction (ou polarisation), la dispersion et l'échange.
L'effet du terme électrostatique représente la répulsion ou l'attraction entre des charges respectivement de même signe ou de signes opposés (figure~\ref{chargecharge}). Son énergie $E_{Coulomb}$ a été définie par Charles-Augustin \textsc{Coulomb} en 1785 comme suit :

\begin{figure}[h]
\centering
\begin{tikzpicture}[scale=0.7, every node/.style={scale=0.7}]
%\nuage{charge+}{-1.5}{0}{}
%\nuage{charge-}{1.5}{0}{}
\draw [latex-latex, thick] (-0.8,0) --++ (1.6,0) node [above, midway] {$F_{Coulomb}$} ;
\end{tikzpicture}
\caption{Interaction électrostatique entre charges.}
\label{chargecharge}
\end{figure}

\begin{equation}
E_{Coulomb} = \frac{1}{4 \pi \epsilon_{0}} * \frac{q_{1} q_{2}}{r}
\end{equation}

\begin{flushleft}
\begin{tabular}{@{}lrp{10cm}}
avec & $q_{1}$ et $q_{2}$ : & charges respectives des particules 1 et 2, \\
& $r$ : & distance entre les particules. 
\end{tabular}
\end{flushleft}

L'induction est quant à elle due à la déformation de la densité électronique d'un atome ou d'une molécule par l'effet du champ électrique d'une molécule voisine. Ces deux contributions sont très bien définies en physique classique, contrairement à la dispersion et l'échange, qui sont liés à des effets quantiques.
En effet, ce sont les phénomènes de fluctuations quantiques de la distribution des charges qui sont à l'origine du terme de dispersion, et la contribution d'échange est dûe au principe d'exclusion de \textsc{Pauli} qui impose que deux électrons ne peuvent pas posséder le même état quantique au même moment. Dans le cas d'une paire d'atomes en interaction ayant leurs couches électroniques partiellement occupées, cette contribution est positive et engendre une liaison chimique liante forte. À l'inverse, lorsqu'il s'agit de systèmes électroniques à couche fermée, elle devient un terme de répulsion à courte distance responsable du phénomène d'exclusion stérique. Ce phénomène étant inclu dans la loi des \og \fg{gaz réels} de VdW (équation~\ref{GR}) sous la forme du volume d'exclusion $nb$.  


D'une manière plus générale, c'est-à-dire en dépassant le cadre des gaz, les interactions de VdW sont générées par les fluctuations de distributions de charge des atomes et molécules et conduisent aux équations suivantes, où l'énergie est exprimée en Joules :

\begin{align}
E_{Keesom} &= - \frac{1}{r^{6}} \left(\frac{\mu_{1}^{2}\mu_{2}^{2}}{3(4\pi \epsilon_{0} \epsilon)^{2} k_{B}T}\right) \\
E_{Debye} &= - \frac{1}{r^{6}} \left(\frac{\mu_{1}^{2}\alpha_{2}+\mu_{2}^{2}\alpha_{1}}{(4\pi \epsilon_{0} \epsilon)^{2}}\right) \\
E_{London} &= - \frac{1}{r^{6}} \left(\frac{3}{4}\frac{h\nu\alpha_{1}\alpha_{2}}{(4\pi \epsilon_{0})^{2}}\right)
\end{align}

\begin{flushleft}
\begin{tabular}{@{}lrp{10cm}}
avec & $\mu_{1}$ et $\mu_{2}$ : & moments dipolaires respectifs des particules 1 et 2, \\
& $\alpha_{1}$ et $\alpha_{2}$ : & polarisabilités respectives des particules 1 et 2, \\
& $\epsilon_{0}$ : & permittivité diélectrique du vide, \\
& $\epsilon$ : & permittivité diélectrique du milieu, \\
& $k_{B}$ : & constante de \textsc{Boltzmann}, égale à la constante des gaz parfaits $R$ divisée par le nombre d'\textsc{Avogadro} $\mathcal{N}\!a$, \\
& $T$ : & température, \\
& $r$ : & distance entre les particules. \\
\end{tabular}
\end{flushleft}

L'énergie de \textsc{Keesom} (effet d'orientation) représente l'énergie d'interaction entre deux dipôles électrostatiques, c'est-à-dire deux molécules ayant un moment dipolaire\footnote{Le moment dipôlaire d'une molécule résulte d'une répartition hétéroclite de charges électriques, telle que le barycentre des charges positives (noyaux) ne coïncide pas avec celui des charges négatives (électrons), les électrons étant en effet attirés par l'atome le plus électronégatif de la liaison.} $\mu{}$ non nul (figure~\ref{figKeesom}). Il s'agit concrètement de l'attraction mutuelle de deux dipôles permanents qui est d’autant plus forte que les moments dipolaires sont élevés (grande charge et petite taille de la molécule) et que la température est basse.

\begin{figure}[h]
\centering
\begin{tikzpicture}[scale=0.7, every node/.style={scale=0.7}]
%\nuage{nnul}{-2}{0}{permanent}
%\nuage{nnul}{2}{0}{permanent}
\draw [latex-latex, thick] (-1,0) --++ (1.4,0) node [above, midway] {$F_{Keesom}$} ;
\end{tikzpicture}
\caption{Interaction entre dipôles électrostatiques.}
\label{figKeesom}
\end{figure}

Celle de \textsc{Debye} (effet d'induction) résulte de la déformation du nuage électronique d'une molécule, d'un atome ou d'un ion, par action du champ électrique engendré par le moment dipolaire d'une molécule voisine (figure~\ref{figDebye}). Il en résulte ainsi un moment dipolaire induit. Elle est souvent nommé interaction dipôle {permanent-dipôle{ induit et fait intervenir le moment dipolaire $\mu$ et la polarisabilité\footnote{La polarisabilité est la capacité du nuage électronique à se déformer sous l'action d'un champ électrique. Le barycentre des charges négatives (électrons) étant légèrement décalé par rapport à celui des charges positives (noyaux) sous l'effet du champ, un moment électronique induit $\vec{m}_{e}$ apparaît, engendrant la notion de polarisabilité $\alpha$.} $\alpha$ des molécules concernées. Le nominateur $\mu_{1}^{2}\alpha_{2}+\mu_{2}^{2}\alpha_{1}$ décrit l'interaction lorsque les deux molécules sont polaires (cas A) mais s'écrit naturellement $\mu_{1}^{2}\alpha_{2}$ quand la seconde molécule est apolaire (cas B).

% mettre les deux cas dans la figure !
\begin{figure}[h]
\centering
\begin{tikzpicture}[scale=0.7, every node/.style={scale=0.7}] %%%%%%%%%%%%%% MISE A L'ÉCHELLE DE LA POLICE AUSSI !!!!
\draw [dashed] (0,-4) -- (0,2) ;
\begin{scope}[xshift=-5.4cm]
\draw (-0.9,1.5) --++ (1.3,0) node [above, midway] {cas A} ;
%\nuage{nnul}{-3.5}{0}{permanent}
%\nuage{nnul}{3.5}{0}{permanent}
\draw (-0.3,0) node {\shortstack{sans\\ interaction}} ;
%\nuage{nnul}{-2}{-3}{induit}
%\nuage{nnul}{2}{-3}{induit}
\draw [latex-latex, thick] (-1,-3) --++ (1.4,0) node [above, midway] {$F_{Debye}$} ;
\end{scope}
\begin{scope}[xshift=6cm]
\draw (-0.9,1.5) --++ (1.3,0) node [above, midway] {cas B} ;
%\nuage{nnul}{-3.5}{0}{permanent}
%\nuage{nul}{3.5}{0}{}
\draw (-0.3,0) node {\shortstack{sans\\ interaction}} ;
%\nuage{nnul}{-2}{-3}{permanent}
%\nuage{nnul}{2}{-3}{induit}
\draw [-latex, thick] (-1,-3) --++ (1.4,0) node [above, midway] {$F_{Debye}$} ;
\end{scope}
\end{tikzpicture}
\caption{Interaction entre une molécule polaire et une seconde polarisable.}
\label{figDebye}
\end{figure}

Finalement, l'énergie de \textsc{London} (effet de dispersion), qui est la plus importante en terme de grandeur, représente l'interaction entre deux dipôles instantanés (figure~\ref{figLondon}). Par définition, une molécule apolaire possède un moment dipolaire moyen nul mais la combinaison des mouvements des noyaux et des électrons fait qu'il existe malgré tout un moment dipolaire instantané. Cette interaction est d'autant plus forte que les deux molécules sont facilement polarisables, donc d'autant plus forte que leur taille est importante. Les forces de \textsc{London} étant présentes entre toutes les particules, quelle que soit leur nature, ce sont principalement elles qui permettent la cohésion de la matière dans l'univers.

Notons que les énergies de VdW varient en fonction de l'inverse de la distance à l'ordre 6.

\begin{figure}[h]
\centering
\begin{tikzpicture}[scale=0.7, every node/.style={scale=0.7}]
%\nuage{nul}{-3.5}{0}{}
%\nuage{nul}{3.5}{0}{}
\draw (-0.3,0) node {\shortstack{sans\\ interaction}} ;
%\nuage{nnul}{-3.5}{-3}{instantanne}
%\nuage{nul}{3.5}{-3}{}
\draw (-0.3,-3) node {\shortstack{sans\\ interaction}} ;
%\nuage{nnul}{-2}{-6}{instantanne}
%\nuage{nnul}{2}{-6}{induit}
\draw [latex-latex, thick] (-1,-6) --++ (1.4,0) node [above, midway] {$F_{London}$} ;
\end{tikzpicture}
\caption{Interaction entre deux dipôles instantanés.}
\label{figLondon}
\end{figure}


Il existe toutefois d'autres types d'interactions électrostatiques que celles décrites par le modèle des forces de VdW. En effet, similaire dans l'esprit à l'énergie de \textsc{Keesom}, l'énergie d'interaction entre un ion et un dipôle permanent est donné par la formule suivante :

\begin{equation}
E_{ion-dipôle} = - \frac{1}{r^{2}} \frac{\mu_{1}q_{2}}{4\pi \epsilon_{0} \epsilon}
\end{equation} 

Il s'agit là encore de l'interaction positive entre une espèce chargée, anion ou cation, et une molécule possédant un moment dipolaire non-nul (\ref*{figiondipole}). Ce phénomène est notamment à l'origine de la dissolution des espèces ioniques (ex~:~NaCl) dans un solvant polaire, de l'étape de solvatation qui suit, puis de la bonne dispersion des charges en solution.

\begin{figure}[h]
\centering
\begin{tikzpicture}[scale=0.7, every node/.style={scale=0.7}]
%\nuage{cation}{-2}{0}{}
\%nuage{nnul}{2}{0}{permanent}
\draw [latex-latex, thick] (-1.4,0) --++ (1.8,0) node [above, midway] {$F_{ion-dipole}$} ;
\end{tikzpicture}
\caption{Interaction entre une espèce chargée et une molécule polaire.}
\label{figiondipole}
\end{figure}

De plus, notons aussi la spécificité de la liaison hydrogène qui est une liaison chimique non covalente, de type dipôle-dipôle. Lorsqu'un hétéroatome possédant au moins une paire libre est suffisamment électronégatif (ex : O, N, F), il vient se positionner aux abords d'un hydrogène acide porté par un autre atome fortement électronégatif afin d'en stabiliser la charge partielle $\delta (+)$ ainsi créée.

Bien que de la même famille que les forces de VdW , ie électrostatique, les liaisons hydrogènes s'en distinguent par une intensité environ dix fois supérieure. Elles restent toutefois une vingtaine de fois plus faibles qu'une liaison covalente. La distance moyenne entre les deux hétéroatomes est de l'ordre de 2,5 \AA .

Dans le cadre de cette thèse, ce sont principalement les interactions entre systèmes conjugués, ie interactions $\pi-\pi$, qu'il sera nécessaire de traduire. Comme nous pouvons le voir dans le cas simple d'un dimère de benzène représenté en figure~\ref{pistackbenz}, l'interaction positive se fait entre les liaisons $\sigma$ et les liaisons $\pi$ alors que les nuages électroniques des liaisons $\pi$ se repoussent naturellement, dû à leurs charges négatives.

Notons que nous retrouvons ce phénomène sous des aspects intramoléculaires, comme par exemple dans le cas de l'hyperconjugaison $\sigma-\pi$ qui vient stabiliser certaines conformations du toluène.

\begin{figure}[h]
\centering
\begin{tikzpicture}[scale=0.7, every node/.style={scale=0.7}]
\shade [shading=ball, ball color=gray, opacity=0.5] (0,0.8) ellipse (2.2cm and 0.6cm) ;
\draw (-2,1.2) --++ (1,0.5) --++ (2,0) --++ (1,-0.5) ;
\fill (-2,1.2) --++ (1,-0.6) --++ (2,0) --++ (1,0.6) --++ (-1,-0.5) --++ (-2,0) -- cycle ;
\draw [dashed] (0,1.2) ellipse (1.4cm and 0.35cm) ;
\shade [shading=ball, ball color=gray, opacity=0.5] (0,1.6) ellipse (2.2cm and 0.6cm) ;
\shade [shading=ball, ball color=gray, opacity=0.5] (0,-1.6) ellipse (2.2cm and 0.6cm) ;
\draw (-2,-1.2) --++ (1,0.5) --++ (2,0) --++ (1,-0.5) ;
\fill (-2,-1.2) --++ (1,-0.6) --++ (2,0) --++ (1,0.6) --++ (-1,-0.5) --++ (-2,0) -- cycle ;
\draw [dashed] (0,-1.2) ellipse (1.4cm and 0.35cm) ;
\shade [shading=ball, ball color=gray, opacity=0.5] (0,-0.8) ellipse (2.2cm and 0.6cm) ;
\draw [latex-latex, thick] (-2.2,-1.2) ..controls +(-1.7,0) and +(-1.5,0).. (-2.4,0.8) node [above, midway, rotate=90, yshift=3mm] {interaction attractive $\sigma - \pi$} ;
\draw [latex-latex, thick] (-2.2,1.2) ..controls +(-1.7,0) and +(-1.5,0).. (-2.4,-0.8) ;
\draw [latex-latex, thick] (2.4,-0.8) ..controls +(1.5,0) and +(1.5,0).. (2.4,0.8) node [below, midway, rotate=90, yshift=-2mm] {interaction répulsive $\pi - \pi$} ;
\end{tikzpicture}
\caption{$\pi$-stacking dans le cas d'un dimère de benzène}
\label{pistackbenz}
\end{figure}

Même si la théorie de la fonctionnelle de la densité connaît un large succès tant elle arrive désormais bien à traduire les phénomènes de liaisons chimiques, les structures géométriques et même la cohésion des solides moléculaires et cristallins, il reste toujours l'obstacle des systèmes chimiques où les forces de VdW sont prédominantes. En effet, les effets de corrélation électronique des forces de dispersion étant purement non-locaux, l'approximation locale ou non-locale qui fait le fondement de la DFT restera problèmatique. Se pose alors la question de savoir comment modéliser ces types d'interaction de façon idiomatique. Nous allons voir que l'élaboration d'une fonctionnelle hybride à longue portée est capable de répondre à cette problèmatique.

\section[LC-DFT-D hybride : WBXD]{Construction d'une LC-DFT-D hybride : cas de la WBXD}

Les DFT hybrides avec correction à longue portée basées sur la théorie KS ont naturellement rencontré un grand engouement puisque la précision apportée n'accroît pas le coût calculatoire par rapport aux DFT hybrides.

\subsection{B88}

Comme nous l'avons vu dans le cadre des approximations de la fonctionnelle de la densité, notées DFAs\footnote{\og Density Functional Approximations \fg{} }, la décroissance en exponentielle du potentiel d'échange-corrélation, au lieu d'être en $1/r$, engendre une mauvaise représentation des interactions à longue distance. Cette erreur, nommée erreur d'auto-interaction (SIE, pour \og self-interaction error \fg{}), est liée au fait que ces approximations, basées sur la densité de spin locale (LSDA, pour \og local spin density approximation \fg{}), décrivent mal l'état fondamental qui devrait être, dans le cadre de la DFT pure, strictement sans auto-interaction.     
C'est pourquoi, afin d'introduire un effet non-local de l'échange-corrélation dans le modèle KS-DFT (partie~\ref{Kohn-Sham}), \textsc{Becke} proposa en 1988 d'incorporer dans sa fonctionnelle d'échange B88~\cite{B88} une petite part d'echange exact HF. 

Dans le cadre général des DFAs, l'énergie d'échange-corrélation s'écrit donc :

\begin{equation}
E_{xc} = c_{x}E_{x}^{HF} + E_{xc}^{DFA}
\label{xcB88}
\end{equation}

\noindent où $c_{x}$ prend généralement des valeurs comprises entre 0,2 et 0,25~\cite{Bcxthermo} pour les données thermodynamiques et entre 0,4 et 0,6~\cite{Bcxcine} pour les études cinétiques.

Basée sur ce modèle, la désormais bien connue DFT hybride B3LYP~\cite{Bcxthermo} (équation~\ref{B3LYP}) donne des résultats comparables à ceux obtenus à partir de la théorie perturbative \textsc{M\o ller-Plesset} à l'ordre 2~\cite{MP2}, noté MP2 (cf annexe~\ref{annexemp2}), souvent utilisée comme référence, dans le cadre de systèmes fortement liés. Depuis, de nombreuses recherches ont porté sur l'amélioration constante de ce potentiel d'échange-corrélation $E_{xc}[\rho]$.

\subsection{B97}

Une avancée significative a de nouveau été faite par \textsc{Becke} en 1997 dans le domaine des KS-DFT. Par une méthode similaire à la combinaison linéaire d'orbitales atomiques, notée LCAO\footnote{\og Linear Combination of Atomic Orbitals \fg{}.}, il a proposé un modèle mathématique basé sur l'approximation de densité de spin local (LSDA), sa première dérivée et une petite fraction d'échange HF pour décrire le potentiel d'échange-corrélation $E_{xc}[\rho]$. Une optimisation systématique des coefficients linéaires à partir d'un jeu classique de données expérimentales a conduit à l'apparition de la méthode B97~\cite{B97}. La base de données contient notamment des valeurs relatives à l'interaction entre systèmes conjugués.

Cette méthodologie a été reprise par F. A. \textsc{Hamprecht} etal, P. J. \textsc{Wilson} etal et T. W. \textsc{Keal} etal pour respectivement conduire à la B97-1~\cite{B97a} (1998), la B97-2~\cite{B97b} (2001) et la B97-3~\cite{B97c} (2005). Il s'agissait alors de réoptimisations des coefficients linéaires par rapport à d'autres bases de données expérimentales plus complètes.

Mais cette reparamétrisation empirique du terme d'échange-corrélation ne résout pas le problème de sa non-décroissance en $1/r$. La prise en compte totale du terme d'échange HF $E_{x}^{HF}$ ($c_{x}$=1 dans l'équation~\ref{xcB88}) pourait résoudre ce problème mais cela serait incompatible avec le terme de corrélation DFA $E_{c}^{DFA}$. En effet, il existerait alors une mauvaise compensation des erreurs respectives.

\subsection{WB}

L'idée de séparer le traitement des interactions courtes (SR, pour \og short range \fg{}) et longues portées (LR, pour \og long range \fg{}) s'est alors présentée comme le choix le plus évident, aussi bien au niveau de la compréhension des phénomènes que mathématiquement parlant. Nous pouvons ainsi traiter séparément à l'aide d'une fonction erreur $(erf)$ les interactions à courtes distances par une fonctionnelle de la densité et celles longue distance par une fonction d'onde. Ce principe conduit naturellement à l'élaboration d'une fonctionnelle hybride à séparation de portée. L'introduction de la fonction erreur, avec un paramètre libre, permet de contrôler le rayon d'action des interactions de courte-portée.

La première proposition faite par \textsc{Iikura} etal\cite{iiakura} a été de traiter la partie d'échange LR par la théorie HF alors que la partie SR est approximée par une DFA; le terme de corrélation est quant à lui le même que celui de \textsc{Coulomb}, quelle que soit la distance :

\begin{equation}
E_{xc}^{LC-DFA} = E_{x}^{LR-HF} + E_{x}^{SR-DFA} + E_{c}^{DFA}
\end{equation}

Ce schéma de séparation de portée a l'avantage de conduire à des temps de calcul très proches des DFT hybrides, mais il reste à développer une fonctionnelle d'échange SR précise et une fonctionnelle de corrélation qui soit entièrement compatible entre elles.

Le type d'opérateur de coupure le plus utilisé dans le cadre des LC-DFT hybrides est la fonction d'erreur standard $(erf)$ :

\begin{equation}
\frac{1}{r} = \frac{erf(\omega r_{12})}{r_{12}} + \frac{erfc(\omega r_{12})}{r_{12}}
\label{erf}
\end{equation}

\begin{flushleft}
\begin{tabular}{@{}lrp{10cm}}
avec & $\frac{erf(\omega r_{12})}{r_{12}}$ : & interaction de courte portée, \\
& $\frac{erfc(\omega r_{12})}{r_{12}}$ : & interaction complémentaire, \\
& $r_{12}$ : & distance entre les particules 1 et 2, \\
& $\omega$ : & paramètre contrôlant la séparation.
\end{tabular}
\end{flushleft}

Notons que l'introduction du paramètre $\omega$, qui s'exprime comme l'inverse d'une distance, permet de donner un sens physique à cette valeur, en cela qu'il est étroitement lié à une longueur caractéristique de la séparation.
Naturellement, il existe différents types de fonctions erreur $(erf)$ afin de faciliter son intégration mathématique dans les codes de calculs. Dans le cas de la WB\cite{WB97X} et, par conséquent, des fonctionnelles WBX et WBXD, c'est la fonction $erf/erfc$ qui a été choisie par Jeng-Da \textsc{Chai} et Martin \textsc{Head-Gordon} dans leurs travaux. \\

Le choix des auteurs s'est porté sur un terme d'échange exact HF longue portée $E_{x}^{LR-HF}$, calculé à partir des spin-orbitales occupées $\phi_{i \sigma}(r)$, et une forme analytique du terme d'échange $E_{x}^{SR-DFA}$ obtenue par l'intégration du carré de la matrice densité LSDA :

\begin{align}
E_{x}^{LR-HF} &= -\frac{1}{2} \sum_{\sigma} \sum_{ij}^{occ.} \iint \phi_{i \sigma}^{*}(r_{1}) \phi_{j \sigma}^{*}(r_{1}) \frac{erf(\omega r_{12})}{r_{12}} \phi_{i \sigma}(r_{2}) \phi_{j \sigma}(r_{2}).dr_{1}.dr_{2}, \\
E_{x}^{SR-LSDA} &= \sum_{\sigma} \int \underbrace{-\frac{3}{2}\left(\frac{3}{4\pi}\right)^{1/3}\rho_{\sigma}^{4/3} (r) F(a_{\sigma})}_{e_{x \sigma}^{SR-LSDA} (\rho_{\sigma}) .dr}.
\end{align}

\noindent où :
\begin{align}
k_{F \sigma}&=(6\pi^{2}\rho_{sigma}(r))^{1/3},\nonumber\\
F(a_{\sigma})&=1-\frac{8}{3}a_{\sigma}\left[\sqrt{\pi}\: erf\left(\frac{1}{2a_{\sigma}}\right)-3a_{\sigma}+4a_{\sigma}^{3}+(2a_{\sigma}-4a_{\sigma}^{3}) \: exp\left(-\frac{1}{4a_{\sigma}^{2}}\right)\right],\nonumber\\
a_{\sigma}&=\frac{\omega}{2k_{F\sigma}}.\nonumber
\end{align}

\begin{flushleft}
\begin{tabular}{@{}lrp{10cm}}
avec & $k_{F\sigma}$ : & vecteur d'onde local de Fermi,\\
& $F(a_{\sigma})$ : & fonction d'atténuation,\\
& $a_{\sigma}$ : & paramètre de contrôle (sans unité) de la fonction d'atténuation $F(a_{\sigma})$.
\end{tabular}
\end{flushleft}

En retenant une fonctionnelle de corrélation basée elle aussi sur la LSDA $E_{c}^{LSDA}$, la plus simple des DFT hybrides à correction de longue portée (RSHX-LDA) s'écrit~:

\begin{equation}
E_{xc}^{RSHXLDA} = E_{x}^{LR-HF} + E_{x}^{SR-LSDA} + E_{c}^{LSDA}
\end{equation}

La fonctionnelle WB\cite{WB97X} s'écrit alors :

\begin{equation}
E_{xc}^{\omega B97} = E_{x}^{LR-HF} + E_{x}^{SR-B97} + E_{c}^{B97}
\end{equation}

Il est à noter que celle-ci ne possède pas d'échange HF à courte portée (SR), comme la plupart des fonctionnelles hybrides à correction de portée.

Malgré plusieurs études visant à optimiser la valeur du paramètre $\omega$, la précision calculatoire reste insuffisante en terme de thermochimie. En effet, nous l'avons déjà vu, une valeur trop grande pour $\omega$ tendrait vers une incompatibilité entre le terme d'échange non-local $E_{x}^{LR-HF}$ et le terme local de corrélation $E_{c}^{LSDA}$. De plus, nous pouvons aisément comprendre, d'après l'équation~\ref{erf}, que plus $\omega$ est petit, plus la contribution du terme d'échange SR $E_{x}^{SR-LSDA}$ sera importante. L'utilisation d'une trop faible valeur  reviendrait alors à traiter le problème dans un cadre très proche de la LDA classique qui, comme nous l'avons vu dans la partie~\ref{lda}, est incapable de traduire correctement le terme d'échange à courte portée.

\subsection{WBX}

Afin d'y remédier, une partie d'échange SR HF $E_{x}^{SR-HF}$, est ajoutée à $E_{x}^{SR-LSDA}$ dans une proportion d'environ 16\%,  de la même manière que \textsc{Becke} dans la fonctionnelle B88. Ceci à l'avantage de ne pas perturber la partie LR qui est dorénavant correcte. Ainsi, la nouvelle fonctionnelle comporte désormais un paramètre $c_{x}$ contrôlant la proportion d'échange exact HF à courte distance, comme nous pouvons le voir dans son expression :

\begin{equation}
E_{xc}^{LC-DFA} = E_{x}^{LR-HF} + c_{x}E_{x}^{SR-HF} + E_{x}^{SR-DFA} + E_{c}^{DFA}
\end{equation}

\noindent où :

\begin{equation}
E_{x}^{SR-HF} = -\frac{1}{2} \sum_{\sigma} \sum_{ij}^{occ.} \iint \phi_{i \sigma}^{*}(r_{1}) \phi_{j \sigma}^{*}(r_{1}) \frac{erfc(\omega r_{12})}{r_{12}} \phi_{i \sigma}(r_{2}) \phi_{j \sigma}(r_{2}).dr_{1}.dr_{2}, \\
\end{equation}

C'est ainsi que la fonctionnelle WBX\cite{WB97X} se décompose de la façon suivante~:

\begin{equation}
E_{xc}^{\omega B97X} = E_{x}^{LR-HF} + c_{x}E_{x}^{SR-HF} + E_{x}^{SR-B97} + E_{c}^{B97}
\end{equation}

La valeur de $\omega$, comme les valeurs des coefficients de développements linéaires et de développements à l'ordre $m$ des fonctionnelles WB et WBX ont été déterminées par la méthode des moindres carrés appliquée à une base de données composées de 412 valeurs précises, expérimentales et théoriques.

Malgré toutes ces optimisations conduisant à une bien meilleure représentation des systèmes en interaction, ces fonctionnelles connaissent encore des lacunes quant à la traduction des interactions de dispersion entre atomes, ie les forces de \textsc{London}. Comme nous allons le voir dans le cas de la fonctionnelle WBXD, ceci peut être corrigé par une prise en compte empirique des effets de dispersion.

\subsection{WBXD}

Cette dernière correction pourrait naturellement passer par le calcul idiomatique de l'énergie de dispersion entre chaque atome, mais cela occasionnerait alors un coût calculatoire prohibitif. C'est pourquoi Jeng-Da \textsc{Chai} et Martin \textsc{Head-Gordon} ont fait le choix d'appliquer cette correction de façon empirique par l'ajout d'un terme $E_{disp}$ à la fonctionnelle KS-DFT, ici la WBX. L'expression de l'énergie de la fonctionnelle WBXD\cite{wB97XD} ainsi créée devient alors :

\begin{equation}
E_{DFT-D}=E_{\omega B97X}+E_{disp}
\end{equation}

L'énergie de dispersion $E_{disp}$ est définie par rapport à une fonction d'amortissement $f_{damp}$ :

\begin{equation}
E_{disp}=-\sum_{i-1}^{N_{at}-1} \sum_{j-i+1}^{N_{at}} \frac{C_{6}^{ij}}{R_{ij}^{6}}f_{damp} (R_{ij})
\end{equation}

\noindent où :
\begin{equation}
f_{damp} (R_{ij})=\frac{1}{1+a(\frac{R_{ij}}{R_{r}})^{-12}}
\end{equation}

Une nouvelle fois, la partie empirique a été paramétrée par rapport à la même base de données que pour les fonctionnelles WB et WBX.


\newpage

\section*{Conclusion}
\markright{CONCLUSION}{}

En résumé, l'apport de la fonction erreur $(erf)$ permet de mieux gérer les contributions d'échange-corrélation selon la distance d'interaction. Les DFT hybrides WB et WBX prennent ainsi en compte la totalité de l'échange exact à longue distance et utilisent la méthode des gradients généralisés à faible distance, alors que la corrélation électronique reste basée sur celle initialement développée par \textsc{Becke} dans la fonctionnelle B97. Ceci a pour effet de supprimer le problème d'auto-interaction de la fonctionnelle d'échange à longue distance.

Les travaux de Jeng-Da \textsc{Chai} et Martin \textsc{Head-Gordon} ont finalement conduit à la fonctionnelle WBXD, de type LC-DFT-D hybride, où la totalité de l'échange exact HF est pris en compte à longue distance, en même temps qu'une petite partie -- environ 22 \% -- de l'échange exact HF est introduite à courte distance pour compléter une fonctionnelle d'échange B97 modifiée ; une correction empirique de la dispersion est finalement appliquée.

Comme toutes les fonctionnelles LC-DFT, le problème de l'auto-intéraction est corrigé à longue distance mais reste quelque peu présent à courte distance. Les effets de corrélation à longue distance sont quant à eux purement et simplement traités par la correction empirique de dispersion.

Cette fonctionnelle est, d'après les tests des auteurs, définitivement plus adaptée à l'étude de systèmes chimiques où les interactions non-covalentes sont importantes.







\include{vib/chap_vib}
	\newgeometry{textwidth=16cm}
	\chapter*{Conclusion Générale}
	\addcontentsline{toc}{chapter}{Conclusion Générale}
	\markboth{Conclusion Générale}{Conclusion Générale}
	\minitoc
	\restoregeometry
	
	
	
	Le cœur du présent travail de thèse consistait en une caractérisation théorique des spectres IR et Raman d'une série d'hétérocycles aromatiques définis comme modèles moléculaires en raison de leur représentativité de la diversité des systèmes asphalténiques. Ces derniers étant un enjeu conséquent pour l'industrie du pétrole, un effort de recherche se concentre en effet sur la compréhension de leur structure chimique et des interactions qui en découlent, responsables des propriétés physico-chimiques observées (agrégation en premier lieu, mais également précipitation et floculation).\\ 
	
	Pour répondre au double objectif de la résolution de la nature chimique des asphaltènes, d'une part, et des interactions à l'origine du phénomène d'agrégation, d'autre part, J. Shaw et K. Michaelian ont proposé un schéma de travail reposant sur la spectroscopie infra-rouge. Ce travail de thèse s'inscrit dans la mise en œuvre de leur méthodologie, en fournissant des données théoriques fiables à même de guider les expérimentateurs dans l'interprétation et la caractérisation de leurs spectres. L'idée générale de nos travaux est de caractériser les signatures vibrationnelles des modes inter et intra-moléculaires, de même que leurs modifications sous l'effet de l'environnement ou encore de la position et du nombre d'hétéroéléments. Pour peu que l'on parvienne à dégager de nos calculs des tendances, sinon des conclusions, quant à la fréquence et à la position des signatures vibrationnelles de ces modèles, en fonction de l'environnement et de la composition chimique, ces tendances serviront de guide à l'expérimentateur pour l'analyse des spectres réels.\\  
	
	La mise en place d'une stratégie calculatoire à même de répondre à ces problématiques s'est révélée ardue, du fait de la dimension des systèmes étudiés et de la nécessité d'incorporer à nos calculs des interactions longue portée, encore mal décrites par les méthodes de type DFT. Une part conséquente du présent travail a donc été consacrée au choix des outils méthodologiques les plus adaptés, ainsi qu'au choix d'approximations judicieuses, permettant de diminuer le coût calculatoire sans entamer la qualité des simulations. 
	Sur les sept chapitres regroupés dans ce manuscrit, trois sont ainsi consacrés au détail des méthodes et approximations employées, tandis qu'un chapitre à part entière présente les résultats de calculs préliminaires, menés sur des molécules test, dans le but de démontrer la pertinence de la stratégie calculatoire mise en place et son aptitude à traduire, avec une précision satisfaisante, les propriétés ciblées. 
	La part importante de ces chapitres préliminaires se veut représentative de l'effort méthodologique réalisé tout au long de ce travail de thèse, essentiel pour répondre au sujet ambitieux que représente la structure chimique des asphaltènes. Ces réflexions méthodologiques s'étendent à toutes les échelles de calcul envisagées, puisque les problématiques poursuivies sont multiples. Rappelons qu'il s'agit de déterminer l'influence 
	\begin{itemize}
	\item du nombre et de la position des hétéroéléments,
	\item de l'environnement (état gazeux, liquide ou solide),
	\item et des interactions à l'origine des processus d'agrégation
	\end{itemize}
sur les modes de vibration des systèmes asphalténiques.\\
 
	Par soucis de clarté, nous avons fait le choix d'associer à chacune de ces problématiques un chapitre spécifique, mais il est bien évident que les résultats obtenus, bien qu'à différents niveaux calculatoires (moléculaire, solide ou dynamique moléculaire), demeurent intimement liés. 
	Ainsi, notre chapitre cinq se concentrait sur l'influence des hétéroatomes, et présentait donc des résultats de calculs moléculaires. Dans cette partie, nos calculs portaient sur une série de six motifs moléculaires, contenant ou non une proportion d'hétéroéléments (N, S ou O), d'abord étudiés isolément puis sous forme de dimères (de sorte à simuler l'effet de l'interaction). Il est à noter que nos résultats se révèlent en bon accord avec les spectres expérimentaux, démontrant une fois encore la pertinence de notre prise en compte des anharmonicités électrique et mécanique pour la REVS. Toutefois, nos calculs ont révélé que les signatures vibrationnelles caractéristiques des modes intermoléculaires associés aux dimères se trouvaient aux bas nombre d'ondes (0-150 cm$^{-1}$), soit dans une zone traditionnellement mal décrites par les calculs moléculaires. Ces conclusions se déduisent de la confrontation des spectres vibrationnels des monomères et des dimères, qui montrent, pour les systèmes contenant de l'oxygène, du soufre et du carbone, une correspondance parfaite pour l'ensemble des bandes, à l'exception des bandes induites par les interactions intermoléculaires, que l'on voit apparaître aux bas nombre d'ondes sur les spectres des dimères. Le cas des motifs contenant de l'azote se révèle plus spécifique, puisque monomères et dimères présentent des différences dans la signature des modes de vibration intramoléculaires dans la région en-dessous de 600 cm$^{-1}$. De fait, les systèmes contenant de l'azote peuvent être plus aisément caractérisés en tant que systèmes isolés ou en tant que systèmes en interaction, au travers de l'analyse de deux zones spectrales distinctes. Pour l'ensemble des autres systèmes, la conclusion quant à la présence ou l'absence d'interactions se déduit d'une analyse aux bas nombres d'onde. 	
Du point de vue structural, et non plus strictement vibrationnel, les configurations envisagées pour les dimères modèles étudiés révèlent qu'il n'existe pas d'orientation privilégiée des hétéroéléments, de sorte que l'interaction prépondérante demeure le $\pi$-stacking.\\ 	

	
	Le passage de l'état gazeux, simulé au travers de calculs moléculaires, à l'état solide, présenté dans le chapitre six, avait pour objectif de comprendre l'effet de l'environnement en caractérisant la modification des signatures des modes de vibration précédemment analysés. Les modes de réseau calculés, par la prise en compte d'un environnement plus représentatif des systèmes réels, ont ainsi été comparés aux modes de vibration intermoléculaires présentés au chapitre cinq. Cette comparaison révèle que, pour l'ensemble des systèmes aromatiques considérés, une bande active caractéristique des interactions intermoléculaires se retrouve systématiquement aux alentours de 100 cm$^{-1}$, quels que soient la taille du système, la nature du noyau aromatique (benzénique ou non) ou encore le type d'hétéroélément présent.\\ 

 
 
 	Enfin, les simulations en dynamique moléculaire menées dans le but d'analyser l'influence des hétéroéléments et des proportions de mélanges des espèces modèles sur le comportement de nano-agrégats, ont fait l'objet d'une analyse au chapitre sept. De ces modélisations, il ressort que les noyaux polyaromatiques, au cœur de la structure chimique des asphaltènes, jouent un rôle prépondérant dans l'agrégation de ces systèmes. La position des hétéroatomes au sein des modèles se révèle influer sur les interactions de $\pi$-stacking à l'origine du phénomène. Toutefois, bien que la proportion d'hétéroatomes soit conséquente au sein des asphaltènes -- ce qui porte à penser qu'ils possèdent une influence remarquable dans les processus de nano-agrégation --, ce sont la taille et la nature des ramifications qui jouent véritablement un rôle décisif dans la description du phénomène. \\
	
	
	En conclusion générale, les modes de vibration caractérisés tout au long de ce travail permettent de fournir aux expérimentateurs des critères d'analyse pour l'interprétation des spectres IR et Raman. 
	Le résultat principal de la présente étude réside dans l'interprétation des modes de vibration intermoléculaires, caractéristiques des interactions de type $\pi-\pi$ et $\pi-H$, identifiables aux bas nombres d'onde (soit aux alentours de 100 cm$^{-1}$). La présence, sur les spectres expérimentaux, de bande(s) dans cette zone révèle sans ambiguité l'existence de systèmes en interaction. Au-delà de ce résultat, au demeurant directement utilisable par les spectroscopistes, l'ensemble de nos calculs représente une avancée vers la compréhension de la structure et des propriétés des asphaltènes. Notamment, la méthodologie calculatoire poursuivie, de par sa fiabilité, ainsi que les différentes échelles de calcul utilisées, font de cette exploration un véritable travail de développement. Pour la rendre complète, la finalisation de cette étude devra toutefois passer par :
	
	\begin{itemize}
	\item le calcul de spectres au bas nombre d'ondes, sur des systèmes pour lesquels on ne dispose pas de données de référence, dans le but de valider nos conclusions
	\item une analyse, par exemple, des composantes principales (ACP) des spectres vibrationnels, afin d'étudier la distribution des échantillons (technique déjà employée en archéologie et en pharmacologie), problématique cruciale dès que l'on dispose d'un trop grand nombre de données à analyser et/ou a hierarchiser comme dans le cas des asphaltènes. 
	\end{itemize}













 	
	
	
	
	
	

	
	
	
\begin{appendices}
	\include{anex/anex1}
	\include{grand-resumes/frances}
\end{appendices}
\bibliography{biblio}
\renewcommand{\bibname}{Références bibliographiques}
\bibliographystyle{unsrt}
\bibliography{biblio}
\end{document}