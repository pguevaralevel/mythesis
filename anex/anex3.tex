\newgeometry{textwidth=16cm}
\chapter{Calcul de l'énergie de polarisation}
\minitoc
\restoregeometry

\newpage

Cherchons à exprimer les différents termes matriciels relatifs à l'énergie de polarisation. Certains de ces termes sont nuls et ne nécessiteront pas de calculs. Les conditions calculatoires vont pouvoir être établies par la connaissance des combinaisons permises entre les nombres quantiques des états considérés au vu des règles de la mécanique quantique. Pour cela, nous allons nous servir des propriétés induites par le théorème de (WE).\\

L'énergie de polarisation représente l'énergie au second ordre des perturbations. Elle se compose de deux termes : un terme de dispersion impliquant simultanément les états excités de $a$ et de $b$ et un terme d'induction ne faisant intervenir qu'un seul ensemble d'états excités de l'un des deux systèmes. En reprenant l'expression \ref{1.33} de l'opérateur de perturbation, on obtient pour expression de l'énergie au second ordre : 

\begin{equation}
E(2) = \sum_{i,j=0}^{\infty} \sum_{i',j'=0}^{\infty} \frac{E_{disp}^{ii'jj'} (\nu , \lambda) + E_{induc}^{ii'jj'}(\nu , \lambda)}{R^{i+i'+j+j'+2}} = \sum_{i+i'+j+j'+2=2}^{\infty} \frac{C_{i+i'+j+j'+2}}{R^{i+i'+j+j'+2}} \label{2.21}
\end{equation}

Le domaine de validité d'un tel développement est limité pour les très grandes distances par les effets de retardement. Pour les faibles distances où le recouvrement orbitalaire ne peut plus être négligé, l'équation \ref{2.21} n'est plus valable et l'on doit prendre en compte l'énergie de répulsion. Pour des distances comprises entre faibles et supérieures à 100 u.a où il n'existe généralement que très peu information expérimentales, les méthodes dites asymptotiques peuvent apporter de précieux résultats, notamment pour les expérimentateurs qui s'intéressent aux études des collisions entre atomes. 

\subsection{Energie de dispersion}

\subsubsection{Eléments de matrices}

L'énergie de dispersion fait intervenir simultanément les états excités de $a$ et de $b$ que nous noterons respectivement $\Psi_{r}(a)$ et $\Psi_{t}(b)$.

\begin{equation}
E_{disp}^{ii'jj'} (\nu , \lambda) = - \sum_{r0}^{\infty} \sum_{t0}^{\infty} \frac{\langle \Psi_{0}^{\nu} (a) \Psi_{0}^{\nu} (b)| V_{ij}|\Psi_{r} (a) \Psi_{t} (b) \rangle  \langle \Psi_{r} (a) \Psi_{t} (b) |V_{i'j'}| \Psi_{0}^{\lambda} (a) \Psi_{0}^{\lambda}(b) \rangle}{(E_{r} (a) - E_{0} (a)) + (E_{t} (b) - E_{0} (b))} \label{2.22}
\end{equation}


En reprenant la forme générale des fonctions d'onde dépendantes des nombres quantiques orbitaux, magnétiques et de spin de chacun des systèmes pris isolément, l'équation \ref{2.22} peut s'écrire : 

\begin{equation}
\begin{array}{c}
E_{disp}^{ii'jj'} (\nu , \lambda) = - \sum_{k=1}^{n_{\nu}} \sum_{l=1}^{n_{\lambda}} \alpha_{\nu , k} \alpha_{\lambda , l} \hspace{1cm} x \\
\\
\sum_{w"_{a} w"_{b}} \frac{\langle S_{k} L_{k} M_{S_{K}} M_{L_{K}} (a) |\langle S'_{k} L'_{k} M'_{S_{K}} M'_{L_{K}} (b)| V_{ij} (ab) | S"_{a} L"_{a} M"_{S_{a}} M"_{L_{a}} (a)\rangle| S"_{b} L"_{b} M"_{S_{b}} M"_{L_{b}} \rangle}{(E_{n"_{a}} - E_{0}(a)) + (E_{n"_{b}} - E_{0}(b))} \hspace{0.5 cm} x \\
\\
\frac{\langle S"_{a} L"_{a} M"_{S_{a}} M"_{L_{a}} (a)| \langle S"_{b} L"_{b} M"_{S_{b}} M"_{L_{b}} | V_{i'j'}(a,b)| S_{l} L_{l} M_{S_{l}} M_{L_{l}} (a) \rangle | S'_{l} L'_{l} M'_{S_{l}} M'_{L_{l}} (b) \rangle}{(E_{n"_{a}} - E_{0} (a)) + (E_{n"_{b}} - E_{0}(b))}
\end{array}
\end{equation}

avec $w"_{a} = (n"_{a}, l"_{a}, S"_{a}, L"_{a}, M"_{S_{a}}, M"_{L_{a}})$. L'indice " sera réservé aux états excités. 

Comme pour l'énergie électrostatique, la démarche adoptée consiste à séparer les parties relatives au système $a$ de celles relatives au système $b$. \\

\subsubsection{Règles de sélection sur les moments orbitaux}

Par application des règles de sélection sur les termes de Wigner qui composent l'expression de $\mathfrak{R}_{L"_{a} L"_{b}}^{ii'jj'}$ (la partie purement radiale), on obtient les valeurs de $i, j, i'$ et $j'$ permises. 


\begin{equation}
\begin{cases}
l_{k} + i + l"_{a} = entier\ pair \\
|l_{k} - l"_{a}| \leq i \leq l_{k} + l"_{a} \\
l"_{a} + i' + l_{l} = entier\ pair \\
|l"_{a} - l_{l}| \leq i' \leq l"_{a} + l_{l} \\
l'_{k} + j + l"_{b} = entier\ pair \\
|l'_{k} - l"_{b}| \leq j \leq l'_{k} + l"_{b} \\
l'_{l} + j' + l"_{b} = entier\ pair \\
|l'_{l} - l"_{b}| \leq j' \leq l'_{l} + l"_{b}
\end{cases}
\end{equation}

\begin{itemize}
	\item Dans le cas d'interaction de type $S + S : l_{k} = l_{l} = 0$ et $l'_{k} = l'_{l} = 0$ on a:
\end{itemize}

\begin{equation}
\begin{cases}
0 + i + l"_{a} = entier\ pair \\
|0 - l"_{a}| \leq i \leq 0 + l"_{a} \\
l"_{a} + i' + 0 = entier\ pair \\
|l"_{a} - 0| \leq i' \leq l"_{a} + 0 \\
0 + j + l"_{b} = entier\ pair \\
|0 - l"_{b}| \leq j \leq 0 + l"_{b} \\
0 + j' + l"_{b} = entier\ pair \\
|0 - l"_{b}| \leq j' \leq 0 + l"_{b}
\end{cases}
\end{equation}

Soit $i = i' = l"_{a}$ et $j = j' = l"_{b}$. L'ensemble des combinaison (i, i', j, j') permises sont telles que $i = i'$ et $j = j'$. En tenant compte de la définition de l'énergie de dispersion \ref{2.21} : 

\begin{equation}
E_{disp}(2) = \sum_{i+i'+j+j'+2}^{\infty} \frac{C_{i+i'+j+j'+2}}{R^{i+i'+j+j'+2}}
\end{equation}

Il est désormais possible de pouvoir dénombrer et connaitre les différents contributions des coefficients de dispersion (encore appelés coefficients de Van der Waals) à la correction énergétique au second ordre sur l'énergie du système [$a - b$].

La combinaison (1,1,1,1) représente la contribution du coefficients $C_{1+1+1+1+2} = C_{6}$.

Les combinaison (1,1,2,2) et (2,2,1,1) représentent les deux contributions au coefficients $C_{8}$. etc



\subsection{Energie d'induction}

\subsubsection{Eléments de matrices}	

Contrairement à l'énergie de dispersion, l'énergie d'induction ne fait intervenir que les états excités de l'un ou l'autre des systèmes $a$ ou $b$ : 

\begin{equation}
E_{ind}(2) = \sum_{i,j=0}^{\infty} \sum_{i',j'=0}^{\infty} \frac{E_{ind}^{ii'jj'} (a; \nu , \lambda) + E_{ind}^{ii'jj'} (b; \nu , \lambda)}{R^{i+i'+j+j'+2}} = \sum_{i+j+j'+i'+1=2}^{\infty} \frac{C_{i+i'+j+j'+2}}{R^{i+i'+j+j'+2}}
\end{equation}


Soit par exemple l'énergie d'induction du système $a$ :

\begin{equation}
E_{ind}^{ii'jj'} (a;\nu , \lambda) = \sum_{r \neq 0}^{\infty} \sum_{t\neq 0}^{\infty} \frac{\langle \Psi_{0}^{\nu} (a) \Psi_{0}^{\nu} (b) |V_{ij} (a,b)|\Psi_{r} (a) \Psi_{0} (b) \rangle  \langle \Psi_{r} (a)\Psi_{0} (b) |V_{i'j'}|\Psi_{0}^{\lambda} (a) \Psi_{0}^{\lambda}(b) \rangle}{(E_{r} (a) - E_{0}(a)}
\end{equation}

Avec $\Psi_{0}^{\nu} (a) = |S_{k} L_{k} M_{S_{k}} M_{L_{k}}\rangle$ et $\Psi_{r}(a) = |S"_{a} L"_{a} M"_{S_{a}} M"_{L_{a}}\rangle$. Il existe une formule similaire $E_{ind}^{ii'jj'} (b;\nu , \lambda)$ pour le système $b$. Chaque terme est calculé séparément.

\subsubsection{Règles de sélection sur les moments orbitaux}

\textbf{Système $a$}

Les règles de sélection sont identiques à celles énoncées dans l'énergie de dispersion. \\

\textbf{Système $b$}

Les règles de sélection propres au système $b$ sont fixées par les lois sur les coefficients : 

\begin{center}
	$\begin{pmatrix}
	L'_{k} & j & L_{b}\\
	0 & 0 & 0 
	\end{pmatrix} \ x \ \begin{pmatrix}
	L_{b} & j' & L'_{l}\\
	0 & 0 & 0
	\end{pmatrix}$
\end{center}

et sont : 

\begin{equation}
\begin{cases}
L_{b} + j + L'_{k} = entier\ pair \\
|L'_{k} - L_{b}| \leq j \leq L'_{k} + L_{b} \\
L_{b} + j' + L'_{l} = entier\ pair \\
|L'_{l} - L_{b}| \leq j' \leq L'_{l} + L_{b}
\end{cases}
\end{equation}

\textbf{Système $a - b$}
