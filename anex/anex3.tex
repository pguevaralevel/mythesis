\newgeometry{textwidth=16cm}
\chapter[Calcul de la polarisabilité]{Calcul de la polarisabilité}
\minitoc
\restoregeometry

\newpage
	
\section{Calcul de la polarisabilité en fonction de la fréquence imaginaire}
	
En l'absence de champ électrostatique, les interactions entre deux systèmes, sont décrites en fonction de leurs moments permanents et des polarisabilités statiques. Pour un champ périodique de pulsation $\omega$, les interactions sont modifiées et les polarisabilités dépendent de cette pulsation. L'étude des forces de dispersion à longue distances nécessite la connaissance d'intégrales qui tiennent compte de l'ensemble de fréquences $\omega$. Pour des raisons mathématiques simples, les intégrations de fonctions discontinues restent difficiles. On doit se ramener alors à la connaissance de la variation des polarisabilités dépendantes des fréquences imaginaires de chacun des deux systèmes pris isolement puisque ces fonctions sont continues.\\
	
Un champ oscillant de fréquence $\omega$ provoque sur un système donné une distorsion de la distribution des charges. Si $\omega$ est loin d'une résonance, les charges induites peuvent facilement s'interpréter à partir des moments électriques du système. Si $\omega$ est proche d'une résonance, les moments électriques induits sont très importants est seule une émission où une absorption d'un photon, associée à un changement d'état quantique du système peut se produire. Nous n'aborderons pas ici le problème dans son détail et renverrons le lecteur au travail effectué par Ayma et al \cite{ayma1997etude}.\\
	
	
L'énergie d'induction résulte de la distorsion d'un système (se trouvant dans un état $m_{b}$) sous l'effet d'un champ $F_{\alpha}^{b}$ et d'un gradient de champ $F_{\beta\gamma}^{b}$, induite par ses plus proches voisins (système $a$). Elle dépend directement des propriétés statiques : 
	
\begin{equation}
E_{ind} = -\frac{1}{2} \alpha_{\alpha\beta}^{m_{b}} (0) F_{\alpha}^{b} F_{\beta}^{b} - \frac{1}{3!} A\alpha , \beta\gamma (0) F_{\alpha}^{b} F_{\beta\gamma}^{b} - \ldots
\end{equation}
	
Elle est rarement la source prépondérante d'une attraction entre deux système et peut être souvent négligée en comparaison des énergies électrostatiques et des dispersion. 
	
Le calcul de l'énergie de dispersion nécessite quant à lui la connaissance des polarisabilités dynamiques qui représentent la réponse du système à une variation dépendante du temps provoquée par le champ électrique externe. 
	
L'expression des coefficients de dispersion déterminée à partir de l'énergie au second ordre de perturbation.
	
\begin{equation}
C_{2k} = \sum_{r\neq=0} \sum_{t\neq=0} 4\pi \sum_{m=-l}^{l} d_{m} (l,l) \frac{|\langle \psi_{r}(a) | \sum_{k_{a}} r_{k_{a}} Y_{l}^{m}| \psi_{0} (a) \rangle \langle \psi_{t}(b)| \sum_{k_{b}} r_{k_{b}} Y_{l}^{-m}| \psi_{0}(b)\rangle|^{2}}{\Delta E_{\gamma_{0} L\rightarrow \gamma_{r} L'} + \Delta E_{\gamma_{0} L\rightarrow \gamma_{t} L"}}
\end{equation}
		
s'exprime facilement en fonction des forces d'oscillateurs ($l + 1$)-polaires :
	
\begin{equation}     
C_{2k} = \sum_{r\neq=0} \sum_{t\neq=0} 4\pi \sum_{m=-l}^{l} d_{m} (l,l)  \frac{f_{\gamma_{0}L\rightarrow \gamma_{r}L'}^{l_{a}} f_{\gamma_{0}L \rightarrow \gamma_{t}L"}^{l_{b}}}{\Delta E_{\gamma_{0}L\rightarrow \gamma_{r}L'} \Delta E_{\gamma_{0}L\rightarrow \gamma_{t}L"} (\Delta E_{\gamma_{0}L\rightarrow \gamma_{r}L'} + \Delta E_{\gamma_{0}L\rightarrow \gamma_{t}L'})}
\end{equation}
		
Chaque terme de cette somme fait intervenir des grandeurs dépendantes de $a$ et de $b$. On peut transformer ce problème à deux centres, en deux problèmes à un centre, en introduisant la polarisabilité à la fréquences imaginaires : 
	
\begin{equation}
\alpha_{l}(i\omega) = \frac{8\pi}{(2l + 1)} \sum_{s} \frac{(E_{s} - E_{0}) |\langle \psi_{0}| \sum_{k_{a}} r_{k_{a}}^{l} Y_{l}^{0} | \psi_{s}\rangle|^{2}}{\Delta E^{2} + \omega^{2}}
\end{equation}

et en utilisant l'astuce mathématique proposée par Dalgarno et al\cite{dalgarno1966calculation,linder1962generalized,chan1965long,dalgarno1966long} où : 
	
\begin{equation}
\frac{1}{X + Y} = \frac{2}{\pi} \int_{0}^{\infty} \frac{XY}{(X^{2} + u^{2}) (Y^{2} + u^{2})} \hspace{0.3cm} du \hspace{1.5cm} avec \hspace{0.5cm} X,Y > 0
\end{equation}
	
L'expression classique des coefficients de dispersion s'écrit : 
	
\begin{center}
\fbox{$C_{2k} = \sum_{(l_{a},l_{b})}^{k-1} \frac{d_{m} (l_{a},l_{b})=0}{2\pi} \int_{0}^{\infty} \alpha^{l_{a}} (i\omega) \alpha^{l_{b}} (i\omega)d\omega$}
\end{center}
		
dans laquelle : 
	
\begin{center}
\fbox{$\alpha_{LL'}^{l_{a}} (i\omega) = \sum_{\gamma L} \frac{f_{\gamma_{0}L \rightarrow \gamma L'}^{l_{a}}}{(\Delta E_{\gamma_{0}L \gamma_{r}L'} + \omega^{2})} $}
\end{center}
	
et où $\gamma$ représente l'ensemble des nombres quantiques excepté le nombre quantique orbitalaire L.
	
Le calcul de $\alpha(i\omega)$ sous cette forme infinie, semble à priori impossible puisque nous ne disposons que de très peu de données expérimentales te théoriques sur les forces d'oscillateurs pour les états très excites. Le calcul des polarisabilités imaginaires est alors effectué à l'aide de la méthode time dependent gauge invariant (TDGI). La gamme de fréquences généralement utilisée est comprise entre 0 et 3 u.a. puisque, au delà, les valeurs de la polarisabilité deviennent très faibles et négligeables.
	
A partir de ces valeurs dynamiques, le coefficients de dispersion d'ordre 6 dont l'expression à depuis longtemps établie par Casimir- Polder \cite{casimir1948influence}, est la suivante : 
	
\begin{equation}
C_{6}= \frac{3}{\pi} \int_{0}^{\infty} \alpha_{LL'_{a}}^{1} (i\omega)\ x \ \alpha_{LL"_{b}}^{1} (i\omega) d\omega \label{3.65}
\end{equation}
	
Dans le cas de deux atomes $a$ et $b$ se trouvant dans leur état fondamental, le terme en $C_{6}$ est obtenu par un simple ajustement de l'expression \ref{3.65}.\\
	
Le terme en $C_{8}$ obtenu pour deux systèmes isotropes est calculé de la même façon. 
	
\begin{equation}
\bar{C_{8}} = \frac{15}{\pi} \int_{0}^{\infty} [\bar{\alpha}_{LL'_{a}}^{1} (i\omega) \bar{C}_{LL"_{b}} (i\omega) + \bar{C}_{LL'_{a}} (i\omega) \bar{\alpha}_{LL"_{b}}^{1} (i\omega)] d\omega
\end{equation}
	
où $C_{LL'}(i\omega) = \frac{1}{3} \alpha^{2}_{LL'}(i\omega)$ \cite{orr1971perturbation} représente la polarisabilité quadrupolaire obtenue en remplaçant l'opérateur $\overrightarrow{r}$ de perturbation par $\overrightarrow{r}^{2}$.
	
			

