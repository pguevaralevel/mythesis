	\newgeometry{textwidth=16cm}
	\chapter[Interactions à longue portée]{Systèmes en interaction à longue portée - De la nécessité des approches DFT}
	\renewcommand{\chaptername}{Chapitre}
	\minitoc
	\restoregeometry
	
	\newpage
	
	\section*{Introduction}
	\markright{INTRODUCTION}{}
	
	Un des challenges qui nous ont été proposés initialement par les expérimentateurs était celui de la caractérisation des molécules et de leurs interactions au sein des mélanges d’asphaltènes. Comme nous l’avons rappelé dans le chapitre 1, très peu d’informations sont connues à ce sujet à l’heure actuelle. Aucune technique de caractérisation, qu'elle soit physique ou chimique, n’est véritablement à même de répondre précisément à cette question. J. Shaw a proposé, bien avant les autres acteurs du domaine, que ces systèmes \og associés \fg{} soient caractérisés à l’aide des spectroscopies Infra-Rouge et Raman, en utilisant notamment la technique de la photo acoustique, dans le but d’accéder aux informations spectrales de très bas nombres d’ondes (zone spectrale suspectée de contenir la signature des modes de libration associés aux interactions intermoléculaires). Au-delà des difficultés inhérentes au manque d’informations, aussi bien quant à la nature des molécules et des familles qui constituent ces mélanges complexes que concernant le rôle des hétéroéléments, une autre des difficultés qui s’est imposée à nous concernait le traitement quantique vibrationnel de systèmes de grande dimension et, qui plus est, en interaction. En admettant même que les deux problèmes vibrationnels majeurs, i.e. la détermination de la PES (Potential Energie Surface) et la résolution de l’équation vibrationnelle de Schr\"{o}dinger, soient absents et indépendants de la dimension du problème (ce qui n’a bien entendu aucun sens, comme nous l’avons montré dans le chapitre 2), la prise en compte dans nos modèles quantiques des interactions est à elle seule un des problèmes majeurs qu’il nous a fallu intégrer dès le départ, sous peine de ne pouvoir fournir aux expérimentateurs des données suffisamment fiables pour rendre compte des observations spectroscopiques. 
	
	Ainsi, pour ce qui concerne le traitement quantique des interactions intermoléculaires, qui fait l’objet principal de ce chapitre, il est important de rappeler que notre travail ne se situe pas dans un contexte de développement méthodologique mais plutôt dans un contexte de recherche des stratégies calculatoires les plus adaptées pour traiter de problèmes vibrationnels concernant des systèmes chimiques en interaction, de grandes dimensions, qu’une méthode fortement corrélée ne permet pas encore d’atteindre\footnote{malgré les progrès particulièrement significatifs réalisés au cours de ces dernières années, tels que l’emploi de bases cc-pVXZ-F12 (X=D,T,Q) développées pour l’utilisation de méthodes explicitement corrélées (citons notamment les bases dédiées à la résolution des approximations d'identité RI (en anglais « Resolution Identity »)), des approximations d'ajustement de la densité DF (en anglais « Density Fitting ») pour le calcul d’intégrales, la non-linéarité des facteurs de corrélation, etc.}. L’utilisation de la DFT s’est donc naturellement imposée à nous.
	
	Force est de constater les progrès réalisés depuis une décennie dans la description du formalisme de la théorie de la fonctionnelle de la densité, qui rendent son utilisation quasi-systématique. Bien loin des effets de mode, on constate les réussites frappantes de la DFT dans la description de propriétés telles que les liaisons, les structures, la cohésion aussi bien au niveau moléculaire que dans les solides. Néanmoins, il est des domaines dans lesquels les progrès sont encore lents et où les méthodes sont encore mises en défaut, tels que : le traitement des atomes lourds, les systèmes possédant des électrons fortement corrélés, le traitement crucial en réactivité des états multi-déterminantaux, le problème de la correction de la self-itération et, bien entendu, le problème qui nous occupe directement de la description des forces de dispersion \cite{gerber2005description}. Cette dernière problématique n’est pas récente puisqu'elle émerge des premiers travaux de Gordon et Kim \cite{gordon1972theory}. Elle s’avère particulièrement prégnante pour l’étude des systèmes fortement liés (tels certains oxydes métalliques) et dans les cas où les forces de van der Waals sont prédominantes (matière molle, complexes de van der Waals, biomolécules et autres matériaux lamellaires). L’origine de ce problème est désormais clairement identifiée : les effets de corrélation électronique des forces de dispersion sont purement non locaux et en aucun cas une approximation locale ou semi-locale ne pourra en rendre compte. Si la fonctionnelle d’échange-corrélation exacte pouvait être connue, il serait cependant possible de répondre à la question posée par la description des forces de van der Waals dans le formalisme Kohn-Sham (KS). Cela fait plus d’une décennie que la recherche dans ce domaine est extrêmement active. L’idée qui prévaut aujourd’hui est celle de développer des méthodes ayant les moyens, pour un moindre coût calculatoire, d’incorporer des corrections au formalisme KS sans perdre de vue les avantages de la DFT, aussi bien au niveau moléculaire que dans le calcul en conditions périodiques. 
	
	Rappelons que l’utilisation pratique de la DFT dans son formalisme KS repose sur l’approximation de la fonctionnelle d’échange-corrélation. L’approximation originelle, i.e. l’Approximation de Densité Locale (LDA pour l’anglais \og Local Density Approximation \fg), s’est révélée étonnamment efficace et difficile à améliorer de manière systématique. Il a été montré que la LDA, étant locale par construction, est particulièrement adaptée pour décrire les corrélations à très courte portée inter-électronique, mais échoue à décrire quantitativement les corrélations à longue portée électronique. Ceci reste vrai avec la plupart des améliorations post-LDA classiques (que nous n’aborderons pas dans ce travail), qui demeurent par essence des approximations de nature locale. Différentes solutions existent désormais pour traiter ce problème. Citons par exemple sans aucune volonté d’exhaustivité, avec différents niveaux d’empirisme, les travaux autour de paramétrisations spécifiques de nouvelles fonctionnelles GGA ou méta-GGA \cite{valero2006nonadiabatic}, les travaux de type DFT+D incluant des corrections de dispersion de Grimme pour ne citer que le plus connu \cite{grimme2006semiempirical,grimme2010consistent}, les travaux de type Symmetry-Adapted Perturbation Theory (SAPT) ou SAPT(DFT) même si ces derniers ne s’attachent plus tout à fait à la recherche d’une fonctionnelle d’échange-corrélation, et les travaux qui vont finalement s’atteler à donner une forme explicite de la fonctionnelle d’échange-corrélation \cite{klimevs2012perspective,tran2013nonlocal}, en recherchant son caractère non-local (notamment au travers de l’application du théorème de fluctuation-dissipation combiné à la connexion adiabatique, comme dans les méthodes AC-FDT, DRSLL \cite{dion2004van,wellendorff2012density}  et Bayesian error estimation functional BEEF \cite{wellendorff2012density}. Parmi ces dernières solutions non-empiriques, on peut citer notamment les modèles d’Anderssson-Langreth-Lundqvist (ALL) ou ceux de type Random Phase Approximation (RPA) qui ont très récemment fait l’objet d’études approfondies de la part, par exemple, de S. Lebègue \textit{et al} \cite{lebegue2010cohesive,mussard2013modelisations}.\\
	
	
	Pour résumer, on peut rappeler que, d’un point de vue théorique, l’énergie d’interaction intermoléculaire est une observable que l’on peut soit calculer directement (à l’aide d’une approche globale de type \og supermolécule \fg), soit chercher à interpréter par différentes décompositions auxquelles on peut adjoindre, dans la mesure du possible, un sens physique. Nous avons fait le choix arbitraire, dans la première partie de ce chapitre, de présenter les termes d’interaction à partir de la théorie de perturbation développée jusqu’au second ordre. Cette présentation nous semblait justifiée par le fait qu’une fois les hypothèses de travail définies et les fonctions de réponses -- nécessaires au calcul des termes d’interaction -- établies, le détail des approches méthodologiques les plus avancées dans ce domaine s’en trouve facilité. De plus, parce qu’il est désormais acquis qu’un tel traitement conduit généralement à une mauvaise convergence de la série de perturbation, nous profiterons de cette présentation pour introduire l’approche SAPT (\og Symmetry-Adapted Perturbation Theory \fg), qui corrige ces divergences et qui est considérée aujourd’hui comme l’une des plus précises pour le calcul des énergies d’interaction intermoléculaires. 
	De façon formelle, toutes les approches présentées dans cette première partie n’imposent pas l’utilisation des approches de la DFT. Néanmoins, de façon pratique, de par la dimension et la diversité des systèmes que l’on cherche à étudier, le recours aux approches DFT est indispensable. Ainsi, après avoir donné, dans la seconde partie de ce chapitre, quelques rappels sur les fondements théoriques de la méthodologie DFT, nous présenterons la notion de fonctionnelle hybride puis présenterons le principe des approches employées dans ce travail, qui ajoutent \textit{ad hoc} à un calcul KS usuel des fonctions semi-empiriques rendant compte des effets de dispersion.\\
	
	
	
	
	
	
	
	
	
	
	
	
	%%%%%%%%%%%%%%%%%%%%%%%%%%%%%%%%%%%%%%%%%%%%%%
	%%%%%%%%%%%%%%%%%%%%%%%%%%%%%%%%%%%%%%%%%%%%%%
	%%%%%%%%%%%%%%%%%%%%%%%%%%%%%%%%%%%%%%%%%%%%%%
	%%%%%%%%%%%%%%%%%%%%%%%%%%%%%%%%%%%%%%%%%%%%%%
	\section[Théorie des perturbations]{Théorie des perturbations}
	%%%%%%%%%%%%%%%%%%%%%%%%%%%%%%%%%%%%%%%%%%%%%%
	%%%%%%%%%%%%%%%%%%%%%%%%%%%%%%%%%%%%%%%%%%%%%%
	%%%%%%%%%%%%%%%%%%%%%%%%%%%%%%%%%%%%%%%%%%%%%%
	%%%%%%%%%%%%%%%%%%%%%%%%%%%%%%%%%%%%%%%%%%%%%%
	
	Au-delà des liaisons hydrogène et des interactions de nature électrostatique (\textit{e.g.} charge-charge, charge-dipôle, dipôle-dipôle), l’interaction non covalente parmi les plus étudiées à l’heure actuelle est probablement celle impliquant les systèmes aromatiques \cite{grimme2008special}. Ces interactions entre cycles aromatiques, dénommées \og $\pi$-stacking \fg{} (peut-être à mauvais escient \cite{martinez2012rethinking}), sont notamment considérées comme responsables de la stabilité d’un très grand nombre de structures remarquables \cite{mcgaughey1998pi}, dont celles qui préfigurent certainement la structure des asplhaltènes. L’étude expérimentale de ces interactions représente encore un défi car il est généralement complexe de séparer les interactions de type $\pi$-stacking des interactions secondaires ou des effets de l’environnement. Du fait de ces difficultés expérimentales, les études en chimie computationnelle apparaissent comme une alternative de choix pour comprendre aussi bien la nature fondamentale de ces interactions non covalentes, que leur influence sur les systèmes chimiques étudiés.Le travail de revue réalisé par Martinez et Iverson recense les principales définitions et les principaux modèles employés pour décrire ces interactions particulières \cite{martinez2012rethinking}.
	Le \og graal \fg{}, pour les expérimentateurs comme pour les théoriciens, serait d’arriver \textit{in fine} à caractériser séparément la contribution de chacune de ces interactions non covalentes et de prévoir tout à la fois leur rôle et leur intensité.\\
	
	À ce jour, nous savons que l’énergie d'interaction intermoléculaire d'un ensemble de molécules se situe généralement entre 1 et 20 kcal/mol selon le nombre et le type de molécules impliquées. Cependant, l'énergie d'une liaison chimique covalente se mesure entre 100 et 300 kcal/mol, et se situe donc dans un autre ordre de grandeur. De même, la portée d’une liaison chimique dépasse rarement quelques Angströms, tandis que les interactions intermoléculaires s'étendent théoriquement jusqu'à l'infini (électrostatique) et de manière pratique entre 2 et 10 Angströms selon la taille et la nature du système.\\
	
	D’un point de vue théorique l’énergie d’interaction intermoléculaire est une observable que l’on peut interpréter par différentes décompositions auxquelles on cherche à adjoindre, dans la mesure du possible, un sens physique. Buckingham \cite{buckingham1967permanent} proposait déjà, en 1967, une décomposition de l’énergie d’interaction intermoléculaire en quatre grandes contributions : l’électrostatique $E_{elec}$, l’induction $E_{ind}$, l’échange-répulsion $E_{rep}$, et la dispersion $E_{disp}$.\\
	
	\begin{equation}
	\Delta E = E_{elec} + E_{ind} + E_{rep} + E_{disp}
	\end{equation} 
	
	L’interaction électrostatique est l’ensemble des interactions coulombiennes existant entre deux densités de charges isolées. Elle est additive et peut être répulsive ou attractive selon l’orientation relative des molécules. Elle constitue la plus grande partie de l’interaction intermoléculaire. 
	
	L’énergie d’induction est quant à elle due à la déformation de la densité électronique d’un atome ou d’une molécule par l’effet du champ électrique d’une molécule voisine. Cette énergie est non additive et toujours attractive.\\
	
	Ces deux contributions sont très bien définies en physique classique, contrairement aux termes de dispersion et d'échange, qui sont liés à des effets quantiques.\\
	
	La répulsion émane du principe de Pauli, qui stipule que deux électrons ne peuvent occuper le même spin au sein d’une même orbitale. C’est une interaction toujours répulsive et qui apparaît seulement à courte distance.\\
	
	Enfin, la dispersion n’a pas d’équivalent classique puisqu'il s'agit d'une interaction liée à la corrélation électronique de deux densités de charge en interaction (fluctuation quantique des distributions de charges). Elle est attractive et existe dans tous les complexes.\\
	
	Il y a, de façon pratique, deux manières différentes de calculer une énergie d’interaction intermoléculaire. La première est la méthode dite « supermolécules », tandis que la seconde consiste en la construction de l’interaction intermoléculaire à partir de la connaissance des réponses des fonctions d’onde des monomères séparés (au sein du dimère) soumis à l’action de perturbations externes.\\
	
	%%%%%%%%%%%%%%%%%%%%%%%%%%%%%%%%%%%%%%%%%%%%%%
	%%%%%%%%%%%%%%%%%%%%%%%%%%%%%%%%%%%%%%%%%%%%%%
	%%%%%%%%%%%%%%%%%%%%%%%%%%%%%%%%%%%%%%%%%%%%%%
	\subsection{L’approche en « supermolécules »}
	%%%%%%%%%%%%%%%%%%%%%%%%%%%%%%%%%%%%%%%%%%%%%%
	%%%%%%%%%%%%%%%%%%%%%%%%%%%%%%%%%%%%%%%%%%%%%%
	%%%%%%%%%%%%%%%%%%%%%%%%%%%%%%%%%%%%%%%%%%%%%%
	
	Dans la méthode supermoléculaire, l’énergie d’interaction intermoléculaire est obtenue par différence entre l’énergie totale du dimère et la somme des énergies totales de chacun des monomères.
	
	\begin{equation}
	\Delta E = E_{AB} - E_{A} - E_{B} \label{eq2}
	\end{equation}
	
	\noindent où $A$ et $B$ représentent les monomères et $AB$ le dimère. En procédant ainsi, une erreur subtile peut être induite autour de l’énergie d’interaction. Cette erreur est connue sous le nom de \og Basis Set Superposition Error \fg{} (BSSE) \cite{sherrill2010counterpoise}. En calculant les deux monomères dans leur base spécifique, puis le dimère dans l’ensemble des fonctions de base des monomères, on peut utiliser les orbitales virtuelles d’un monomère pour agrandir la base disponible pour la distribution de charge de l’autre monomère, et vice versa. Il en résulte une augmentation de la qualité de la base pour le dimère vis-à-vis des monomères, et par conséquent une surestimation de l’énergie d'interaction. Afin de corriger l’erreur de BSSE, une possibilité consiste à travailler dans une base complète ou saturée pour les monomères et le dimère. Toutefois, la méthode la plus couramment employée reste la méthode dite \og counterpoise \fg, proposée par Boys et Bernardi \cite{boys1970calculation}, dans laquelle les énergies respectives des monomères sont calculées dans la base du dimère. Cette corrections implique donc le calcul, pour chaque distance intermoléculaire, de l’énergie totale du dimère et des monomères.\\
	
	L’énergie sans correction donnée pour l’équation \ref{eq2} peut-être modifiée pour l'estimation de la quantité pour laquelle le monomère A est stabilisé artificiellement pour la base supplémentaire du monomère B et vice versa, avec la relation suivante:
	
	\begin{equation}
	E_{BSSE}(A) = E_{A}^{AB}(A) - E_{A}^{A}(A)
	\end{equation}
	
	\begin{equation}
	E_{BSSE}(B) = E_{B}^{AB}(B) - E_{B}^{B}(B)
	\end{equation}
	
	\noindent où l'exposant désigne la base utilisée, l'indice désigne la géométrie, et le symbole entre parenthèses est le système chimique considéré.
	L'énergie d'interaction corrigée est donc donnée par l'expression suivante :
	
	\begin{equation}
	E^{CP} = E_{A-B}^{A-B}(A-B) - E_{A}^{A-B}(A) - E_{B}^{A-B}(B)
	\end{equation}
	
	Dans ces conditions, les électrons de chaque fragment du système bénéficient de la base d'orbitales des autres fragments. La correction de la BSSE est par définition la différence entre l'énergie d'interaction non corrigée et l'énergie corrigée :
	
	\begin{equation}
	\Delta E^{CP} = E - E^{CP} = (E_{A-B}^{A-B}(A-B) - E_{A}^{A}(A) - E_{B}^{B}(B)) - (E_{A-B}^{A-B}(A-B) - E_{A}^{A-B}(A) - E_{B}^{A-B}(B))
	\end{equation}
	
	\begin{equation}
	\Delta E^{CP} =  E_{A}^{A-B}(A) - E_{B}^{A-B}(B) - E_{A}^{A}(A) - E_{B}^{B}(B)
	\end{equation}
	
	Les bases ayant une extension spatiale finie, cette erreur est d'autant moins importante que les molécules sont éloignées les une des autres, la BSSE étant nulle dans la limite des bases infinies. En pratique, le calcul de la BSSE doit donc être effectué pour chaque conformation d’énergie minimale déterminée pour chacune des distances intermoléculaires qui séparent les espèces A et B. Plus la base est petite, plus cette erreur est importante. À l'inverse, dans la limite d’une base complète, la BSSE s’annule.
	Les calculs Hartree-Fock (HF) et Kohn-Sham (KS) souffrent moins des effets de la BSSE \cite{garza2005role} que les calculs post-HF (MPn, CCSD(T)), du fait en particulier du traitement de l’espace des états virtuels, nécessaire pour ces approches.\\
	
	
	Malheureusement, le calcul direct d’une énergie d’interaction par la méthode \og supermolécules \fg{} est extrêmement coûteux et ne donne aucune information sur la nature des interactions mises en jeu.\\
	
	%%%%%%%%%%%%%%%%%%%%%%%%%%%%%%%%%%%%%%%%%%%%%%
	%%%%%%%%%%%%%%%%%%%%%%%%%%%%%%%%%%%%%%%%%%%%%%
	%%%%%%%%%%%%%%%%%%%%%%%%%%%%%%%%%%%%%%%%%%%%%%
	\subsection{Détermination théorique des coefficients du développement multipolaire de l’énergie d'interaction à grande distance : potentiel électrostatique}
	%%%%%%%%%%%%%%%%%%%%%%%%%%%%%%%%%%%%%%%%%%%%%%
	%%%%%%%%%%%%%%%%%%%%%%%%%%%%%%%%%%%%%%%%%%%%%%
	%%%%%%%%%%%%%%%%%%%%%%%%%%%%%%%%%%%%%%%%%%%%%%
	
	
	La première théorie proposée en mécanique quantique pour déterminer la nature des interactions intermoléculaires a été développée dès 1927 par Wang \cite{wang1927mutual} puis reprise et très largement étendue à partir de 1930 par London \textit{et al.} \cite{london1930z}. Nous avons fait le choix, dans les paragraphes qui vont suivre, de ne mener le recensement des termes d’interaction introduits par London qu'aux plus bas ordres de perturbation, ceux-ci contenant les termes porteurs du sens physique. Notamment, le second ordre de perturbation contient \textit{a priori} les termes de distorsion et de dispersion que l’on peut naturellement relier aux propriétés électroniques des fragments en interaction. Les choix du modèle et de la troncature sont eux aussi totalement arbitraires, puisqu’il existe aujourd’hui d’autres modèles tels ceux de Kitaura et Morokuma \cite{morokuma1977molecules} ou de Stevens et Fink \cite{stevens1987frozen}, pour ne citer qu’eux. Le modèle perturbatif est celui qui, selon nous, permet la meilleure compréhension de la physique qui se cache derrière les forces de dispersion. Il présente l'avantage de rendre compte des coefficients dits de van der Waals ainsi que des fonctions de réponse qui leur sont associées telles que les polarisabilités multipolaires, elles-mêmes reliées aux fonctions spectrales telles que les forces d’oscillateur et les énergies d’excitation \cite{begue1999dynamic}. Enfin, le choix de présenter les termes d’interaction à partir de la théorie de perturbation nous semblait justifié par le fait qu’une fois les hypothèses de travail définies et les fonctions de réponse établies, la présentation des approches de type SAPT (Symmetry-Adapted Perturbation Theory) et des méthodes basées sur la détermination des fonctions semi-empiriques rendant compte des effets de dispersion s’en trouve facilitée et justifiée.
	
	
	Dans ce paragraphe consacré aux interactions à longues distances, nous allons donner les expressions générales du Hamiltonien d'interaction, et établir à l'aide de la méthode de perturbation les équations du problème.\\ 
	
	Soient deux systèmes $a$ et $b$ qui, à l'état isolé, sont susceptibles de se trouver dans les états $\phi_{0}^{a}$ ... $\phi_{p}^{a}$ et $\phi_{0}^{b}$ ... $\phi_{q}^{b}$, respectivement fonctions propres des Hamiltoniens $H_{0}(a)$ et $H_{0}(b)$, auxquels correspondent les énergies $E_{0}^{a}$ ... $E_{p}^{a}$ et $E_{0}^{b}$ ... $E_{q}^{b}$. Lorsque ces deux systèmes interagissent, on peut exprimer le Hamiltonien global $H$ du système comme : 
	
	\begin{equation}
	H = H_{0}(a) + H_{0}(b) + H'
	\end{equation}
	
	L'association [$a-b$] de ces deux systèmes engendre au niveau de $b$ un potentiel électrostatique $\Phi_{Kb}$ avec lequel les charges de $a$ interagissent, donnant naissance à une énergie d'interaction (et réciproquement au niveau de $a$).
	
	%%%%%%%%%%%%%%%%%%%%%%%%%%%%%%%%%%%%%%%%%%%%%%
	%%%%%%%%%%%%%%%%%%%%%%%%%%%%%%%%%%%%%%%%%%%%%%
	\subsubsection{Hamiltonien à l'ordre zéro pour le système [$a-b$]}
	%%%%%%%%%%%%%%%%%%%%%%%%%%%%%%%%%%%%%%%%%%%%%%
	%%%%%%%%%%%%%%%%%%%%%%%%%%%%%%%%%%%%%%%%%%%%%%
	
	A distance suffisamment grande pour que les distributions de charges liées aux systèmes $a$ et $b$ ne se recouvrent pas (de telle sorte que l'on puisse les considérer comme séparées), le Hamiltonien à l'ordre zéro associé au système [$a-b$] est simplement la somme des Hamiltoniens propres du système $a$ et du système $b$. Soit : 
	
	\begin{equation}
	H_{0} = H_{0}(a) + H_{0}(b) \label{1.2}
	\end{equation}
	
	\noindent avec : 
	
	\begin{equation}
	H_{0}(a) = -\frac{1}{2} \sum_{k_{a}=1}^{n_{a}} \triangledown^{2} k_{a} - \sum_{k_{a}=1}^{n_{a}} \frac{Z_{a}}{r_{k_{a}}} + \sum_{k'_{a}>k_{a}} \frac{1}{r_{k_{a},k'_{a}}} + H_{l.s}(a)  \label{1.3}
	\end{equation}
	
	Où $\triangledown_{k_{a}}$, $Z_{a}$, $r_{k_{a}}$ et $r_{k_{a},k'_{a}}$ représentent respectivement l'opérateur énergie cinétique, la charge du noyau de $a$, les distances noyau- électron $k_{a}$ et électron $k_{a}$- électron $k'_{a}$. L'expression de $H_{0}(b)$ se déduit facilement de celle de $H_{0}(a)$ en remplaçant les coordonnées des $n_{a}$ électrons du système $a$ par les coordonnées des $n_{b}$ électrons du système $b$. $H_{l.s}(a)= \sum_{k_{a}=1}^{n_{a}} \zeta_{k_{a}} \widehat{l_{k_{a}}} \widehat{S_{k_{a}}}$ est le hamiltonien de spin-orbite de l'atome $a$, avec $\widehat{l_{k_{a}}}$ et $\widehat{S_{k_{a}}}$ les opérateurs moment orbital et moment de spin de l'électron $k_{a}$, et $\zeta_{k_{a}}$ une constante de couplage spin-orbite pour le même électron.
	
		
	
	%%%%%%%%%%%%%%%%%%%%%%%%%%%%%%%%%%%%%%%%%%%%%%
	%%%%%%%%%%%%%%%%%%%%%%%%%%%%%%%%%%%%%%%%%%%%%%
	\subsubsection{Fonction d'onde moléculaire à l'ordre zéro}
	%%%%%%%%%%%%%%%%%%%%%%%%%%%%%%%%%%%%%%%%%%%%%%
	%%%%%%%%%%%%%%%%%%%%%%%%%%%%%%%%%%%%%%%%%%%%%%
	
	La structure électronique d'un système moléculaire et les propriétés qui en découlent peuvent être déterminées à partir de la résolution de l'équation de Schr\"{o}dinger. Pour un système constitué de $N$ électrons se déplaçant dans le champ électrostatique créé par les noyaux, cette équation s'écrit : $H\Psi = E\Psi$. La résolution de l'équation intégro-différentielle de Schr\"{o}dinger aux états stationnaires pour les systèmes pluriélectroniques n'est pas envisageable sans approximations. Les fonctions propres du Hamiltonien $H$ décrivent les états du système dont l'énergie est égale à la valeur propre correspondante. A partir de l'expression \ref{1.2} du Hamiltonien non perturbé, on peut exprimer la fonction d'onde $\Psi_{00}$ décrivant l'interaction entre deux atomes $a$ et $b$ comme : 
	
	\begin{equation}
	\Psi_{00} = \phi_{0}^{a} \cdot \phi_{0}^{b}
	\end{equation}
	
	\noindent qui n'est autre que le produit non antisymétrisé des fonctions d'onde $\phi_{0}^{a}$ et $\phi_{0}^{b}$ des systèmes $a$ et $b$ non perturbés. En toute rigueur, le simple produit des fonctions $\phi_{0}^{a}$ et $\phi_{0}^{b}$ n'est pas correct du point de vue des lois de la mécanique quantique puisque le principe d'indiscernabilité n'est pas vérifié. Néanmoins, pour de grandes distances interatomiques -- pour lesquelles les électrons ne peuvent s'échanger d'un atome à l'autre -- on peut raisonnablement négliger le recouvrement entre les fonctions d'onde atomiques (et par conséquent ne pas tenir compte des termes d'échange).\\
	
	Le dernier terme de l'expression \ref{1.3} -- $H_{l.s}(a) = \sum_{k_{a}=1}^{n_{a}} \zeta_{k_{a}} \widehat{l_{k_{a}}} \widehat{S_{k_{a}}}$ -- traduit le couplage spin-orbite engendré par l'interaction entre les deux dipôles magnétiques, ou plus précisément par le spin et le mouvement de chacun des électrons sur son orbite. Ce terme de couplage est ainsi de nature relativiste. Dans ce travail, nous négligerons les interactions des spins avec les orbites des autres spins, ainsi que les interactions spin-spin, approximations d'autant plus justifiées que le terme répulsif est faible. Cette approximation est dite \og non couplée \fg{} \cite{fontana1961theory,fontana1962theory} et les fonctions d'onde moléculaires correspondantes représentant le système [$a-b$] ont pour expression générale : 
	
	\begin{equation}
	\Psi_{00}^{\nu} = \sum_{k=1}^{n_{\nu}} \alpha_{\nu k}| S_{k} L_{k} M_{S_{k}} M_{L_{k}} \rangle _{a} | S'_{k} L'_{k} M'_{S_{k}} M'_{L_{k}} \rangle _{b}
	\end{equation}
	
	\noindent où $S_{k}$, $L_{k}$, $M_{S_{k}}$ et $M_{L_{k}}$ représentent respectivement les nombres quantiques de spin, orbitalaire et magnétique de l'atome $a$. Les termes $S'_{k}$, $L'_{k}$, $M'_{S_{k}}$ et $M'_{L_{k}}$ représentent de même les nombres quantiques associés au système $b$.
	
	%%%%%%%%%%%%%%%%%%%%%%%%%%%%%%%%%%%%%%%%%%%%%%
	%%%%%%%%%%%%%%%%%%%%%%%%%%%%%%%%%%%%%%%%%%%%%%
	\subsubsection{Perturbation}
	%%%%%%%%%%%%%%%%%%%%%%%%%%%%%%%%%%%%%%%%%%%%%%
	%%%%%%%%%%%%%%%%%%%%%%%%%%%%%%%%%%%%%%%%%%%%%%
	
	Deux atomes ou molécules à couches fermées engendrent toujours une énergie d'interaction due aux forces de van der Waals. Cette énergie d'interaction peut être calculée à l'aide d'une méthode de perturbation. En effet, aux grandes distances interatomiques $R>>R_{eq}$, l'énergie d'interaction est proche de l'énergie associée aux deux molécules placées à l'infini l'une de l'autre.  
	Tout le problème consiste donc à évaluer le déplacement de l'état fondamental du à l'introduction de $H'$, et en particulier sa dépendance à $R$. L'interaction coulombienne $V$, entre les électrons et le noyau de $a$ et les électrons et le noyau de $b$ sera traitée comme une perturbation, en considérant que $H'$ reste faible devant le Hamiltonien du système global isolé ($H_{0}(a)+ H_{0}(b)$). 
	
	
	\begin{equation}
	V = - \sum_{k_{a}=1}^{n_{a}} \frac{Z_{b}}{r_{bk_{a}}} - \sum_{k_{b}=1}^{n_{b}} \frac{Z_{a}}{r_{ak_{b}}} + \sum_{k_{a}=1}^{n_{a}} \sum_{k_{b}=1}^{n_{b}} \frac{1}{r_{k_{a}k_{b}}} + \frac{Z_{a} Z_{b}}{R}
	\end{equation}
	
	%%%%%%%%%%%%%%%%%%%%%%%%%%%%%%%%%%%%%%%%%%%%%%
	%%%%%%%%%%%%%%%%%%%%%%%%%%%%%%%%%%%%%%%%%%%%%%
	\subsubsection{Le Hamiltonien d'interaction}
	%%%%%%%%%%%%%%%%%%%%%%%%%%%%%%%%%%%%%%%%%%%%%%
	%%%%%%%%%%%%%%%%%%%%%%%%%%%%%%%%%%%%%%%%%%%%%%
	
	Soit un système dans lequel les charges sont localisées. Le Hamiltonien d'interaction $H'$ s'exprime en fonction du potentiel électrostatique ($\Phi_{K_{b}}= \sum_{k_{a}=1}^{N_{a}} \frac{e_{k_{a}}}{r_{k_{a}}}$) induit par les charges du système $a$ sur les charges du système $b$ (ou réciproquement par son équivalent $\Phi_{K_{a}}$ relatif au système $a$).
	
	\begin{equation}
	H' = \sum_{k_{b}=1}^{N_{b}} e_{k_{b}} \Phi_{k_{b}} = \sum_{k_{a}=1}^{N_{a}} \sum_{k_{b}=1}^{N_{b}} \frac{e_{k_{a}} e_{k_{b}}}{r_{k_{a}k_{b}}}
	\end{equation}
	
	En développant le potentiel $\Phi_{K_{b}}$ en série de Taylor par rapport au centre de la distribution de charges de $b$ (noté $o$) :
	
	\begin{equation}
	\Phi_{K_{b}} = \Phi_{o} + (\triangledown_{\alpha} \Phi)_{o} r_{K_{b}\alpha} + \frac{1}{2} (\triangledown_{\alpha} \triangledown_{\beta} \Phi)_{o} r_{K_{b}\alpha} r_{K_{b}\beta} + \ldots
	\end{equation}
	
	On obtient pour expression de $H'$ : 
	
	\begin{equation}
	H' = \sum_{K_{b}=1}^{N_{b}} e_{K_{b}} \left(\Phi_{o} + (\triangledown_{\alpha} \phi)_{o} r_{K_{b}\alpha} + \frac{1}{2}(\triangledown_{\alpha} \triangledown_{\beta}\phi)_{o} r_{K_{b}\alpha} r_{K_{b}\beta} + \ldots \right)
	\end{equation}
	
	\noindent où $\alpha,\beta=$ (x, y ou z) et où l'indice $k_{b}$ se rapporte indifféremment au noyau et aux électrons du système $b$ ($N_{b}$). $\triangledown_{\alpha}$ correspond à une des trois composantes de l'opérateur gradient ($\frac{\partial}{\partial x}$, $\frac{\partial}{\partial y}$ ou $\frac{\partial}{\partial z}$) et $r_{K_{b}\alpha}$ représente la composante rayon du vecteur décrivant la particule $k_{b}$. En tenant compte des définitions des moments multipolaires centrés sur l'origine du système : 
	
	\begin{flushleft}
		\begin{equation*}
		\sum_{k_{b}=1}^{N_{b}} e_{k_{b}} = q^{b} \hspace{8.1cm}\textup{charge électrique de la molecule $b$}     
		\end{equation*}
	\end{flushleft}
	
	\begin{flushleft}
		\begin{equation*}
		\sum_{k_{b}=1}^{N_{b}} e_{k_{b}} r_{k_{b}\alpha} = \mu_{\alpha}^{b}  \hspace{4.8cm} \textup{composante du moment dipolaire de $b$ à l'origine}
		\end{equation*}
	\end{flushleft}
	
	
	\begin{flushleft}
		\begin{equation*}
		\frac{1}{2} \sum_{k_{b}=1}^{N_{b}} e_{k_{b}} (3r_{k_{b}\alpha} r_{k_{b}\beta}- r^{2}_{k_{b}}\delta_{\alpha \beta}) = \theta_{\alpha \beta}^{b}  \hspace{1cm} \textup{élément $\alpha\beta$ du tenseur moment quadripolaire de $b$}
		\end{equation*}
	\end{flushleft}
	
	Ainsi que des définitions du champ :
	
	\begin{equation}
	F_{\alpha}^{b} = - \left(\frac{\partial \phi_{k_{b}}}{\partial r_{\alpha}}\right)_{o} = - (\triangledown_{\alpha} \phi_{k_{b}})_{o} \label{1.11}
	\end{equation}
	
	et du gradient du champ électrique en $o$ : 
	
	\begin{equation}
	F_{\alpha\beta}^{b} = - \left(\frac{\partial^{2} \phi_{k_{b}}}{\partial r_{\alpha} \partial r_{\beta}}\right)_{o} = - (\triangledown_{\alpha} \triangledown_{\beta} \phi_{k_{b}})_{o} \label{1.12}
	\end{equation}
	
	on retrouve l'expression classique du Hamiltonien d'interaction : 
	
	\begin{equation}
	H' = \Phi_{o} q^{b} - \sum_{\alpha} F_{\alpha}^{b} \mu_{\alpha}^{b} - \frac{1}{3} \sum_{\alpha\beta} F_{\alpha\beta}^{b} \theta_{\alpha\beta}^{b} + \ldots
	\end{equation}
	
	En posant le vecteur $\overrightarrow{R_{k}}= \overrightarrow{R} - \overrightarrow{r_{k}}$ qui représente la distance de la particule $k$ à l'origine de la distribution des charges, le potentiel $\Phi_{o}$ créé en $o$ peut s'écrire soit sous sa forme classique \ref{1.14}:
	
	\begin{equation}
	\Phi_{o} = \frac{q^{a}}{R} + \sum_{\alpha} \mu_{\alpha}^{a} \frac{R_{\alpha}}{R^{3}} + \sum_{\alpha\beta} \theta_{\alpha,\beta}^{a} \frac{R_{\alpha} R_{\beta}}{R^{4}} + \ldots \label{1.14}
	\end{equation}
	
	soit sous une forme plus simplifiée : 
	
	\begin{equation}
	\Phi_{o} = q^{a} T - \sum_{\alpha} \mu_{\alpha}^{a} T_{\alpha} + \sum_{\alpha\beta} \theta_{\alpha\beta}^{a} T_{\alpha\beta} + \ldots
	\end{equation}
	
        \noindent dans laquelle, les quantités $T$, $T_{\alpha}$ et $T_{\alpha\beta}$ sont posées égales à :
	
	\begin{equation}
	\begin{cases}
	T = R^{-1} \\
	T_{\alpha} = \triangledown_{\alpha} R^{-1} = - \frac{R_{\alpha}}{R^{3}}\\
	T_{\alpha\beta} = \triangledown_{\alpha} \triangledown_{\beta} R^{-1} = \frac{3R_{\alpha} R_{\beta}- R^{2}\delta_{\alpha\beta}}{R^{5}}
	\end{cases}
	\end{equation}
	
	$F_{\alpha}^{b}$ (\ref{1.11}) et $F_{\alpha\beta}^{b}$ (\ref{1.12}) sont exprimés en fonction des moments multipolaires \cite{buckingham1965general} précédemment définis : 
	
	\begin{equation}
	F_{\alpha}^{b} = -(\triangledown_{\alpha}\phi_{B})_{o} = -q^{a} T_{\alpha} + \mu_{\alpha}^{a} T_{\alpha\beta} - \frac{1}{3} \theta_{\beta\gamma}^{a} T_{\alpha\beta\lambda} + \ldots 
	\end{equation}
	
	\begin{equation}
	F_{\alpha\beta}^{b} = -(\triangledown_{\alpha} \triangledown_{\beta}\phi_{B})_{o} = q^{a} T_{\alpha\beta} + \mu_{\delta}^{a} T_{\alpha\beta\lambda} - \frac{1}{3} \theta_{\gamma\delta}^{a} T_{\alpha\beta\gamma\delta} + \ldots 
	\end{equation}
	
	Le Hamiltonien d'interaction devient alors : 
	
	\begin{equation}
	H' = q^{a} q^{b} T + T_{\alpha}(q^{a} \mu_{\alpha}^{a}) + T_{\alpha\beta} (\frac{1}{3}q^{a}\delta^{b}_{\alpha\beta}+ \frac{1}{3} q^{b}\delta^{a}_{\alpha\beta} - \mu_{\alpha}^{a}\mu_{\beta}^{b}) + \ldots  \label{1.19}
	\end{equation}
	
	\noindent et $H'$ apparaît comme la somme d'une infinité de termes : 
	
	\begin{itemize}
		\item contribution de la charge totale du système : $\frac{q^{a}q^{b}}{R}$. Ce terme représente le potentiel en ($1/R$) que créent les charges des systèmes $a$ et $b$.
		\item contribution du moment dipolaire électrique du système : interaction charge de $a$--moment dipolaire de $b$ et charge de $b$--moment dipolaire de $a$. Au total, on obtient un terme variant en $T_{\alpha}$ soit en ($1/R^{2}$).	
		\item un terme d'interaction dipôle--dipôle et deux termes charge de $a$--quadrupole de $b$ et charge de $b$--quadrupole de $a$, variant en ($1/R^{3}) \ldots$
	\end{itemize}
	
	La première contribution au potentiel d'interaction des moments multipolaires est généralement due au terme d'interaction dipôle--dipôle variant en ($1/R^{3}$). Aux grandes distances interatomiques, la contribution des termes au développement \ref{1.19} décroit très vite et seuls les premiers termes suffisent à exprimer le potentiel d'interaction. Nous verrons dans les applications les conséquences que cela entraîne sur le développement de l'énergie d'interaction, dont l'expression analytique sera obtenue à l'aide des termes de van der Waals ($C_{3}, C_{5}, C_{6}, \ldots$).\\
	
	
	%%%%%%%%%%%%%%%%%%%%%%%%%%%%%%%%%%%%%%%%%%%%%%
	%%%%%%%%%%%%%%%%%%%%%%%%%%%%%%%%%%%%%%%%%%%%%%
	\subsubsection{Développement à un centre du potentiel électrostatique dû à une distribution de charges discrètes.}
	%%%%%%%%%%%%%%%%%%%%%%%%%%%%%%%%%%%%%%%%%%%%%%
	%%%%%%%%%%%%%%%%%%%%%%%%%%%%%%%%%%%%%%%%%%%%%%
	
	Le système de coordonnées cartésiennes est le système le plus utilisé et le plus naturel puisqu'il s'appuie sur la tridimensionnalité de notre espace quotidien. Néanmoins, son utilisation devient très vite complexe lorsque l'on cherche à traiter des problématiques d'interaction. Il existe ainsi un système de référence plus adapté à l'étude de ce type de propriétés, qui s'appuie sur le principe simple de proportionnalité entre les harmoniques sphériques. Ce référentiel est le système de coordonnées sphériques. Nous étudierons dans ce chapitre les développements à un et deux centres du potentiel électrostatique dû à une distribution de charges discrètes.\\
	
	Soit $\Phi_{\rho}$, le potentiel électrostatique créé par une distribution de charges centrée en $a$, pour un point $\rho$ extérieur à cette distribution : 
	
	\begin{equation}
	\Phi_{\rho} = \sum_{k_{a}=1}^{n_{a}} \frac{e_{k_{a}}}{r_{k_{a}\rho}} \label{1.20}
	\end{equation}
	
	\noindent où $e_{k_{a}}$ représente la charge de la $k_{a}^{\text{ème}}$ particule et $r_{k_{a}\rho}$ la distance qui sépare cette charge du point $\rho$ considéré. Le développement de $\Phi_{\rho}$ sera effectué dans le système de coordonnées sphériques, en utilisant la formulation de Laplace pour le développement de la quantité $\frac{1}{r_{k_{a}\rho}}$.
	
	\begin{equation}
	\frac{1}{r_{k_{a}\rho}} = \sum_{i=0}^{\infty} \sum_{m=-i}^{+i} \frac{(i-|m|)!}{(i+ |m|)!} \frac{r^{i}<}{r^{i+1}>} P_{m}^{i} (\cos\theta_{k_{a}}) P^{i}_{m}(\cos\theta_{\rho})e^{im(\phi_{k_{a}}- \phi_{\rho})} \label{1.21}
	\end{equation}
	
	où $r <$ correspond à $\min \{r_{k_{a}}, r_{\rho}\} \equiv r_{k_{a}}$ et
	$r >$ correspond à $\max \{r_{k_{a}}, r_{\rho}\} \equiv r$
	et où les termes $P_{m}^{i} (\cos\theta_{\rho})$ représentent les fonctions associées au polynôme de Legendre de première espèce. Ces fonctions s'identifient aux harmoniques sphériques $Y_{i}^{m}(\theta,\phi)$ à un facteur près : 
	
	\begin{equation}
	P_{m}^{i}(\cos\theta) = (-1)^{m-|m|} \frac{\sqrt{4\pi}}{\sqrt{2i+ 1}} \frac{\sqrt{(i+ |m|)!}}{\sqrt{(i-|m|)!}} e^{-im\phi} Y_{i}^{m}(\theta,\phi) \label{1.22}
	\end{equation}
	
	D'après les équations \ref{1.21} et \ref{1.22}, le développement de Laplace pour l'inverse de la distance $r_{k_{a}\rho}$ entre la charge $q_{a}$ appartenant à la distribution $a$ et le point $\rho(r_{\rho}, \theta_{\rho}, \phi_{\rho})$ qui lui est extérieur, s'exprime comme : 
	
	\begin{equation}
	\frac{1}{r_{k_{a}\rho}} = \sum_{i=0}^{\infty} \sum_{m=-i}^{+i} (-1)^{|m|} \frac{4\pi}{2i +1} \frac{r_{k_{a}}^{i}}{r_{\rho}^{i+1}} Y^{m}_{i} (\theta_{k_{a}}, \phi_{k_{a}}) Y_{i}^{-m} (\theta_{\rho},\phi_{\rho})  \label{1.23}
	\end{equation}
	
	En remplaçant l'expression \ref{1.23} dans la formulation du potentiel électrostatique \ref{1.20}, l'expression de $\Phi_{\rho}$ obtenue dépend de $\theta$ et $\phi$ : 
	
	\begin{equation}
	\Phi_{\rho} = \sum_{i=0}^{\infty} \sum_{m=-i}^{+i} (-1)^{|m|}\frac{4\pi}{2i +1} \frac{1}{r_{\rho}^{i+1}} Y_{i}^{-m} (\theta_{\rho},\phi_{\rho}) \sum_{k_{a}} e_{k_{a}}r^{i}_{k_{a}} Y^{m}_{i} (\theta_{k_{a}}, \phi_{k_{a}})
	\end{equation}
	
	Dans un souci de simplification, nous poserons la variable $Q_{m}^{i}(a)$ égale à : 
	
	\begin{equation}
	Q_{m}^{i}(a)= \frac{\sqrt{4\pi}}{\sqrt{2i + 1}} \sum_{k_{a}} e_{k_{a}}r_{k_{a}}^{i} Y_{i}^{m} (\theta_{k_{a}}, \phi_{k_{a}}) \label{1.25}
	\end{equation}
	
	\noindent d'où il vient que :
	
	\begin{equation}
	\Phi_{\rho} = \sum_{i=0}^{\infty} \sum_{m=-i}^{+i} (-1)^{|m|} \frac{4\pi}{2i +1} \frac{1}{r_{\rho}^{i+1}} Y_{i}^{-m} (\theta_{\rho},\phi_{\rho})Q_{m}^{i}(a)
	\end{equation}
	
	Physiquement, les quantités $Q_{m}^{i}(a)$ représentent les combinaisons linéaires des composantes des opérateurs moment dipolaire du système $a$. 
	
	\begin{itemize}
		\item i=0, m=0 \hspace{0.9cm} $Q_{0}^{0}(a) = \sum_{k_{a}} e_{k_{a}} = q^{a}$ \hspace{3.5cm} charge de $a$
		\item i=1, m=0 \hspace{0.9cm} $Q_{_{0}^{1}}(a)= \sum_{k_{a}} e_{k_{a}} r_{k_{a}} P^{0}_{1} (\cos\theta_{k_{a}}) = \mu_{z}^{a}$ \hspace{1cm} composante z de l'opérateur	
		\item i=1, m=1 \hspace{0.9cm} $Q_{1}^{1}(a)= \sum_{k_{a}} \frac{1}{\sqrt{2}}e_{k_{a}} r_{k_{a}} P^{1}_{1} (\cos\theta_{k_{a}}) e^{i\phi k_{a}}$ \hspace{0.6cm} moment dipolaire de l'atome $a$
		\item i=1, m=-1 \hspace{0.9cm} $Q_{-1}^{1}(a) = -\sum_{k_{a}} \frac{1}{\sqrt{2}}e_{k_{a}} r_{k_{a}} P^{1}_{-1} (\cos\theta_{k_{a}}) e^{-i\phi k_{a}}$
	\end{itemize}
	
	Si l'on considère que l'axe $z$ est l'axe de plus haute symétrie, les formulations des composantes perpendiculaires $x$ et $y$ du moment dipolaire électrique du système s'expriment de la façon suivante : 
	
	\begin{equation}
	\mu_{x}^{a} = \frac{1}{\sqrt{2}} (Q_{1}^{1}(a)- Q_{-1}^{1}(a))
	\end{equation}
	
	\begin{equation}
	\mu_{y}^{a} = \frac{1}{\sqrt{2i}} (Q_{1}^{1}(a)+ Q_{-1}^{1}(a))
	\end{equation}
	
	Les autres composantes du tenseur d'ordre deux seront obtenues en remplaçant $m$ par ses valeurs permises ($m=\pm 1, m=\pm 2$).
	
	Les propriétés de ces opérateurs sont indépendantes du système étudié et satisfont à un ensemble de règles de sélection sur lesquelles nous ne reviendrons pas dans ce travail \cite{hall2015lie}. 
	
	\begin{itemize}
		\item i=2, m=0 \hspace{1cm} $Q_{0}^{2}(a) = \theta_{zz}^{a}$ \hspace{1cm} composante $z$ de l'opérateur moment quadripolaire 
	\end{itemize}
	
	%%%%%%%%%%%%%%%%%%%%%%%%%%%%%%%%%%%%%%%%%%%%%%
	%%%%%%%%%%%%%%%%%%%%%%%%%%%%%%%%%%%%%%%%%%%%%%
	\subsubsection{Développement à deux centres de l'interaction coulombienne entre deux distributions de charges ne se recouvrant pas}
	%%%%%%%%%%%%%%%%%%%%%%%%%%%%%%%%%%%%%%%%%%%%%%
	%%%%%%%%%%%%%%%%%%%%%%%%%%%%%%%%%%%%%%%%%%%%%%
	
	De nombreux travaux \cite{buehler1951bipolar,hylleraas1931elektronenterme,proctor1977long,davison1968atomic} ont montré que l'inverse de la distance $r_{ij}$ existant entre une charge $i$ et une charge notée $j$ peut s'écrire sous forme d'un développement bipolaire dans le système de coordonnées de deux distributions (système de coordonnées bipolaires).
	
	\begin{equation}
	\frac{1}{r_{k_{a}k_{b}}} = \sum_{i,j=0}^{\infty} \sum_{-l<}^{l>} B_{ij}^{|m|}(r_{k_{a}}, r_{k_{b}}; R) P_{i}^{m} (\cos\theta_{k_{b}}) P_{j}^{m}(\cos\theta_{k_{a}}) e^{im(\phi_{k_{a}}-\phi_{k_{b}})}
	\end{equation}
	
	\noindent où $l<$ = $\inf (i,j)$
	
	
	Dans l'hypothèse où les deux distributions de charges ne se recouvrent pas, on peut considérer que $R> r_{i} +r_{j}$ et poser : 
	
	\begin{equation}
	B_{ij}^{|m|}(r_{k_{a}}, r_{k_{b}}; R) = \frac{(1-)^{j+|m|} (i+j)!}{(i+|m|)! (j+|m|)!} r^{i}_{k_{a}} r_{k_{a}}^{j} \frac{1}{R^{i+j+1}}
	\end{equation}
	
	Le potentiel d'interaction entre les deux distributions de charges s'écrit : 
	
	\begin{equation}
	V = \sum_{k_{a}=1}^{N_{a}} \sum_{k_{b}=1}^{N_{b}} \frac{e_{k_{a}}e_{k_{b}}}{r_{k_{a}k_{b}}}
	\end{equation}
	
	
	Le polynôme de Legendre associé s'exprime en fonction des harmoniques sphériques : 
	
	\begin{equation}
	V = \sum_{i,j=0}^{\infty} \frac{1}{R^{i+j+1}} \sum_{-l<}^{l>} \frac{(-1)^{j} (i+j)!} {\sqrt{(i+m)! (i-m)! (j-m)! (j+m)!}}
	\end{equation}
	
	\begin{equation}
	\frac{\sqrt{4\pi}}{\sqrt{2i}+ 1} \sum_{k_{a}=1}^{N_{a}} e_{k_{a}} r_{k_{a}}^{i} Y_{i}^{m} (\theta_{k_{a}},\phi_{k_{a}}) \frac{\sqrt{4\pi}}{\sqrt{2j+ 1}} \sum_{k_{b}=1}^{N_{b}} e_{k_{b}}r_{k_{b}}^{j} Y_{j}^{-m}(\theta_{k_{b}},\phi_{k_{b}})
	\end{equation}
	
	\noindent où $V$ se développe en fonction des opérateurs moments multipolaires de $a$ et de $b$, définis par la relation \ref{1.25}.
	
	En posant la quantité $\frac{(-1)^{j} (i+j)!} {\sqrt{(i+m)! (i-m)! (j-m)! (j+m)!}}=d_{m}(i,j)$ on obtient pour expression simplifiée du potentiel d'interaction l'équation générale suivante : 
	
	\begin{equation}
	V = \sum_{i,j=0}^{\infty} \frac{1}{R^{i+j+1}} \sum_{-l<}^{l>} d_{m}(i,j) Q_{m}^{i}(a) Q_{-m}^{j} (b) = \sum_{i=0}^{\infty} \sum_{j=0}^{\infty} \frac{V_{ij}(a,b)}{R^{i+j+1}} \label{1.33}
	\end{equation}
	
	Les opérateurs multipolaires $Q_{m}^{i;j}$ (charge, dipôle, quadripôle,...) contiennent toute l'information sur la distribution de charges du système considéré. Le potentiel $V(a,b)$ contient donc chacune de ces informations pour chacun des deux systèmes $a$ et $b$ pris isolément. En prenant l'ensemble des valeurs de $i$ et de $j$ permises, les différentes interactions intervenant entre les multipôles (induits ou permanents) des deux systèmes sont :
	
	\begin{itemize}
		\item $i=j=0$ : $V_{00}$ représente l'interaction entre la charge de $a$ et la charge de $b$. Ce terme est nul pour les systèmes neutres. 	
		\item $i=0$ et $j=1$ : $V_{01}$ représente l'interaction entre la charge de $a$ et le dipôle de $b$. 	
		\item $i=j=1$ : $V_{11}$ représente l'interaction entre le dipôle de $a$ et le dipôle de $b$.
	\end{itemize}
	
	$i$ et $j$ prennent des valeurs théoriquement infinies, mais nous verrons que l'expression \ref{1.33} peut être tronquée pour des valeurs de $i$ et $j$ faibles. De plus, suivant la nature des espèces en présence, toutes les combinaisons des paramètres $i$ et $j$ ne sont pas permises.
	
	
	
	
	%%%%%%%%%%%%%%%%%%%%%%%%%%%%%%%%%%%%%%%%%%%%%%
	%%%%%%%%%%%%%%%%%%%%%%%%%%%%%%%%%%%%%%%%%%%%%%
	\subsubsection{Calcul de l’énergie électrostatique.}
	%%%%%%%%%%%%%%%%%%%%%%%%%%%%%%%%%%%%%%%%%%%%%%
	%%%%%%%%%%%%%%%%%%%%%%%%%%%%%%%%%%%%%%%%%%%%%%
	
	
	Les deux parties qui vont suivre seront consacrées au recensement des coefficients du développement multipolaire de l’énergie d'interaction à grande distance (énergie d’interaction décomposée en l'énergie électrostatique et en l'énergie de polarisation) obtenus à partir d'un calcul au premier et au second ordre de perturbation. Nous présenterons également les termes prépondérants d'ordre 6 et 8 que nous pouvons estimer \cite{saute1982calculated}.
	
	%%%%%%%%%%%%%%%%%%%%%%%%%%%%%%%%%%%%%%%%%%%%%%
	\paragraph{Expression des éléments de matrice de l'énergie électrostatique}
	%%%%%%%%%%%%%%%%%%%%%%%%%%%%%%%%%%%%%%%%%%%%%%
	
	\begin{equation}
	E_{1} = \langle \psi_{00}^{\nu}|V| \psi_{00}^{\lambda}\rangle
	\end{equation}
	
	
	En remplaçant l'opérateur $V_{ij}$ de perturbation par son expression \ref{1.33}, l'énergie électrostatique d'un système [$a-b$] s'exprime sous la forme d'une suite infinie : 
	
	\begin{equation}
	E = \sum_{i=0}^{\infty} \sum_{j=0}^{\infty} \frac{E_{ij} (\nu,\lambda)}{R^{i+j+1}} = \sum_{m} \frac{C_{m}}{R^{m}}
	\end{equation}
	
	\noindent dans laquelle les coefficients $C_{m}$ représentent directement les éléments de matrice $E_{ij}$ de l'énergie d'interaction électrostatique entre deux états $\nu$ et $\lambda$ : 
	
	\begin{equation}
	E_{ij}(\nu, \lambda) = \langle \psi_{00}^{\nu}|V_{ij}(a,b)|\psi_{00}^{\lambda} \rangle = C_{m}
	\end{equation}
	
	\noindent avec $m= i+j+1$ \\
	En reprenant les expressions des fonctions d'onde :
	
	\begin{equation}
	F_{l}^{m} (\theta,\phi) = \lambda (l)Y_{l}^{m} (\theta,\phi)
	\end{equation}
	
	Les éléments de matrice $E_{ij}$ s'expriment en fonction du terme d'interaction induit par les multipôles électrostatiques de $a$ et de $b$ : 
	
	\begin{equation*}
	E_{ij} (\nu , \lambda) = \sum_{k=1}^{n_{\nu}} \sum_{l=1}^{n_{\lambda}} \alpha_{\nu k} \alpha_{\lambda l}
	\end{equation*}
	
	\begin{equation}
	\langle S_{k}L_{k}M_{S_{k}}M_{L_{k}}|\langle S'_{k}L'_{k}M'_{S_{k}}M'_{L_{k}}| V_{ij}(a,b)| S_{l}L_{l}M_{S_{l}} M_{L_{l}} \rangle| S'_{l}L'_{l}M'_{S_{l}}M'_{L_{l}} \rangle
	\end{equation}
	
	\noindent avec : 
	
	\begin{equation}
	V_{ij}(a,b) = \sum_{m=-l<}^{l>} d_{m}(i,j) Q_{m}^{i}(a) Q_{-m}^{j}(b) 
	\end{equation}
	
	Le terme $V_{ij}$ apparaît comme le produit d'un opérateur tensoriel irréductible de rang $i$, agissant sur les seules coordonnées de $a$, par l'opérateur tensoriel irréductible de rang $j$ n'agissant que sur les coordonnées de $b$. Dans le cadre de cette propriété, les éléments de matrice de l'énergie électrostatique se décomposent en deux parties : une partie relative au système $a$ et l'autre relative au système $b$. 
	
	\begin{equation*}
	E_{ij}(\nu , \lambda)= \sum_{k=1}^{n_{\nu}} \sum_{l=1}^{n_{\lambda}} \alpha_{\nu k} \alpha_{\lambda l} \sum_{m=-l<}^{l>} d_{m}(i,j)
	\end{equation*}
	
	\begin{equation}
	\langle S_{k}L_{k}M_{S_{k}}M_{L_{k}}| Q_{m}^{i} (a)| S_{l}L_{l}M_{S_{l}}M_{L_{l}}\rangle \langle S'_{k}L'_{k}M'_{S_{k}}M'_{L_{k}}| Q_{-m}^{j} (b) | S'_{l}L'_{l}M'_{S_{l}}M'_{L_{l}}\rangle
	\end{equation}
	
	%%%%%%%%%%%%%%%%%%%%%%%%%%%%%%%%%%%%%%%%%%%%%%
	\paragraph{Règles de sélection pour la détermination des coefficients de dispersion $C_{m}$}
	%%%%%%%%%%%%%%%%%%%%%%%%%%%%%%%%%%%%%%%%%%%%%%
	
	Les règles de sélection nous ont conduits aux relations :
	
	\begin{equation}
	\begin{cases}
	L_{k} + i + L_{l} = entier\ pair \\
	|L_{k} - L_{l}| \leq i \leq L_{k} + L_{l} \\
	L'_{k} + j + L'_{l} = entier\ pair \\
	|L'_{k} - L'_{l}| \leq j \leq L'_{k} + L'_{l}
	\end{cases}
	\end{equation}
	
	\begin{itemize}
		\item Si $L_{k} + L_{l}$ est pair, $i$ est nécessairement pair (idem pour $j$). Il s'ensuit que $i + j$ est pair.	
		\item Si $L_{k} + L_{l}$ est impair, $i$ est nécessairement impair (idem pour j). Il en découle que $i + j$ est pair.
	\end{itemize}
	
	Ainsi, quelles sue soient les valeurs des moments orbitaux, $i + j$ est toujours pair. L'énergie électrostatique : 
	
	\begin{equation}
	E(l) = \sum_{n=0}^{\infty} \frac{C_{i+j+1}}{R^{i+j+1}}
	\end{equation}
	
	sera donc toujours représentée par un développement en puissance impaire ($i+ j + 1$) de $R$.
	
	\begin{itemize}
		\item pour $L_{k}=L_{l}=L'_{k}=L'_{l} = 0$, nous aurons donc $i=j=0$ ; interaction charge-charge ($Q_{0}^{0}- Q_{0}^{0}$)	
		\item pour $L_{k}=L_{l}=0$, $L'_{k}=L'_{l}=1$, les sommes $i+1$ et $j+1$ sont entières et paires. Soit $i=j=1$, alors l'énergie électrostatique possède un terme de résonance non nul d'ordre trois : $C_{3}/R^{3}$.
		\item pour $L_{k}=L_{l}=L'_{k}=L'_{l}=1$, les sommes $i+2$ et $j+2$ sont entières et paires. Ces règles impliquent que $i=j=2$. L'énergie électrostatique possède un terme d'ordre cinq non nul : $C_{5}/R^{5}$.
	\end{itemize}
	
	
	%%%%%%%%%%%%%%%%%%%%%%%%%%%%%%%%%%%%%%%%%%%%%%
	%%%%%%%%%%%%%%%%%%%%%%%%%%%%%%%%%%%%%%%%%%%%%%
	\subsubsection{Calcul de l’énergie de polarisation}
	%%%%%%%%%%%%%%%%%%%%%%%%%%%%%%%%%%%%%%%%%%%%%%
	%%%%%%%%%%%%%%%%%%%%%%%%%%%%%%%%%%%%%%%%%%%%%%
	
	L'énergie de polarisation représente l'énergie au second ordre des perturbations. Elle se compose de deux termes : un terme de dispersion impliquant simultanément les états excités de $a$ et de $b$, et un terme d'induction ne faisant intervenir qu'un seul ensemble d'états excités issu de l'un des deux systèmes. En reprenant l'expression \ref{1.33} de l'opérateur de perturbation, on obtient pour expression de l'énergie au second ordre : 
	
	\begin{equation}
	E(2) = \sum_{i,j=0}^{\infty} \sum_{i',j'=0}^{\infty} \frac{E_{disp}^{ii'jj'} (\nu , \lambda) + E_{induc}^{ii'jj'}(\nu , \lambda)}{R^{i+i'+j+j'+2}} = \sum_{i+i'+j+j'+2=2}^{\infty} \frac{C_{i+i'+j+j'+2}}{R^{i+i'+j+j'+2}} \label{2.21}
	\end{equation}
	
	Le domaine de validité d'un tel développement est limité pour les très grandes distances. Pour les faibles distances où le recouvrement orbitalaire ne peut plus être négligé, l'équation \ref{2.21} n'est plus valable et l'on doit prendre en compte l'énergie de répulsion. Pour des distances comprises entre ces deux domaines, où il n'existe généralement que très peu d'informations expérimentales, les méthodes dites asymptotiques peuvent apporter de précieux résultats.
	
	
	%%%%%%%%%%%%%%%%%%%%%%%%%%%%%%%%%%%%%%%%%%%%%%
	\paragraph{Energie de dispersion}
	%%%%%%%%%%%%%%%%%%%%%%%%%%%%%%%%%%%%%%%%%%%%%%
	
	L'énergie de dispersion fait intervenir simultanément les états excités de $a$ et de $b$, que nous noterons respectivement $\Psi_{r}(a)$ et $\Psi_{t}(b)$.
	
	\begin{equation}
	E_{disp}^{ii'jj'} (\nu , \lambda) = - \sum_{r0}^{\infty} \sum_{t0}^{\infty} \frac{\langle \Psi_{0}^{\nu} (a) \Psi_{0}^{\nu} (b)| V_{ij}|\Psi_{r} (a) \Psi_{t} (b) \rangle  \langle \Psi_{r} (a) \Psi_{t} (b) |V_{i'j'}| \Psi_{0}^{\lambda} (a) \Psi_{0}^{\lambda}(b) \rangle}{(E_{r} (a) - E_{0} (a)) + (E_{t} (b) - E_{0} (b))} \label{2.22}
	\end{equation}
	
	En reprenant la forme générale des fonctions d'onde dépendantes des nombres quantiques orbitaux, magnétiques et de spin de chacun des systèmes pris isolément, l'équation \ref{2.22} peut s'écrire : 
	
	\begin{equation}
	\begin{array}{c}
	E_{disp}^{ii'jj'} (\nu , \lambda) = - \sum_{k=1}^{n_{\nu}} \sum_{l=1}^{n_{\lambda}} \alpha_{\nu , k} \alpha_{\lambda , l} \hspace{1cm} \times \\
	\\
	\sum_{w"_{a} w"_{b}} \frac{\langle S_{k} L_{k} M_{S_{K}} M_{L_{K}} (a) |\langle S'_{k} L'_{k} M'_{S_{K}} M'_{L_{K}} (b)| V_{ij} (ab) | S"_{a} L"_{a} M"_{S_{a}} M"_{L_{a}} (a)\rangle| S"_{b} L"_{b} M"_{S_{b}} M"_{L_{b}(b)} \rangle}{(E_{n_{a}^{"}} - E_{0}(a)) + (E_{n_{b}^{"}} - E_{0}(b))} \hspace{0.5 cm} \times \\
	\\
	\frac{\langle S"_{a} L"_{a} M"_{S_{a}} M"_{L_{a}} (a)| \langle S"_{b} L"_{b} M"_{S_{b}} M"_{L_{b}} (b) | V_{i'j'}(a,b)| S_{l} L_{l} M_{S_{l}} M_{L_{l}} (a) \rangle | S'_{l} L'_{l} M'_{S_{l}} M'_{L_{l}} (b) \rangle}{(E_{n_{a}^{"}} - E_{0} (a)) + (E_{n_{b}^{"}} - E_{0}(b))}
	\end{array}
	\end{equation}
	
	\noindent avec $w"_{a} = (n"_{a}, l"_{a}, S"_{a}, L"_{a}, M"_{S_{a}}, M"_{L_{a}})$. L'indice \og seconde \fg{} sera réservé aux états excités. 
	
	Comme pour l'énergie électrostatique, la démarche adoptée consiste à séparer les parties relatives au système $a$ de celles relatives au système $b$. \\
	
	Par application des règles de sélection sur les termes de Wigner qui composent la partie purement radiale, on obtient les valeurs de $i, j, i'$ et $j'$ permises :
	
	\begin{equation}
	\begin{cases}
	l_{k} + i + l"_{a} = entier\ pair \\
	|l_{k} - l"_{a}| \leq i \leq l_{k} + l"_{a} \\
	l"_{a} + i' + l_{l} = entier\ pair \\
	|l"_{a} - l_{l}| \leq i' \leq l"_{a} + l_{l} \\
	l'_{k} + j + l"_{b} = entier\ pair \\
	|l'_{k} - l"_{b}| \leq j \leq l'_{k} + l"_{b} \\
	l'_{l} + j' + l"_{b} = entier\ pair \\
	|l'_{l} - l"_{b}| \leq j' \leq l'_{l} + l"_{b}
	\end{cases}
	\end{equation}
	
	Par exemple, pour $l_{k}=l_{l}=l’_{k}=l’_{l}=0$, $i=i'=l"_{a}$ et $j=j'=l"_{b}$. L'ensemble des combinaisons $(i, i', j, j')$ permises sont telles que $i=i'$ et $j=j'$. En tenant compte de la définition de l'énergie de dispersion \ref{2.21},
	
	\begin{equation}
	E_{disp}(2) = \sum_{i+i'+j+j'+2}^{\infty} \frac{C_{i+i'+j+j'+2}}{R^{i+i'+j+j'+2}}
	\end{equation}
	
	il est possible de dénombrer et connaître les différentes contributions des coefficients de dispersion (encore appelés coefficients de van der Waals) à la correction énergétique au second ordre sur l'énergie du système [$a - b$].
	
	La combinaison (1,1,1,1) représente la contribution du coefficients $C_{1+1+1+1+2} = C_{6}$.
	
	Les combinaisons (1,1,2,2) et (2,2,1,1) représentent les deux contributions au coefficient $C_{8}$, etc..
	
	
	%%%%%%%%%%%%%%%%%%%%%%%%%%%%%%%%%%%%%%%%%%%%%%
	\paragraph{Energie d’induction}
	%%%%%%%%%%%%%%%%%%%%%%%%%%%%%%%%%%%%%%%%%%%%%%
	
	Contrairement à l'énergie de dispersion, l'énergie d'induction ne fait intervenir que les états excités de l'un ou l'autre des systèmes $a$ et $b$ : 
	
	\begin{equation}
	E_{ind}(2) = \sum_{i,j=0}^{\infty} \sum_{i',j'=0}^{\infty} \frac{E_{ind}^{ii'jj'} (a; \nu , \lambda) + E_{ind}^{ii'jj'} (b; \nu , \lambda)}{R^{i+i'+j+j'+2}} = \sum_{i+j+j'+i'+1=2}^{\infty} \frac{C_{i+i'+j+j'+2}}{R^{i+i'+j+j'+2}} \label{2.42}
	\end{equation}
	
	Soit par exemple l'énergie d'induction du système $a$ :
	
	\begin{equation}
	E_{ind}^{ii'jj'} (a;\nu , \lambda) = \sum_{r \neq 0}^{\infty} \sum_{t\neq 0}^{\infty} \frac{\langle \Psi_{0}^{\nu} (a) \Psi_{0}^{\nu} (b) |V_{ij} (a,b)|\Psi_{r} (a) \Psi_{0} (b) \rangle  \langle \Psi_{r} (a)\Psi_{0} (b) |V_{i'j'} (a,b)|\Psi_{0}^{\lambda} (a) \Psi_{0}^{\lambda}(b) \rangle}{(E_{r} (a) - E_{0}(a)}
	\end{equation}
	
	\noindent avec $\Psi_{0}^{\nu} (a) = |S_{k} L_{k} M_{S_{k}} M_{L_{k}}\rangle$ et $\Psi_{r}(a) = |S"_{a} L"_{a} M"_{S_{a}} M"_{L_{a}}\rangle$. On pourra trouver de même une formule similaire $E_{ind}^{ii'jj'} (b;\nu , \lambda)$ pour le système $b$. Chaque terme est calculé séparément.\\
	
	Les règles de sélection concernant le système $a$ sont identiques à celles énoncées pour l'énergie de dispersion.
	Les règles de sélection propres au système $b$ sont : 
	
	\begin{equation}
	\begin{cases}
	L_{b} + j + L'_{k} = entier\ pair \\
	|L'_{k} - L_{b}| \leq j \leq L'_{k} + L_{b} \\
	L_{b} + j' + L'_{l} = entier\ pair \\
	|L'_{l} - L_{b}| \leq j' \leq L'_{l} + L_{b}
	\end{cases}
	\end{equation}
	
	\begin{itemize}
		\item pour $l_{k}=l_{l}=0$ et  $L'_{k}=L'_{l}=0$, on a $i=i'=l"_{a}$ et $j=j'=0$. Les combinaisons (i,i',j,j') permises seront celles pour lesquelles $i = i'$; (i,i,0,0). La somme $i + i'$ est paire. Les termes d'induction sont donc pairs d'ordre 4. Ces termes représentent les interactions dipôle de $a$--charge de $b$, quadrupôle de $a$--charge de $b$, etc., et sont toujours nuls dans le cas de systèmes neutres. Il est évident que le résultat demeure identique en considérant les états excités de $b$ plutôt que ceux de $a$. 
		\item pour $l_{k} = l_{l} = 1 $ et  $L'_{k} = L'_{l} = 0$, les règles de sélection conduisent pour $b$ aux mêmes conclusions que dans le cas précédent (j = j' = 0). Ce terme est toujours nul dans le cas des atomes neutres.
		\item pour $l_{k} = l_{l} = 0 $ et  $L'_{k} = L'_{l} = 1$, les règles de sélection conduisent pour $b$ à $j=0, 2$ et $j'=0, 2$ et pour $a$ à $i = i'$. Dans ce cas, les combinaisons permises sont (1,1,2,2) pour le terme d'ordre 8, (2,2,2,2) pour le terme d'ordre 10, etc. Notons que la valeur 0 est interdite pour les systèmes neutres. Le premier terme qui contribue à la valeur de l'énergie d'induction est donc le terme d'ordre 8 ($C_{8}$) et cette énergie s'exprime comme : 
		
		\begin{equation}
		E_{ind}^{ii'22} (a;\nu , \lambda) = \sum_{i=i'=1}^{\infty} \frac{C_{i+i'+6}}{R^{i+i'+6}} \hspace{2cm} avec \hspace{0.5cm} j = 2
		\end{equation}
		
	\end{itemize}
	
	\ldots
	
	
	%%%%%%%%%%%%%%%%%%%%%%%%%%%%%%%%%%%%%%%%%%%%%%
	%%%%%%%%%%%%%%%%%%%%%%%%%%%%%%%%%%%%%%%%%%%%%%
	%%%%%%%%%%%%%%%%%%%%%%%%%%%%%%%%%%%%%%%%%%%%%%
	\subsection{Symmetry-Adapted Perturbation Theory (SAPT)}
	%%%%%%%%%%%%%%%%%%%%%%%%%%%%%%%%%%%%%%%%%%%%%%
	%%%%%%%%%%%%%%%%%%%%%%%%%%%%%%%%%%%%%%%%%%%%%%
	%%%%%%%%%%%%%%%%%%%%%%%%%%%%%%%%%%%%%%%%%%%%%%
	\markright{Symmetry-Adapted Perturbation Theory}{}
	
	
	Comme nous l’avons déjà mentionné, les interactions de van der Waals ne se distinguent clairement des autres types d’interactions qu’à longue distance. Dans la région intermédiaire, qui inclut la région du minimum, il est même particulièrement difficile de décomposer de manière univoque l’énergie d'interaction totale en différentes contributions. Plusieurs schémas de décomposition ont néanmoins été proposés au fil du temps (comme celui présenté ci-avant). Le principe de ces dernières approches est d’évaluer par morceaux les différentes contributions, telles que l’électrostatique, l’induction ou encore la dispersion. Cependant, ces méthodes possèdent le défaut majeur de conduire à des processus non convergés (ou mal convergés) de la série perturbationnelle, tant du fait d'une troncature à des ordres souvent trop bas que du fait de leur non complétude. C’est pour palier ce dernier défaut que la théorie SAPT (Symmetry-Adapted Perturbation Theory) a été proposée. Dans cette approche, chacune des composantes précédentes inclut une correction d’échange afin de rendre compte de la nature anti-symétrique de la fonction d’onde du complexe formé. Cette théorie a par ailleurs été développée en partant du constat que le terme de polarisation se révèle en grande partie responsable de la divergence de la série perturbative, ce que vise à corriger l'introduction d'une correction d'échange. \\ 
		
	L’approche SAPT est en quelque sorte une théorie de double perturbation : l’une concerne le potentiel intermoléculaire, tandis que l’autre traite du potentiel de corrélation intra-moléculaire. Afin de comprendre cette remarque, il est nécessaire de développer \textit{a minima} les principes mathématiques sur lesquels repose la théorie SAPT.\\
	
	Un moyen simple de présenter cette méthode est de considérer une partition classique du Hamiltonien d'un complexe moléculaire constitué de deux monomères A et B :
	
	\begin{equation}
	H = H^{A} + H^{B} + V
	\end{equation}
	
	\noindent dans laquelle le terme $V$ représente la perturbation intermoléculaire.\\
	
	Le Hamiltonien total s'écrit encore :
	
	\begin{equation}
	H = H_{0}^{A} + W^{A} + H_{0}^{B} + W^{B} + V
	\end{equation}
	
	\noindent dans lequel les termes $H_{0}^{A/B}$ sont les opérateurs de Fock des monomères A et B, $W^{A/B}$ représentent les termes de perturbation intramoléculaire des monomères et traduisent donc la corrélation \og normale \fg{} de chaque fragment (\textit{i.e.} des opérateurs de Möller-Plesset des monomères A et B), et où V représente toujours la perturbation intermoléculaire (\textit{i.e.} l’interaction électrostatique entre A et B).\\
	
	Le premier niveau de perturbation s’obtient en ne tenant pas compte, dans un premier temps, de la corrélation intramoléculaire, soit en considérant l’interaction classique intramoléculaire. Dans ce cas, et en introduisant un terme noté $\zeta$ visant à moduler la perturbation (due à la partie intermoléculaire), $H = H_0 + \zeta.V$, l’équation de Schr\"{o}dinger s’écrit :
	
	\begin{equation}
	(H_{0} + \xi V) \psi (\xi) = (E_{int} + E_{0}) \psi (\xi)
	\end{equation}
	
	ou encore :
	
	\begin{equation}
	(H_{0} - E_{0}) \psi (\xi) = (E_{int} - \xi V) \psi (\xi)
	\end{equation}
	
	La théorie des perturbations de Rayleigh-Schr\"{o}dinger nous fournit une série de corrections pour l’énergie \cite{chipman1973perturbation}, de formes identiques à celles déjà fournies pour le seul ordre 2 dans le paragraphe précédent (voir formules \ref{2.22} et \ref{2.42}). Par souci de simplification, nous noterons ces termes :
	
	\begin{equation}
	E_{pol}^{(n)} = \langle \phi_{0}|V| \phi_{pol}^{(n)} \rangle
	\end{equation}
	
	\noindent dans lesquels $n$ représente l’ordre de la première perturbation. Le terme \og $pol$ \fg{} rappelle que l'on se place dans l'approximation de polarisation, sans effet lié à l’échange d’électron entre les monomères, violant ainsi le principe d’antisymétrie. Les fonctions issues d’un tel développement ne sont ni orthogonalisées, ni relaxées.\\
	
	Il est malheureusement courant de constater que la série perturbationnelle ne converge pas (il arrive même qu'elle diverge), ce même en réalisant un développement aux ordres supérieurs à 2. Pour assurer la convergence de la fonction et de l’énergie, on emploie un terme d’atténuation asymptotique : 
	
	\begin{equation}
	E_{int} = \sum_{n=1}^{N} E_{pol}^{(n)} + O(1/R^{2 (N+1)})
	\end{equation}
	
	Afin de rétablir le principe d’antisymétrie, un opérateur d’antisymétrisation classique est initialement appliqué sur la fonction d’onde $\mathscr{A} \phi_{pol}^{(n)}$, permettant ainsi d’établir l’expression de l’énergie SRS (Symmetrized Rayleigh-Schr\"{o}dinger), d’ou découle le nom de la théorie SAPT, corrigée des différents ordres $n$ de perturbation :
	
	\begin{equation}
	E_{RS}^{(n)} = \frac{1}{\langle\phi_{0}|\mathscr{A}\phi_{0}\rangle} \left[\langle\phi_{0} | V| \mathscr{A} \phi_{pol}^{(n-1)}\rangle - \sum_{\kappa=1}^{n-1} E_{RS}^{\kappa} \langle \phi_{0}|\mathscr{A} \phi_{pol}^{(n-\kappa)}\rangle\right]
	\end{equation}
	
	soit :
	
	\begin{equation}
	E_{SRS}^{(1)} = E_{pol}^{(1)} + E_{exch}^{(1)}
	\end{equation}
	
	\begin{equation}
	E_{SRS}^{(2)} = E_{pol}^{(2)} + E_{exch}^{(2)}
	\end{equation}
	
	jusqu’à un ordre de perturbation $n$ déterminé et suffisamment poussé pour contenir toute l’information à longue portée (rappelons que $pol$ = $disp$ (variant à minima en $R^{-6}$) + $ind$).\\
	
	
	Suivant la méthodologie SAPT, le second niveau de perturbation s’obtient en tenant compte de la corrélation intramoléculaire (à présent qu'a été incluse la perturbation intermoléculaire). Ainsi, sur chaque ordre de perturbation intermoléculaire $k$, on peut ajouter une série de perturbations intramoléculaires propres au monomère A (d’ordre $m$ représentatif pour développer l’opérateur $W^{A}$) et au monomère B (d’ordre $p$ représentatif pour développer l’opérateur $W^{B}$). La somme sur tous les ordres, $m$ et $p$, conduit aux formules correctives suivantes :
	
	\begin{equation}
	E_{pol}^{(k)} = \sum_{m,p} E_{pol}^{(kmp)}
	\end{equation}
	
	\begin{equation}
	E_{exch}^{(k)} = \sum_{m,p} E_{exch}^{(kmp)}
	\end{equation}
	
	soit une énergie d’interaction corrigée globale :
	
	\begin{equation}
	E_{int} = \sum_{k=1}^{N} ( \sum_{m,p} E_{pol}^{(kmp)} + \sum_{m,p} E_{exch}^{(kmp)} ) + O(1/R^{2 (N+1)})
	\end{equation}
	
	\noindent dans laquelle $E_{pol}^{(kmp)}$ représente l'énergie de polarisation, identique aux corrections obtenues \textit{via} la théorie des perturbations de Rayleigh-Schr\"{o}dinger. Les corrections d'échange $E_{exch}^{(kmp)}$ proviennent de l'utilisation de l'opérateur d'antisymétrisation permettant l'échange des électrons entre les monomères lorsqu'ils sont suffisamment proches les uns des autres. Le défaut de la théorie des perturbations de Rayleigh-Schr\"{o}dinger classique est donc corrigé. A une distance interatomique R suffisamment grande pour négliger les termes d'échange, la théorie SAPT est similaire à la théorie des perturbations de Rayleigh-Schr\"{o}dinger.\\
	
	En pratique, dans une approche SAPT, quelques termes seulement sont calculés de manière exacte, les autres termes (non tronqués dans les séries perturbationnelles) étant calculés à partir des fonctions d’ondes corrélées sur les monomères. Jeziorski et coll \cite{patkowski2004unified,jeziorski1994perturbation} rapportent dans leur travail l’ensemble des termes non nuls issus des séries perturbationnelles développées et montrent que l’on peut aisément séparer la contribution Hartree-Fock à l’énergie d’interaction, des contributions calculées, au moyen d’approches post-HF nécessaires à l’estimation de l’énergie de corrélation pour chaque monomère. 
	
	\begin{equation}
	E_{int}^{HF} =  E_{pol}^{(100)} + E_{exch}^{(100)} + E_{ind}^{(200)} + E_{exch-ind}^{(200)} + \delta E_{int}^{HF}
	\end{equation}
	
	Le terme $\delta E_{int}^{HF}$ contient ainsi tous les termes d'induction et d'échange d’ordres supérieurs à deux.
	
	En s’arrêtant à l’ordre deux de perturbation intermoléculaire, il est possible de calculer explicitement la correction de la partie intermoléculaire sur l’énergie d’interaction :
	
	\begin{equation}
	\Delta E_{int}^{inter} =  E_{disp}^{(200)} + E_{exch-disp}^{(200)}
	\end{equation}
	
	Il est de même possible d’accéder à la correction de la partie intramoléculaire sur l’énergie d’interaction, permettant ainsi d’atteindre une décomposition énergétique globale. À terme pour le présent travail, il sera alors intéressant de relier cette décomposition de l'énergie aux phénomènes physiques spécifiques, tels que la nature des interactions. \\
	
	Jeziorski \textit{et al} \cite{patkowski2004unified,jeziorski1994perturbation} relèvent dans leurs travaux que le développement en série du terme d’échange n’est pas unique et a conduit à plusieurs variantes de méthodes SAPT. Ainsi, il est évident que si l’expression des termes $E_{exch}^{(kmp)}$ est propre à chacune de ces méthodes, la convergence n'est pas équivalente entre les différentes versions de la théorie SAPT. Il est bon de noter que, pour l’étude de systèmes non polaires, dominés principalement par des forces de van der Waals, les différences reportées ci-dessus sont minimes puisque ce sont alors les termes du second ordre (pratiquement identiques dans chaque approche) qui dominent la série perturbationnelle. Le choix de l’approche SAPT n’a dans ce cas pratiquement aucune incidence et il n’est pas nécessaire de pousser l’ordre de perturbation au delà du second ordre. Cette remarque n’est pas sans conséquence, lorsque l’on connaît le coût, en termes de temps de calcul, du recours aux approches SAPT pour l’étude de systèmes moléculaires. Il est à noter également que l’approche Symmetrized Rayleigh-Schr\"{o}dinger (SRS), présentée et utilisée dans ce travail, est actuellement considérée comme la méthode la plus adaptée à l’étude des systèmes à faible interaction intermoléculaire, comme les interactions $\pi-\pi$ pour lesquelles il n’existe, à notre avis, que très peu d’informations.\\
	
	
	En pratique, l’énergie d'interaction SAPT peut être calculée à différents niveaux de corrélation intramonomère. Ainsi, d’autres variantes de la méthode SAPT existent.
	Le recensement exhaustif de ses approches figure dans le travail de Jeziorski \textit{et al}. Mais nous citerons toutefois l’approche SAPT(DFT) principalement développée par 
	Misquitta \textit{et al.} \cite{misquitta2005intermolecular}, Heßelmann et Jansen \cite{hesselmann2002first}, initiée à partir les idées de Williams et Chabalowski \cite{williams2001using} et basée sur le développement de la corrélation électronique des monomères à partir d’un formalisme KS (principalement PBE0), incluant les énergies dispersives calculées au niveau TDDFT. \textit{A priori}, seuls les termes intermoléculaires restent donc à évaluer par le biais du formalisme SAPT.\\
	
	
	Enfin, et toujours de façon pratique, un calcul SAPT utilise en principe les orbitales des monomères non relaxées. Aussi, le terme de transfert de charge est partiellement absent. Pour éviter cet écueil, nous avons réalisé, dans nos approches SAPT, pour chaque situation géométrique considérée, une ré-optimisation de chaque paramètre structural (à l'exception du paramètre fixant la distance entre deux monomères). L’importance de ces termes s’est révélée pratiquement négligeable dans nos études.\\
	
	A titre d’exemple, nous reportons dans ce chapitre un calcul SAPT(DFT) réalisé sur une molécule de référence, choisie tant pour ce qu'elle présente les interactions de type $\pi-\pi$ stacking qui ont motivé notre travail, que pour l'ensemble de calculs et de mesures expérimentales dont regorge la littérature à son sujet : le dimère du benzène. \\
	
	Krause \textit{et al.} \cite{krause1991binding} ont reporté une estimation de l’énergie de liaison du dimère, obtenue à partir de la mesure des énergies de dissociation de l'ion et des potentiels d’ionisation du dimère et du monomère : $D_{0}= 1.6 \pm 0.2$ kcal mol$^{1}$. Cette valeur est toutefois éloignée de celle reportée initialement par Grover \textit{et al.} \cite{grover1987dissociation}, qui avançaient $2.4 \pm 0.4$ kcal mol$^{-1}$. 
	L’ensemble des études expérimentales (et théoriques) disponibles traitent toutes de trois configurations, comme reportées dans la figure ci-dessous \ref{figprot} .
	
	\begin{figure}[H]
		\centering
		\includegraphics[scale=0.8]{image/Prot} 
		\caption[Structures du dimère de Benzène]{Structures possibles des dimères de benzène : (S) Sandwich, (T) En forme de T et (PD) Parallèle Déplacé} \label{figprot}
	\end{figure}
	
	L’analyse de ces travaux montre que le débat visant à déterminer la conformation la plus stable est loin d’être clos. La majorité des travaux annoncent les formes T et PD comme les formes les plus stables. Toutefois, intuitivement, il n'est pas illogique de penser que la configuration S, dite \og en sandwich \fg, qui correspond à une superposition maximale des deux monomères et de fait à une maximisation des interactions dispersives, puisse apparaitre comme stable. La configuration parallèle déplacée est, par exemple, souvent observée dans les expériences réalisées à l’état cristallin sur des composés purement aromatiques \cite{hunter1991pi,fyfe1997synthetic,rebek1996assembly} ou dans des études visant à caractériser les interactions des chaînes latérales aromatiques de protéines \cite{hunter1991pi,burley1985aromatic}. Klemperer \textit{et al.} \cite{janda1975benzene} ont, quant à eux, rapporté que la configuration en T était prédominante à l’état gazeux. D’autres travaux, par Arunan et Gutowsky \cite{arunan1993rotational}, issus de l'analyse des spectres rotationnels et en accord avec les études Raman de Henson \textit{et al.} \cite{henson1992raman}, rapportent que les structures favorables du dimère de benzène s’approchent de la conformation T sans pour autant la confirmer.\\ 
	
		\begin{table}[H]
			\caption{Energies d'interaction (en kcal/mol) du dimère de benzène calculées au niveau CCSD(T) et DFT-SAPT (PBE0)} \label{table-benprot}
			\begin{center}
				\begin{tabular}{l c r r r c r r c r r r}
					\toprule
					& & & \multicolumn{2}{p{2cm}}{\centering S}  &	& \multicolumn{2}{p{2cm}}{\centering
						T}& &\multicolumn{3}{p{3cm}}{\centering PD}\\
					\cline{4-12}
					& & & & R & &  &  R & & & R$_{1}$ & R$_{2}$ \\
					\midrule
					Park et Lee$^{1}$ & & & &  & &-2,67& 5,0 & &-3,03 & 3,5 & 1,8\\
					Tsuzuki et al$^{2}$ & & & -1,48& 3,8 &  &-2,46& 5,0&  & -2,48 & 3,5& 1,8\\
					Sinnokrot et al$^{3}$ & & & -1,81 & 3.8 & &-2,74& 5,0&  & -2,78 & 3,4 & 1,6\\
					Hobza et al$^{4}$ & & &-1,12 & 4.1 &  &-2,17& 5,1& & -2,01 & 3,6 & 1.8\\
					Sherrill et al$^{5}$& &  & -1,65 & 3,9 & & -2,69& 5,0 & & -2,67 & 3,5 & 1,7 \\
					Rocca$^{6}$ & & & -1,06& 3,9& & -2,15& 5,0 & & -1,82 & 3,4 & 1,8\\ 
					DFT-SAPT (PBE0)$^{7}$ & & & -1,47 & 3.9 &  &-2,44 &5,0& & -2,35& 3,5 &1,7\\
					\bottomrule
				\end{tabular}
			\end{center}
			\centering
			\footnotemark[1]{ref \cite{park2006accurate}}, CCSD(T)/CBS,
			\footnotemark[2]{ref \cite{tsuzuki2002origin}}, CCSD(T)/CBS,
			\footnotemark[3]{ref \cite{sinnokrot2002estimates}}, CCSD(T)/CBS,
			\footnotemark[4]{ref \cite{hobza1996potential}}, CCSD(T)/aug-cc-pVDZ
			\footnotemark[5]{ref \cite{sherrill2009assessment}}, CCSD(T)/CBS,
			\footnotemark[6]{ref \cite{rocca2014random}}, RPA, 
			\footnotemark[7]{SAPT(PBE0)/aug-cc-pVTZ (notre travail)}
			
			\label{benzene}
		\end{table}
		
	Il existe dans la littérature une très grande variété d’approches théoriques concernant l’étude des dimères du benzène. Nous faisons le choix de ne reporter, dans le tableau ci-dessous, que les modélisations conduisant aux résultats faisant référence, jugés \textit{a priori} comme les plus précis du fait des méthodes employées. Les travaux de Park et Lee \cite{park2006accurate}, de Tsuzuki \textit{et al.} \cite{tsuzuki2002origin} et de Sinnokrot \textit{et al.} \cite{hobza1996potential} ont tous été obtenus à partir de calculs CCSD(T) avec extrapolation CBS de base (en anglais \og Complete Basis Set extrapolation \fg). Cette méthode est une technique de calcul empirique basée sur un minimum de trois calculs séparés utilisant des bases de plus en plus complètes, telles que les bases de type cc-pVXZ (X=T, Q, 5 …). Ceci revient à obtenir une estimation approchée du résultat que l'on obtiendrait en utilisant une base de dimensions infinies. Elle permet donc d’atteindre la limite de précision de la méthode de calcul la plus performante. Nous avons reporté dans le tableau \ref{table-benprot} les résultats que nous avons obtenu, entre autre, sur la forme PD, cette forme nous intéressant \textit{a priori} plus spécifiquement puisqu'elle a pu être caractérisée expérimentalement à l'état solide (état dans lequel seront les structures à modéliser dans le présent travail, après extraction des données IR obtenues en photo acoustique). On peut remarquer, en accord avec toutes les données actuellement disponibles par le biais des approches SAPT, que notre valeur d’énergie d’interaction calculée au niveau SAPT(PBE0)/aug-cc-pVTZ est conforme à la valeur attendue. Sans surprise, l’approche SAPT permet donc de reproduire avec une excellente vélocité les énergies d’interaction de systèmes en interaction $\pi-\pi$ et d’atteindre la précision nécessaire pour reproduire des variations aussi faibles que 1 à 2 kcal/mol.
	

	
	
	
	
	
	
	\newpage
	%%%%%%%%%%%%%%%%%%%%%%%%%%%%%%%%%%%%%%%%%%%%%%
	%%%%%%%%%%%%%%%%%%%%%%%%%%%%%%%%%%%%%%%%%%%%%%
	%%%%%%%%%%%%%%%%%%%%%%%%%%%%%%%%%%%%%%%%%%%%%%
	%%%%%%%%%%%%%%%%%%%%%%%%%%%%%%%%%%%%%%%%%%%%%%
	\section{De la nécessité de la DFT}
	%%%%%%%%%%%%%%%%%%%%%%%%%%%%%%%%%%%%%%%%%%%%%%
	%%%%%%%%%%%%%%%%%%%%%%%%%%%%%%%%%%%%%%%%%%%%%%
	%%%%%%%%%%%%%%%%%%%%%%%%%%%%%%%%%%%%%%%%%%%%%%
	%%%%%%%%%%%%%%%%%%%%%%%%%%%%%%%%%%%%%%%%%%%%%%
	
	Dans ce chapitre, nous rappellerons dans un premier temps les fondements de la théorie de la fonctionnelle de la densité, abrégée en DFT, par le biais de l'évolution des différents modèles qui ont été proposés. Nous verrons ensuite comment les théorèmes de Hohenberg et Kohn prouvent que la seule connaissance de la densité électronique permet de résoudre l'équation de Schr\"{o}dinger dans le cadre de la DFT. La fonctionnelle universelle $F_{HK}[\rho]$, qui permettrait une résolution exacte du problème, restant inconnue, nous aborderons dans une troisième sous-partie l'approche KS qui contourne ce problème et légitime certaines approximations. Ces dernières donnant naissance à différents types de fonctionnelles. Après une présentation succincte, nous introduirons la notion de fonctionnelle hybride puis détaillerons le principe des approches qui ajoutent \textit{ad hoc} à un calcul KS usuel des fonctions semi-empiriques rendant compte des effets de dispersion.
	
	%%%%%%%%%%%%%%%%%%%%%%%%%%%%%%%%%%%%%%%%%%%%%%
	%%%%%%%%%%%%%%%%%%%%%%%%%%%%%%%%%%%%%%%%%%%%%%
	%%%%%%%%%%%%%%%%%%%%%%%%%%%%%%%%%%%%%%%%%%%%%%
	\subsection{Les fondements de la DFT}
	%%%%%%%%%%%%%%%%%%%%%%%%%%%%%%%%%%%%%%%%%%%%%%
	%%%%%%%%%%%%%%%%%%%%%%%%%%%%%%%%%%%%%%%%%%%%%%
	%%%%%%%%%%%%%%%%%%%%%%%%%%%%%%%%%%%%%%%%%%%%%%
	
	Contrairement aux méthodes Hartree-Fock (HF), et \textit{a fortiori} post-HF, qui décrivent le système électronique par une fonction d'onde $\Psi_{(\vec{r})}$, la théorie de la fonctionnelle de la densité le décrit par la densité électronique, notée $\rho_{(\vec{r})}$, qui est liée à la fonction d'onde $\Psi_{(\vec{r})}$ par la relation suivante~:
	
	\begin{align}
	\rho_{(\vec{r})} &= \int \Psi_{(\vec{r})}^{*} \Psi_{(\vec{r})} \\
	&= \int |\Psi_{(\vec{r})}^{2}| \notag
	\end{align}
	
	\begin{flushleft}
		\begin{tabular}{@{}lrp{10cm}}
			avec & $\vec{r}$ : & ensemble des coordonnées électroniques. 
		\end{tabular}
	\end{flushleft}
	
	
	L'énergie de l'état fondamental est ainsi une fonctionnelle de la densité électronique, c'est-à-dire que $E_{0} = E_{(\rho)}$.
	
	%%%%%%%%%%%%%%%%%%%%%%%%%%%%%%%%%%%%%%%%%%%%%%
	%%%%%%%%%%%%%%%%%%%%%%%%%%%%%%%%%%%%%%%%%%%%%%
	\subsubsection{Modèle de \textsc{Thomas-Fermi}}
	%%%%%%%%%%%%%%%%%%%%%%%%%%%%%%%%%%%%%%%%%%%%%%
	%%%%%%%%%%%%%%%%%%%%%%%%%%%%%%%%%%%%%%%%%%%%%%
	
	Le terme d'énergie cinétique a été exprimé comme une fonctionnelle de la densité pour la première fois en 1927 par Thomas et Fermi:
	
	\begin{equation}
	\hat{T}_{TF}[\rho] = \frac{3}{10} (3\pi)^{2/3} \int \rho_{(\vec{r})}^{5/3} .d\vec{r}
	\label{ener_cin_thom_ferm}
	\end{equation}
	
	Cette fonctionnelle est alors combinée aux expressions classiques des interactions électron-noyau et électron-électron, exprimées elles aussi en fonction de la densité électronique :
	
	\begin{equation}
	E_{TH}[\rho] = T_{TH}[\rho] + V_{Ne}[\rho] + V_{ee}[\rho]
	\end{equation}
	
	%%%%%%%%%%%%%%%%%%%%%%%%%%%%%%%%%%%%%%%%%%%%%%
	%%%%%%%%%%%%%%%%%%%%%%%%%%%%%%%%%%%%%%%%%%%%%%
	\subsubsection{Modèle de \textsc{Thomas-Fermi-Dirac}}
	%%%%%%%%%%%%%%%%%%%%%%%%%%%%%%%%%%%%%%%%%%%%%%
	%%%%%%%%%%%%%%%%%%%%%%%%%%%%%%%%%%%%%%%%%%%%%%
	
	Le terme d'échange, résultant du principe d'exclusion de Pauli, a été ajouté par Dirac en 1930 afin d'affiner le modèle :
	
	\begin{align}
	K[\rho] = E_{x}[\rho] &= \int \rho_{\vec{r}} \epsilon_{x}[\rho] .d\vec{r} \\
	&= -\frac{3}{4} \left(\frac{3}{\pi}\right)^{1/3} \int \rho_{(\vec{r})}^{4/3} .d\vec{r} \notag
	\end{align}
	
	\begin{flushleft}
		\begin{tabular}{@{}lrp{10cm}}
			avec & $\epsilon_{X}[\rho]$ : & énergie d'échange par électron. 
		\end{tabular}
	\end{flushleft}
	
	Le modèle de Thomas-Fermi-Dirac est défini par la combinaison de cette expression avec l'équation~\ref{ener_cin_thom_ferm} et le potentiel d'interaction électrons-noyaux $V_{Ne}[\rho]$. 
	
	
	%%%%%%%%%%%%%%%%%%%%%%%%%%%%%%%%%%%%%%%%%%%%%%
	%%%%%%%%%%%%%%%%%%%%%%%%%%%%%%%%%%%%%%%%%%%%%%
	\subsection{Modèle de \textsc{Slater}}
	%%%%%%%%%%%%%%%%%%%%%%%%%%%%%%%%%%%%%%%%%%%%%%
	%%%%%%%%%%%%%%%%%%%%%%%%%%%%%%%%%%%%%%%%%%%%%%
	
	Partant d'une approche basée sur la méthode HF, Slater propose en 1951 de substituer le terme d'énergie d'échange par une fonctionnelle de la densité issue de l'énergie d'échange de Dirac. Ce terme d'échange dans le formalisme HF peut être généralisé en introduisant le paramètre $\alpha$ :
	
	\begin{equation}
	E_{x}[\rho] = - \frac{9\alpha}{8} \left(\frac{3}{\pi}\right)^{1/3} \int \rho_{(\vec{r})}^{4/3} .d\vec{r}
	\end{equation}
	
	Des analyses empiriques basées sur différents types de systèmes chimiques ont conduit à une valeur de $3/4$ pour $\alpha$, offrant une meilleure précision que la valeur originelle de l'expression de Dirac ($2/3$).
	
	%%%%%%%%%%%%%%%%%%%%%%%%%%%%%%%%%%%%%%%%%%%%%%
	%%%%%%%%%%%%%%%%%%%%%%%%%%%%%%%%%%%%%%%%%%%%%%
	%%%%%%%%%%%%%%%%%%%%%%%%%%%%%%%%%%%%%%%%%%%%%%
	\subsection{Les théorèmes de \textsc{Hohenberg} et \textsc{Kohn}}
	%%%%%%%%%%%%%%%%%%%%%%%%%%%%%%%%%%%%%%%%%%%%%%
	%%%%%%%%%%%%%%%%%%%%%%%%%%%%%%%%%%%%%%%%%%%%%%
	%%%%%%%%%%%%%%%%%%%%%%%%%%%%%%%%%%%%%%%%%%%%%%
	
	Tous ces modèles, bien qu'ils constituent les fondements de la DFT, ne démontrent pas formellement que seule la connaissance de la densité est importante pour accéder à l'énergie totale d'un système. C'est ainsi que Hohenberg et Kohn eurent l'idée, en 1965, de démontrer, par le biais de deux théorèmes, que l'équation de Schr\"{o}dinger pouvait être résolue de façon exacte dans le cadre de l'approximation de Born-Oppenheimer, uniquement grâce à la densité électronique.
	
	%%%%%%%%%%%%%%%%%%%%%%%%%%%%%%%%%%%%%%%%%%%%%%
	%%%%%%%%%%%%%%%%%%%%%%%%%%%%%%%%%%%%%%%%%%%%%%
	\subsubsection{Premier théorème : preuve d'existence}
	%%%%%%%%%%%%%%%%%%%%%%%%%%%%%%%%%%%%%%%%%%%%%%
	%%%%%%%%%%%%%%%%%%%%%%%%%%%%%%%%%%%%%%%%%%%%%%
	
	Ce premier théorème énonce que l'ensemble des propriétés du système, notamment l'énergie, peuvent être calculées à partir de la seule densité électronique de l'état fondamental. Elles peuvent donc être décrites comme une fonctionnelle de la densité électronique, et l'énergie totale s'écrit :
	
	\begin{equation}
	E[\rho] = F_{HK}[\rho] + \int \rho_{(\vec{r})} \nu_{ext} .d\vec{r}
	\label{Hohen_Kohn}
	\end{equation}
	\noindent où :
	\begin{align}
	F_{HK}[\rho] &= T_{e}[\rho] + V_{ee}[\rho] \\
	\nu_{ext} &= V_{Ne}[\rho] \notag
	\end{align}
	
	Notons que la fonctionnelle universelle $F_{HK}[\rho]$, qui regroupe les termes d'énergie cinétique des électrons et celui d'énergie potentielle d'interaction électron-électron, n'est pas liée au potentiel externe $\nu_{ext}$. L'énergie de l'état fondamental est \textit{a priori} accessible de manière exacte car cette fonctionnelle ne repose sur aucune approximation.
	
	%%%%%%%%%%%%%%%%%%%%%%%%%%%%%%%%%%%%%%%%%%%%%%
	%%%%%%%%%%%%%%%%%%%%%%%%%%%%%%%%%%%%%%%%%%%%%%
	\subsubsection{Second théorème : théorème variationnel}
	%%%%%%%%%%%%%%%%%%%%%%%%%%%%%%%%%%%%%%%%%%%%%%
	%%%%%%%%%%%%%%%%%%%%%%%%%%%%%%%%%%%%%%%%%%%%%%
	
	Basé sur l'équation~\ref{Hohen_Kohn}, Hohenberg et Kohn ont ensuite établi un principe variationnel pour déterminer la densité électronique de l'état fondamental :
	
	\begin{equation}
	E[\rho] \geq E[\rho_{0}]
	\end{equation}
	
	\begin{flushleft}
		\begin{tabular}{@{}lrp{10cm}}
			avec & $\rho_{0}$ : & densité électronique de l'état fondamental, \\
			& $\rho$ : & densité électronique quelconque.
		\end{tabular}
	\end{flushleft}
	
	Dans cette équation, à une densité d'essai $\rho$ correspond une seule énergie potentielle $\int \rho_{(\vec{r})} \nu_{ext} .d\vec{r}$ et une seule fonction d'onde $\Psi_{\rho}$. La méthode de double minimisation, \textit{i.e.} sous contrainte de Levy, permet de différencier la fonction d'onde $\Psi_{\rho_{0}}$, correspondant à l'état fondamental, parmi le jeu infini des fonctions d'ondes $\Psi_{\rho}$ donnant la même densité. Ainsi, nous pouvons déterminer, parmi toutes les densités, celle qui minimisera l'énergie par la relation suivante :
	
	\begin{equation}
	E[\rho_{0}] = \min\limits_{\rho}\, (\min\limits_{\Psi\rightarrow\rho}\, (F[\rho] + \int \rho_{\vec{r}} \nu_{\vec{r}}\, .d\vec{r}\, ))
	\end{equation}
	
	Si les théorèmes de Hohenberg et Kohn démontrent une correspondance unique entre une densité $\rho_{\vec{r}}$ et la fonction d'onde $\Psi$ du système, la fonctionnelle universelle $F_{HK}[\rho]$ reste cependant inconnue.
	
	%%%%%%%%%%%%%%%%%%%%%%%%%%%%%%%%%%%%%%%%%%%%%%
	%%%%%%%%%%%%%%%%%%%%%%%%%%%%%%%%%%%%%%%%%%%%%%
	%%%%%%%%%%%%%%%%%%%%%%%%%%%%%%%%%%%%%%%%%%%%%%
	\subsection{Approche Kohn-Sham}\label{Kohn-Sham}
	%%%%%%%%%%%%%%%%%%%%%%%%%%%%%%%%%%%%%%%%%%%%%%
	%%%%%%%%%%%%%%%%%%%%%%%%%%%%%%%%%%%%%%%%%%%%%%
	%%%%%%%%%%%%%%%%%%%%%%%%%%%%%%%%%%%%%%%%%%%%%%
	
	Afin de contourner ce problème, Kohn et Sham substituèrent au Hamiltonien réel, décrivant un système de $n$ particules en interaction, un Hamiltonien de référence décrivant un système de $n$ particules sans interaction mais ayant la même densité que le système réel. Le problème est ainsi réduit à la résolution de $n$ équations monoélectronique couplées, analogues aux équations HF. L'opérateur monoélectronique de Kohn-Sham $\hat{K}_{KS}$ s'exprime ainsi :
	
	\begin{equation}
	\hat{H}_{KS} = -\frac{1}{2} \nabla^{2} + \nu_{H}[\rho] + \nu_{xc}[\rho] + \nu_{ext}[\rho]
	\end{equation}
	
	\noindent où :
	\begin{align}
	\nu_{H}[\rho] &= \int \frac{\rho_{(\vec{r})} - \rho_{(\vec{r}')}}{|\vec{r} - \vec{r}'|} .d\vec{r}' \\
	\nu_{xc}[\rho] &= \frac{\partial E_{xc}[\rho_{(\vec{r})}]}{\partial\rho_{(\vec{r})}}
	\end{align}
	
	\noindent $\nu_{H}[\rho]$ et $\nu_{xc}[\rho]$ étant respectivement le potentiel de Hartree et le potentiel d'échange et de corrélation, dans lequel $E_{xc}[\rho_{(\vec{r})}]$ est l'énergie d'échange et de corrélation.
	
	En définissant un potentiel fictif $\nu_{eff(\vec{r})}$ pouvant être appliqué à des systèmes sans interaction de densité $\rho$ :
	
	\begin{equation}
	\nu_{eff(\vec{r})} = \nu_{H}[\rho] + \nu_{xc}[\rho] + \nu_{ext}[\rho]
	\end{equation}
	
	\noindent nous introduisons un jeu d'orbitales $\psi_{(\vec{r})}$, appelées orbitales de Kohn-Sham, et nous obtenons un jeu d'équations aux valeurs propres :
	
	\begin{equation}
	\hat{H}_{KS} \psi_{i(\vec{r})} = \epsilon_{i} \psi_{i(\vec{r})}
	\end{equation}
	
	Comme dans le cas de la méthode HF, l'énergie du système peut être minimisée en résolvant ce jeu d'équations de façon auto-cohérente grâce à l'utilisation des orbitales de Kohn-Sham. L'énergie du système est alors donnée par :
	
	\begin{equation}
	E_{KS}^{tot}[\rho] = T_{s}[\rho] + J[\rho] + E_{xc}[\rho] + \int V_{ext(\vec{r})}\rho_{(\vec{r})} .d\vec{r}
	\end{equation}
	
	\begin{flushleft}
		\begin{tabular}{@{}lrp{10cm}}
			avec & $T_{s}[\rho]$ : & énergie cinétique des électrons sans interaction, \\
			& $J[\rho]$ : & énergie d'interaction coulombienne entre les électrons, \\
			& $E_{xc}[\rho]$ : & énergie d'échange et de corrélation, \\
			& $\int V_{ext(\vec{r})}\rho_{(\vec{r})} .d\vec{r}$ : & énergie d'interaction avec le potentiel externe. 
		\end{tabular}
	\end{flushleft}
	
	D'après les théorèmes de Hohenberg et Kohn, $E_{KS}^{tot}[\rho]$ doit être égale à l'énergie totale du système réel $E_{reel}^{tot}[\rho]$, qui peut être décrite comme suit :
	
	\begin{equation}
	E_{reel}^{tot}[\rho] = T[\rho] + V_{ee}[\rho] + \int V_{ext(\vec{r})}\rho_{(\vec{r})} .d\vec{r}
	\end{equation}
	
	Le terme d'échange-corrélation peut ainsi être explicité comme la somme de la correction à l'énergie cinétique due à l'interaction entre électrons ($T[\rho] - T_{s}[\rho]$) et des corrections non classiques à la répulsion électron-électron ($V_{ee}[\rho] - J[\rho]$) :
	
	\begin{equation}
	E_{xc}[\rho] = T[\rho] - T_{s}[\rho] + V_{ee}[\rho] - J[\rho]
	\end{equation}
	
	La théorie de la fonctionnelle de la densité de Kohn-Sham (KS) a connu un grand succès parmi les méthodes de calcul appliquées aux systèmes de grande dimension, du fait de son ratio coût calculatoire/performance très intéressant. C'est pour cet avantage indéniable qu'elle sert de base à de nombreuses évolutions de la DFT.
	
	%%%%%%%%%%%%%%%%%%%%%%%%%%%%%%%%%%%%%%%%%%%%%%
	%%%%%%%%%%%%%%%%%%%%%%%%%%%%%%%%%%%%%%%%%%%%%%
	%%%%%%%%%%%%%%%%%%%%%%%%%%%%%%%%%%%%%%%%%%%%%%
	\subsection{Les différentes classes de fonctionnelles}
	%%%%%%%%%%%%%%%%%%%%%%%%%%%%%%%%%%%%%%%%%%%%%%
	%%%%%%%%%%%%%%%%%%%%%%%%%%%%%%%%%%%%%%%%%%%%%%
	%%%%%%%%%%%%%%%%%%%%%%%%%%%%%%%%%%%%%%%%%%%%%%
	
	Formellement, la DFT est donc une méthode exacte, dans la limite de la connaissance de la fonctionnelle universelle $F_{HK}[\rho]$ ou de sa fonctionnelle exacte d’échange et de corrélation $F_{xc}[\rho]$. Malheureusement, la forme exacte de l’énergie d’échange et de corrélation est inconnue, si bien qu’il est nécessaire de faire des approximations. Dans la pratique, l’énergie d’échange et de corrélation $E_{xc}[\rho]$ est calculée à l’aide de fonctionnelles d’échange et de corrélation définies comme suit :
	
	\begin{equation}
	\int F_{xc}[\rho_{(\vec{r})}].d\vec{r} = E_{xc}[\rho_{(\vec{r})}]
	\end{equation}
	
	L’énergie d’échange et de corrélation est généralement séparée en deux termes distincts, l’un d’échange -- $E_{x}[\rho]$ -- et l’autre de corrélation -- $E_{c}[\rho]$ :
	
	\begin{equation}
	E_{xc}[\rho_{(\vec{r})}] = E_{x}[\rho_{(\vec{r})}] + E_{c}[\rho_{(\vec{r})}]
	\end{equation}
	
	Plusieurs fonctionnelles ont donc été développées pour traiter chacune de ces contributions, de façon simultanée ou indépendante. Nous allons ici donner un bref aperçu -- non exhaustif -- des différentes familles de fonctionnelles, en suivant le critère de classification élaboré par Perdew et couramment appelé \og échelle de Perdew \fg{}. Il a proposé de classer les fonctionnelles en fonction du degré d’informations non locales contenues dans leur forme analytique. Au premier échelon se trouvent les fonctionnelles qui dépendent uniquement de la densité électronique, dites fonctionnelles LDA (\og Local Density Approximation \fg{}). Viennent ensuite les fonctionnelles corrigées par gradient, dites GGA (\og Generalized Gradient Approximation \fg{}), dans lesquelles la non-localité est introduite grâce à leur dépendance vis-à-vis du gradient de la densité. A ce même niveau se trouvent également les fonctionnelles de type méta-GGA, dépendant aussi de l’énergie cinétique (calculée à partir des orbitales moléculaires remplies). Il faut noter que, d’un point de vue mathématique, toutes ces familles de fonctionnelles sont strictement locales. Pour parvenir à des fonctionnelles véritablement non locales, il faut encore monter d’un cran dans l’échelle de Perdew, jusqu’aux fonctionnelles dites hybrides, où la présence d’un pourcentage (variable) d’échange HF calculé en utilisant les orbitales KS, permet d’introduire un véritable terme non local. Plus récemment une autre famille de fonctionnelles hybrides a été développée (dite à longue portée ou encore à séparation de portée) dans laquelle le pourcentage d’échange calculé de façon HF n’est pas constant mais dépend de la distance inter-électronique. Enfin, des fonctionnelles non locales montrant une dépendance explicite des orbitales KS occupées et vacantes représentent le dernier niveau dans l’échelle de Perdew.
	
	%%%%%%%%%%%%%%%%%%%%%%%%%%%%%%%%%%%%%%%%%%%%%%
	%%%%%%%%%%%%%%%%%%%%%%%%%%%%%%%%%%%%%%%%%%%%%%
	\subsubsection{Local Density approximation (LDA)}\label{lda}
	%%%%%%%%%%%%%%%%%%%%%%%%%%%%%%%%%%%%%%%%%%%%%%
	%%%%%%%%%%%%%%%%%%%%%%%%%%%%%%%%%%%%%%%%%%%%%%
	
	L’approximation locale de la densité est l’approximation la plus grossière, dans laquelle l’énergie d’échange et de corrélation $E_{xc}$ n’est fonction que de la seule densité électronique :
	
	\begin{equation}
	E_{xc}^{LDA}[\rho_{(\vec{r})}] = \int \rho_{(\vec{r})} \epsilon_{xc}[\rho_{(\vec{r})}].d\vec{r}
	\end{equation}
	
	\noindent où la valeur de $\epsilon_{xc}$ à une position $\vec{r}$ est calculée exclusivement à partir de la valeur de la densité électronique $\rho$ à cette position. En pratique, $\epsilon_{xc}$ décrit l’énergie d’échange et de corrélation par particule pour un gaz uniforme d’électrons de densité $\rho$. Le potentiel d’échange et de corrélation correspondant est alors :
	
	\begin{equation}
	\nu_{xc(\vec{r})}^{LDA} = \epsilon_{xc}[\rho_{(\vec{r})}] + \rho_{(\vec{r})} \frac{\partial \epsilon_{xc}[\rho_{(\vec{r})}]}{\partial \rho}
	\end{equation}
	
	Malgré le fait que les résultats obtenus soient généralement en bon accord avec les résultats expérimentaux, notamment au niveau de la géométrie et de la structure électronique, cette approximation reste une approximation locale, dans laquelle il n'est pas tenu compte de l’inhomogénéité de la densité électronique.
	
	%%%%%%%%%%%%%%%%%%%%%%%%%%%%%%%%%%%%%%%%%%%%%%
	%%%%%%%%%%%%%%%%%%%%%%%%%%%%%%%%%%%%%%%%%%%%%%
	\subsubsection{Generalized Gradient Approximation (GGA)}
	%%%%%%%%%%%%%%%%%%%%%%%%%%%%%%%%%%%%%%%%%%%%%%
	%%%%%%%%%%%%%%%%%%%%%%%%%%%%%%%%%%%%%%%%%%%%%%
	
	L’idée directrice de l’approximation du gradient généralisé est donc de mieux tenir compte de l’inhomogénéité de la densité du système, en introduisant une dépendance de la densité $\rho$ à son gradient $\nabla \rho$. L’expression générale des fonctionnelles de type GGA est la suivante :
	
	\begin{equation}
	E_{xc}^{GGA}[\rho_{(\vec{r})}] = A_{x} \int \rho_{(\vec{r})}^{4/3} E^{GGA}(s) .d\vec{r}^{3}
	\end{equation}
	
	\noindent où $s$, gradient de la densité réduite, est tel que :
	
	\begin{equation}
	s = \frac{|\nabla \rho_{(\vec{r})}|}{2 k_{F} \rho_{(\vec{r})}}
	\end{equation}
	
	\noindent avec $k_{F} = (3 \pi^{2} \rho_{(\vec{r})})^{1/3}$. Ainsi, on fait apparaître par le biais du terme $s$ un terme quasi-local, dépendant non seulement de la densité électronique mais également de son gradient au voisinage de $\vec{r}$.
	
	Un exemple de fonctionnelle GGA est celle de Perdew, Burke et Ernzherhof, notée PBE~\cite{perdew1996generalized}.
	
	%%%%%%%%%%%%%%%%%%%%%%%%%%%%%%%%%%%%%%%%%%%%%%
	%%%%%%%%%%%%%%%%%%%%%%%%%%%%%%%%%%%%%%%%%%%%%%
	\subsubsection{Fonctionnelles hybrides}
	%%%%%%%%%%%%%%%%%%%%%%%%%%%%%%%%%%%%%%%%%%%%%%
	%%%%%%%%%%%%%%%%%%%%%%%%%%%%%%%%%%%%%%%%%%%%%%
	
	La dernière grande famille de fonctionnelles est celle des fonctionnelles hybrides. L’idée consiste à introduire une fraction d’échange calculée de façon exacte (telle qu’utilisée dans la méthode HF) dans une fonctionnelle d’échange de type GGA. L’expression de $E_{xc}$ devient alors :
	
	\begin{equation}
	E_{xc}^{hybride}[\rho_{(\vec{r})}] = (1- \alpha) E_{xc}^{GGA}[\rho_{(\vec{r})}] + \alpha E_{xc}^{HF}[\rho_{(\vec{r})}]
	\end{equation}
	
	\noindent où le coefficient de la combinaison $\alpha$ donne le rapport HF/DFT. \\
	La fonctionnelle PBE0~\cite{adamo1999toward}, par exemple, présente 25\% d’échange HF~\cite{adamo1997toward} dans une fonctionnelle GGA de type PBE~\cite{perdew1996generalized} :
	
	\begin{equation}
	E_{xc}^{PBE0} = E_{xc}^{PBE} + \frac{1}{4} (E_{x}^{HF} - E_{x}^{PBE})
	\end{equation}
	
	Elle a l’avantage d’être non paramétrée (car le pourcentage d’HF inclus n’est pas empirique mais basé sur des arguments de théorie perturbationnelle), et de fournir des résultats très précis, tant au niveau du calcul des structures moléculaires et électroniques que pour le calcul de propriétés spectroscopiques \cite{adamo1999toward}.
	
	Dans cette catégorie figure également la fonctionnelle la plus utilisée pour traiter des systèmes moléculaires, à savoir la fonctionnelle B3LYP \cite{becke1993density}. Comme son nom l’indique, elle inclus trois paramètres et se base sur les fonctionnelles GGA d’échange et de corrélation de Becke (B) \cite{becke1988density} et Lee, Yang et Parr (LYP) \cite{chengteh1988development}, suivant l’expression :
	
	\begin{equation}
	E_{xc}^{B3LYP} = (1-a) E_{x}^{LSDA} + a E_{x}^{HF} + b \Delta E_{x}^{B} + (1-c) E_{c}^{LSDA} + c E_{c}^{LYP}
	\label{B3LYP}
	\end{equation}
	
	\noindent avec $a$, $b$ et $c$ fixés respectivement à 0,20 , 0,72 et 0,81.
	
	
	%%%%%%%%%%%%%%%%%%%%%%%%%%%%%%%%%%%%%%%%%%%%%%
	%%%%%%%%%%%%%%%%%%%%%%%%%%%%%%%%%%%%%%%%%%%%%%
	\subsubsection{Nécessité d'une fonctionnelle « longue portée »}
	%%%%%%%%%%%%%%%%%%%%%%%%%%%%%%%%%%%%%%%%%%%%%%
	%%%%%%%%%%%%%%%%%%%%%%%%%%%%%%%%%%%%%%%%%%%%%%
	
	Bien que la théorie de la fonctionnelle de la densité connaisse un large succès, elle conduit toujours à des écueils dès lors que l'on tente de l'employer pour l'étude de systèmes chimiques où les forces de van der Waals sont prédominantes. 
	En effet, les effets de corrélation électronique des forces de dispersion étant purement non-locaux, l'approximation locale ou non-locale qui fait le fondement de la DFT reste problématique. Se pose alors la question de savoir comment modéliser uniformément ces interactions.
	 Comme nous allons l'expliciter, les fonctionnelles hybrides dites \og à longue portée \fg{} permettent de répondre, au premier degré de l'échelle de précision, à cette problématique. 
	
	
	%%%%%%%%%%%%%%%%%%%%%%%%%%%%%%%%%%%%%%%%%%%%%%
	%%%%%%%%%%%%%%%%%%%%%%%%%%%%%%%%%%%%%%%%%%%%%%
	%%%%%%%%%%%%%%%%%%%%%%%%%%%%%%%%%%%%%%%%%%%%%%
	\subsection[LC-DFT-D hybride : $\omega$BXD]{Construction d'une LC-DFT-D hybride : cas de la $\omega$BXD}
	%%%%%%%%%%%%%%%%%%%%%%%%%%%%%%%%%%%%%%%%%%%%%%
	%%%%%%%%%%%%%%%%%%%%%%%%%%%%%%%%%%%%%%%%%%%%%%
	%%%%%%%%%%%%%%%%%%%%%%%%%%%%%%%%%%%%%%%%%%%%%%
	Les fonctionnelles hybrides avec correction à longue portée basées sur la théorie Kohn-Sham ont naturellement rencontré un grand engouement puisque la précision apportée n'accroît pas le coût calculatoire par comparaison aux méthodes DFT hybrides.
	
	%%%%%%%%%%%%%%%%%%%%%%%%%%%%%%%%%%%%%%%%%%%%%%
	%%%%%%%%%%%%%%%%%%%%%%%%%%%%%%%%%%%%%%%%%%%%%%
	\subsubsection{B88}
	%%%%%%%%%%%%%%%%%%%%%%%%%%%%%%%%%%%%%%%%%%%%%%
	%%%%%%%%%%%%%%%%%%%%%%%%%%%%%%%%%%%%%%%%%%%%%%
	
	Comme nous l'avons vu dans le cadre des approximations de la fonctionnelle de la densité \footnote{\og Density Functional Approximations \fg{} }, notées DFAs, la décroissance exponentielle du potentiel d'échange-corrélation engendre une mauvaise représentation des interactions à longue distance, qui varient quant à elles en $1/r$. Cette erreur, baptisée erreur d'auto-interaction (ou SIE, pour \og self-interaction error \fg{}), est liée au fait que ces approximations se basent sur la densité de spin locale (LSDA, pour \og local spin density approximation \fg{}). De fait, elles décrivent mal l'état fondamental qui devrait être, dans le cadre de la DFT pure, strictement sans auto-interaction.     
	C'est pourquoi, afin d'introduire un effet non-local de l'échange-corrélation dans le modèle KS-DFT (partie~\ref{Kohn-Sham}), Becke proposa en 1988 d'incorporer dans sa fonctionnelle d'échange B88~\cite{becke1988density} une fraction d'échange calculée de façon exacte Hartree-Fock. 
	
	Dans le cadre général des DFAs, l'énergie d'échange-corrélation s'écrit donc :
	
	\begin{equation}
	E_{xc} = c_{x}E_{x}^{HF} + E_{xc}^{DFA}
	\label{xcB88}
	\end{equation}
	
	\noindent où $c_{x}$ prend généralement des valeurs comprises entre 0,2 et 0,25~\cite{becke1993density} pour traduire au mieux les propriétés thermodynamiques, et entre 0,4 et 0,6~\cite{boese2004development} pour traduire au mieux les données cinétiques.
	
	Basée sur ce modèle, la désormais bien connue DFT hybride B3LYP \cite{becke1993density} (équation~\ref{B3LYP}) donne des résultats comparables à ceux obtenus à partir de la théorie perturbative M\o ller-Plesset à l'ordre 2 \cite{moller1934note} (MP2), souvent utilisée comme référence, dans le cadre de systèmes fortement liés. Depuis, de nombreuses recherches ont porté sur l'amélioration constante de ce potentiel d'échange-corrélation $E_{xc}[\rho]$.
	
	À ce stade, il me paraît essentiel de rappeler le caractère quelque peu \og erratique \fg{} des résultats de nombre de travaux menés en DFT, caractère qui tient au choix de la fonctionnelle d'échange et/ou de corrélation. 
	 Lorsque, pour un système donné, une fonctionnelle d’échange-corrélation se révèle capable de traduire des paramètres physiques avec une précision satisfaisante autour de la position d’équilibre, rien ne garantit qu’elle soit en mesure de donner la même précision pour un autre (type de) système(s). Il en va de même pour l’étude de dimères (atomiques ou moléculaires). Dans le présent travail, une attention toute particulière a donc été portée au choix de la fonctionnelle d’échange-corrélation, s'agissant d'une étude portant sur une très grande variété de systèmes, \textit{i.e.} non seulement des familles de molécules comportant un ou plusieurs hétéroatome(s) mais les homo-dimères correspondants. La fonctionnelle $\omega$b97X a retenu notre attention dans ce travail.
	
	
	%%%%%%%%%%%%%%%%%%%%%%%%%%%%%%%%%%%%%%%%%%%%%%
	%%%%%%%%%%%%%%%%%%%%%%%%%%%%%%%%%%%%%%%%%%%%%%
	\subsubsection{B97}
	%%%%%%%%%%%%%%%%%%%%%%%%%%%%%%%%%%%%%%%%%%%%%%
	%%%%%%%%%%%%%%%%%%%%%%%%%%%%%%%%%%%%%%%%%%%%%%
	
	Une avancée significative a de nouveau été faite par Becke en 1997 dans le domaine des KS-DFT. Par une méthode similaire à la combinaison linéaire d'orbitales atomiques \footnote{\og Linear Combination of Atomic Orbitals \fg{}.}, notée LCAO, Becke a proposé un modèle mathématique basé sur l'approximation de densité de spin local (LSDA), sa première dérivée et une petite fraction d'échange HF pour décrire le potentiel d'échange-corrélation $E_{xc}[\rho]$. Une optimisation systématique des coefficients linéaires à partir d'un jeu classique de données expérimentales a conduit à l'apparition de la méthode B97~\cite{becke1997density}. La base de données contient notamment des valeurs relatives à l'interaction entre systèmes conjugués.
	
	Cette méthodologie a été reprise par F. A. Hamprecht \textit{et al.}, P. J. Wilson \textit{et al.} et T. W. Keal \textit{et al.} pour respectivement conduire à la B97-1 \cite{hamprecht1998development} (1998), la B97-2 \cite{wilson2001hybrid} (2001) et la B97-3 \cite{keal2005semiempirical} (2005). Il s'agissait alors de réoptimisations des coefficients linéaires par rapport à de nouvelles bases de données expérimentales, plus complètes.
	
	Mais cette nouvelle paramétrisation empirique du terme d'échange-corrélation ne résout pas le problème de sa non-décroissance en $1/r$. La prise en compte totale du terme d'échange HF $E_{x}^{HF}$ ($c_{x}$=1 dans l'équation~\ref{xcB88}) pourrait résoudre ce problème mais serait incompatible avec le terme de corrélation DFA $E_{c}^{DFA}$, engendrant une mauvaise compensation des erreurs respectives.
	
	%%%%%%%%%%%%%%%%%%%%%%%%%%%%%%%%%%%%%%%%%%%%%%
	%%%%%%%%%%%%%%%%%%%%%%%%%%%%%%%%%%%%%%%%%%%%%%
	\subsubsection{$\omega$B97}
	%%%%%%%%%%%%%%%%%%%%%%%%%%%%%%%%%%%%%%%%%%%%%%
	%%%%%%%%%%%%%%%%%%%%%%%%%%%%%%%%%%%%%%%%%%%%%%
	
	L'idée de séparer le traitement des interactions courtes (SR, pour \og short range \fg{}) et longues portées (LR, pour \og long range \fg{}) s'est alors présentée comme le choix le plus évident, aussi bien au niveau de la compréhension des phénomènes que sur le plan mathématique. Le principe est de traiter séparément, à l'aide d'une fonction erreur $(erf)$, les interactions à courte et à longue distance. Les interactions à courte distance sont représentées par une fonctionnelle de la densité, tandis que les interactions à longue distance sont traduites par une fonction d'onde. Ce principe conduit naturellement à l'élaboration d'une fonctionnelle hybride à séparation de portée. L'introduction de la fonction erreur, avec un paramètre libre, permet de contrôler le rayon d'action des interactions à courte portée.
	
	Iikura \textit{et al.} \cite{iikura2001long} ont été parmi les premiers à proposer ce type de solutions. Leur méthode consiste à traiter la partie d'échange (LR) par la théorie HF, tandis que la partie SR est approximée par une DFA ; le terme de corrélation est quant à lui le même que celui de Coulomb, quelle que soit la distance :
	
	\begin{equation}
	E_{xc}^{LC-DFA} = E_{x}^{LR-HF} + E_{x}^{SR-DFA} + E_{c}^{DFA}
	\end{equation}
	
	Ce schéma de séparation de portée a l'avantage de conduire à des temps de calcul très proches des DFT hybrides. Reste toutefois à développer une fonctionnelle d'échange SR précise ainsi qu'une fonctionnelle de corrélation compatible avec la fonctionnelle d'échange.
	
	Le type d'opérateur de coupure le plus utilisé dans le cadre des LC-DFT hybrides est la fonction d'erreur standard, notée $(erf)$ :
	
	\begin{equation}
	\frac{1}{r} = \frac{erf(\omega r_{12})}{r_{12}} + \frac{erfc(\omega r_{12})}{r_{12}}
	\label{erf}
	\end{equation}
	
	\begin{flushleft}
		\begin{tabular}{@{}lrp{10cm}}
			avec & $\frac{erf(\omega r_{12})}{r_{12}}$ : & interaction de courte portée, \\
			& $\frac{erfc(\omega r_{12})}{r_{12}}$ : & interaction complémentaire, \\
			& $r_{12}$ : & distance entre les particules 1 et 2, \\
			& $\omega$ : & paramètre contrôlant la séparation.
		\end{tabular}
	\end{flushleft}
	
	Dans cette équation, l'introduction du paramètre $\omega$, qui s'exprime comme l'inverse d'une distance, permet de donner un sens physique à la fonction d'erreur, puisqu'il est étroitement lié à une longueur caractéristique de la séparation.
	Naturellement, il existe différents types de fonctions erreur $(erf)$, qui facilitent son intégration mathématique dans les codes de calculs. Dans le cas de la fonctionnelle $\omega$B97 \cite{chai2008long} et de ses améliorations (à savoir les fonctionnelles $\omega$B97X et $\omega$B97X-D), c'est la fonction $erf/erfc$ qui a été choisie par Jeng-Da Chai et Martin Head-Gordon dans leurs travaux. 
	Le choix des auteurs s'est porté sur un terme d'échange exact HF longue portée $E_{x}^{LR-HF}$, calculé à partir des spin-orbitales occupées $\phi_{i \sigma}(r)$, et une forme analytique du terme d'échange $E_{x}^{SR-DFA}$ obtenue par intégration du carré de la matrice densité LSDA :
	
	\begin{align}
	E_{x}^{LR-HF} &= -\frac{1}{2} \sum_{\sigma} \sum_{ij}^{occ.} \iint \phi_{i \sigma}^{*}(r_{1}) \phi_{j \sigma}^{*}(r_{1}) \frac{erf(\omega r_{12})}{r_{12}} \phi_{i \sigma}(r_{2}) \phi_{j \sigma}(r_{2}).dr_{1}.dr_{2}, \\
	E_{x}^{SR-LSDA} &= \sum_{\sigma} \int \underbrace{-\frac{3}{2}\left(\frac{3}{4\pi}\right)^{1/3}\rho_{\sigma}^{4/3} (r) F(a_{\sigma})}_{e_{x \sigma}^{SR-LSDA} (\rho_{\sigma}) .dr}.
	\end{align}
	
	\noindent où :
	\begin{align}
	k_{F \sigma}&=(6\pi^{2}\rho_{sigma}(r))^{1/3},\nonumber\\
	F(a_{\sigma})&=1-\frac{8}{3}a_{\sigma}\left[\sqrt{\pi}\: erf\left(\frac{1}{2a_{\sigma}}\right)-3a_{\sigma}+4a_{\sigma}^{3}+(2a_{\sigma}-4a_{\sigma}^{3}) \: exp\left(-\frac{1}{4a_{\sigma}^{2}}\right)\right],\nonumber\\
	a_{\sigma}&=\frac{\omega}{2k_{F\sigma}}.\nonumber
	\end{align}
	
	\begin{flushleft}
		\begin{tabular}{@{}lrp{10cm}}
			avec & $k_{F\sigma}$ : & vecteur d'onde local de Fermi,\\
			& $F(a_{\sigma})$ : & fonction d'atténuation,\\
			& $a_{\sigma}$ : & paramètre de contrôle (sans unité) de la fonction d'atténuation $F(a_{\sigma})$.
		\end{tabular}
	\end{flushleft}
	
	En retenant une fonctionnelle de corrélation basée elle aussi sur la LSDA $E_{c}^{LSDA}$, la plus simple des DFT hybrides à correction de longue portée (RSHX-LDA pour l’anglais \og Range Separated Hybrid eXchange \fg{} \cite{angyan2005van} s'écrit~:
	
	\begin{equation}
	E_{xc}^{RSHXLDA} = E_{x}^{LR-HF} + E_{x}^{SR-LSDA} + E_{c}^{LSDA}
	\end{equation}
	
	La fonctionnelle $\omega$B97\cite{chai2008long} s'écrit alors :
	
	\begin{equation}
	E_{xc}^{\omega B97} = E_{x}^{LR-HF} + E_{x}^{SR-B97} + E_{c}^{B97}
	\end{equation}
	
	Il est à noter que celle-ci ne possède pas d'échange Hartree-Fock à courte portée (SR), comme la plupart des fonctionnelles hybrides à correction de portée.
	
	En dépit de nombreuses études visant à optimiser la valeur du paramètre $\omega$, la précision calculatoire reste insuffisante du point de vue de la thermochimie. En effet, comme nous l'avons souligné ci-avant, une valeur trop élevée du paramètre $\omega$ engendrerait une incompatibilité entre le terme d'échange non-local, $E_{x}^{LR-HF}$, et le terme local de corrélation, $E_{c}^{LSDA}$. De plus, d'après l'équation~\ref{erf}, plus $\omega$ est petit, plus la contribution du terme d'échange (SR), $E_{x}^{SR-LSDA}$, sera importante. Attribuer à $\omega$ une valeur trop faible reviendrait alors à traiter le problème dans un cadre très proche de la LDA classique qui, comme nous l'avons vu dans la partie~\ref{lda}, est incapable de traduire correctement le terme d'échange à courte portée.
	
	%%%%%%%%%%%%%%%%%%%%%%%%%%%%%%%%%%%%%%%%%%%%%%
	%%%%%%%%%%%%%%%%%%%%%%%%%%%%%%%%%%%%%%%%%%%%%%
	\subsubsection{$\omega$B97X}
	%%%%%%%%%%%%%%%%%%%%%%%%%%%%%%%%%%%%%%%%%%%%%%
	%%%%%%%%%%%%%%%%%%%%%%%%%%%%%%%%%%%%%%%%%%%%%%
	
	Afin de remédier à ce problème, une partie d'échange (SR) HF, notée $E_{x}^{SR-HF}$, est ajoutée à $E_{x}^{SR-LSDA}$ dans une proportion d'environ 16\%, de la même manière que Becke dans la fonctionnelle B88. Ceci présente l'avantage de ne pas perturber la partie LR, qui est quant à elle correcte. La nouvelle fonctionnelle comporte donc désormais un paramètre, noté $c_{x}$, contrôlant la proportion d'échange exact HF à courte distance :
	
	\begin{equation}
	E_{xc}^{LC-DFA} = E_{x}^{LR-HF} + c_{x}E_{x}^{SR-HF} + E_{x}^{SR-DFA} + E_{c}^{DFA}
	\end{equation}
	
	\noindent où :
	
	\begin{equation}
	E_{x}^{SR-HF} = -\frac{1}{2} \sum_{\sigma} \sum_{ij}^{occ.} \iint \phi_{i \sigma}^{*}(r_{1}) \phi_{j \sigma}^{*}(r_{1}) \frac{erfc(\omega r_{12})}{r_{12}} \phi_{i \sigma}(r_{2}) \phi_{j \sigma}(r_{2}).dr_{1}.dr_{2}, \\
	\end{equation}
	
	C'est ainsi que la fonctionnelle $\omega$B97X\cite{chai2008long} se décompose de la façon suivante~:
	
	\begin{equation}
	E_{xc}^{\omega B97X} = E_{x}^{LR-HF} + c_{x}E_{x}^{SR-HF} + E_{x}^{SR-B97} + E_{c}^{B97}
	\end{equation}
	
	La valeur de $\omega$ et les valeurs des coefficients de développement linéaire et de développement à l'ordre $m$ des fonctionnelles $\omega$B97 et $\omega$B97X, ont été déterminées par la méthode des moindres carrés appliquée à une base de données composées de 412 valeurs précises, expérimentales et théoriques.
	
	Malgré toutes ces optimisations conduisant à une bien meilleure représentation des systèmes en interaction, ces fonctionnelles connaissent encore des lacunes quant à la traduction des interactions de dispersion entre atomes (\textit{i.e.} les forces de London). Comme nous allons le voir dans le cas de la fonctionnelle $\omega$B97X-D, ces écueils peuvent être corrigés par une prise en compte empirique des effets de dispersion.
	
	%%%%%%%%%%%%%%%%%%%%%%%%%%%%%%%%%%%%%%%%%%%%%%
	%%%%%%%%%%%%%%%%%%%%%%%%%%%%%%%%%%%%%%%%%%%%%%
	\subsubsection{$\omega$B97X-D}
	%%%%%%%%%%%%%%%%%%%%%%%%%%%%%%%%%%%%%%%%%%%%%%
	%%%%%%%%%%%%%%%%%%%%%%%%%%%%%%%%%%%%%%%%%%%%%%
	
	Cette dernière correction pourrait naturellement passer par le calcul de l'énergie de dispersion entre chaque atome, mais ceci engendrerait une augmentation démesurée du coût calculatoire. C'est pourquoi Jeng-Da Chai et Martin Head-Gordon ont fait le choix d'introduire cette correction de façon empirique au travers de l'ajout d'un terme, noté $E_{disp}$, à la fonctionnelle KS-DFT (dans le cadre du présent travail, la fonctionnelle $\omega$B97X). L'expression de l'énergie de la fonctionnelle $\omega$B97X-D \cite{chai2008long} ainsi obtenue devient alors :
	
	\begin{equation}
	E_{DFT-D}=E_{\omega B97X}+E_{disp}
	\end{equation}
	
	L'énergie de dispersion $E_{disp}$ est définie par rapport à une fonction d'amortissement $f_{damp}$ :
	
	\begin{equation}
	E_{disp}=-\sum_{i-1}^{N_{at}-1} \sum_{j-i+1}^{N_{at}} \frac{C_{6}^{ij}}{R_{ij}^{6}}f_{damp} (R_{ij})
	\end{equation}
	
	\noindent où :
	\begin{equation}
	f_{damp} (R_{ij})=\frac{1}{1+a(\frac{R_{ij}}{R_{r}})^{-12}}
	\end{equation}
	
	En conclusion, les travaux de Jeng-Da Chai et Martin Head-Gordon ont conduit à la fonctionnelle $\omega$B97X-D, de type LC-DFT-D hybride, dans laquelle la totalité de l'échange exact HF est prise en compte à longue distance, en même temps qu'une petite partie -- environ 22 \% -- de l'échange exact HF est introduite à courte distance pour compléter une fonctionnelle d'échange B97 modifiée ; une correction empirique de la dispersion est finalement appliquée. La partie empirique a été paramétrée par rapport à la base de données précédemment employée pour les fonctionnelles $\omega$B97 et $\omega$BX97.
	
	%%%%%%%%%%%%%%%%%%%%%%%%%%%%%%%%%%%%%%%%%%%%%%
	%%%%%%%%%%%%%%%%%%%%%%%%%%%%%%%%%%%%%%%%%%%%%%
	%%%%%%%%%%%%%%%%%%%%%%%%%%%%%%%%%%%%%%%%%%%%%%
	\subsection{DFT+D}
	%%%%%%%%%%%%%%%%%%%%%%%%%%%%%%%%%%%%%%%%%%%%%%
	%%%%%%%%%%%%%%%%%%%%%%%%%%%%%%%%%%%%%%%%%%%%%%
	%%%%%%%%%%%%%%%%%%%%%%%%%%%%%%%%%%%%%%%%%%%%%%
	
	Les fonctionnelles DFT-D -- \textit{i.e.} BLYP-D, B3LYP-D, B97-D, PBE-D, M05-2x et M06-2x -- sont traditionnellement incluses dans la famille des fonctionnelles dites \og {semi-empirical dispersion-corrected functionals \fg. En effet, pour ces fonctionnelles, les effets de corrélation à longue distance sont exclusivement traités par une correction empirique de dispersion selon trois niveaux d’empirisme notés respectivement DFT-D1, DFT-D2 et DFT-D3. 
	
	%%%%%%%%%%%%%%%%%%%%%%%%%%%%%%%%%%%%%%%%%%%%%%
	%%%%%%%%%%%%%%%%%%%%%%%%%%%%%%%%%%%%%%%%%%%%%%
	\subsubsection{DFT-D2}
	%%%%%%%%%%%%%%%%%%%%%%%%%%%%%%%%%%%%%%%%%%%%%%
	%%%%%%%%%%%%%%%%%%%%%%%%%%%%%%%%%%%%%%%%%%%%%%
	
	S. Grimme \cite{grimme2006semiempirical} propose en 2006 une première amélioration de sa correction empirique initiale de la dispersion \cite{grimme2004accurate}, appelée D2, dans le but de corriger trois défauts principalement constatés : 
	\begin{itemize}
	\item les coefficients C$_{6}$ n’étaient disponibles, dans la version originale, que pour les atomes les plus légers de la classification périodique
	\item l’utilisation des données tabulées pour les atomes de la troisième période conduisaient à des erreurs systématiques
	\item les résultats obtenus ne permettaient pas de retrouver certains principes usuels de la thermochimie.  
	\end{itemize}
	
	Conformément à ce que nous avons reporté dans le paragraphe précédent, l’expression de l'énergie de la fonctionnelle s’exprime toujours comme :
	
	\begin{equation}
	E_{DFT-D2} = E_{KS-DFT} + E_{disp}^{(2)}
	\end{equation}
	
	L'énergie de dispersion $E_{disp}$ est quant à elle de nouveau définie par rapport à une fonction d'amortissement $f_{damp}$ :
	
	\begin{equation}
	E_{disp}^{(2)}=-s_{6} \sum_{i=1}^{N_{at}-1} \sum_{j=i+1}^{N_{at}} \frac{C_{6}^{ij}}{R_{ij}^{6}} f_{d,6} (R_{ij})
	\end{equation}
	
	\noindent où :
	\begin{equation}
	f_{d,6} (R_{ij})= \frac{s_{6}}{1+exp^{(-d(\frac{R_{ij}}{s_{R}R_{Oij}}-1)}}  
	\end{equation}
	
	dans laquelle le paramètre $s_{6}$ est optimisé suivant la fonctionnelle d’échange-corrélation utilisée. Les coefficients de van der Waals d’ordre 6 et les rayons de van der Waals des espèces en interaction sont calculés à partir des données tabulées sur les atomes à partir des expressions suivantes :
	
	\begin{equation}
	C_{6ij} =\sqrt{C_{6ii}C_{6jj}}  
	\end{equation}
	
	\begin{equation}
	C_{6ii} = 0.05NI_{p}{i} \alpha_{i}
	\end{equation}
	
	et 
	
	\begin{equation}
	R_{Oij} = R_{Oi} + R_{Oj}
	\end{equation}
	
	Les termes $I_{p}{i}$ et $\alpha_{i}$ représentent respectivement le potentiel d’ionisation atomique et la polarisabilité statique de l’élément $i$.
	
	La correction D2 a été principalement testée par Grimme \cite{grimme2006semiempirical} pour l’étude des systèmes de référence que sont les gaz diatomiques, le benzène ainsi que de petites molécules aromatiques telles que l’anthracène. Quelques années plus tard, Park \textit{et al.} \cite{park2011ab} ont étudié des systèmes graphitiques et ont rapporté d’excellentes corrélations avec les données expérimentales, notamment en ce qui concerne la description des paramètres de maille. D’autres auteurs tels que Lee \textit{et al.} \cite{lee2013sum} ont commencé à publier, dès 2013, des travaux dont le but était de calculer des fréquences de vibration à l’aide du logiciel VASP.
	
	%%%%%%%%%%%%%%%%%%%%%%%%%%%%%%%%%%%%%%%%%%%%%%
	%%%%%%%%%%%%%%%%%%%%%%%%%%%%%%%%%%%%%%%%%%%%%%
	\subsubsection{DFT-D3}
	%%%%%%%%%%%%%%%%%%%%%%%%%%%%%%%%%%%%%%%%%%%%%%
	%%%%%%%%%%%%%%%%%%%%%%%%%%%%%%%%%%%%%%%%%%%%%%
	
	Quatre ans plus tard, en 2010, S. Grimme \cite{grimme2006semiempirical} propose une nouvelle amélioration de sa correction empirique initiale de la dispersion \cite{grimme2004accurate}, baptisée D3. Dans cette version, la majorité des termes empiriques sont calculés au niveau KS-DFT.
	
	L'énergie de la fonctionnelle s’exprime désormais comme :
	
	\begin{equation}
	E_{DFT-D3} = E_{KS-DFT} + E_{disp}^{(2)} + E_{disp}^{(3)}
	\end{equation}
	
	où : 
	
	\begin{equation}
	E_{disp}^{(2)}=- \sum_{i>j} \sum_{n=6,8,10,…} s_{n} \frac{C_{n}^{ij}}{R_{ij}^{n}} f_{damp}^{n} (R_{ij})
	\end{equation}
	
	\begin{equation}
	E_{disp}^{(3)}= -\sum_{i>j>k}^{N_{at}} \frac{C_{9}^{ijk}(3\cos\theta_{I}\cos\theta_{J}\cos\theta_{K}+ 1)}{(R_{ij} R_{jk} R_{ki})^{3}} f_{damp}^{(3)} (\overline{R}_{ijk})
	\end{equation}
	
	Dans cette dernière expression, les termes $\theta_{I}$, $\theta_{J}$ et $\theta_{K}$ représentent les angles internes du triangle $ijk$. Deux fonctions d'amortissement complètent l’expression de l’énergie :
	
		\begin{equation} f_{damp}^{n} (R_{ij}) =\frac{1}{1+6(R_{ij} / (s_{r,n} R_{O}^{ij}))^{-\alpha_{n}}} \end{equation}  
		  
		\begin{equation} f_{damp}^{(3)} (\overline{R}_{ijk}) =\frac{1}{1+6(\overline{R}_{ijk} / (4 R_{O}^{ijk}/3))^{-16}}\end{equation}

	
	Dans cette équation : 
	\begin{itemize}
	\item les rayons de coupure $R_{O}^{ij}$ et $R_{O}^{ijk}$ sont évalués de façon semi-empirique,
	\item les coefficients $s_{n}$ ($n$ = 8,10, \dots) sont ajustés à partir de références dont les valeurs sont ajustées selon la fonction d’échange-corrélation utilisée
	\item les coefficients de van der Waals sont évalués au moyen de calculs Time Dependant (TDKS) et à l’aide de relations récursives,
	\item le terme $C_{6}^{ij}$, jusque-là calculé au moyen de formules d’interpolation dérivées de manière empirique est, dans la version D3, évalué à partir de l’expression de Casimir et Polder selon la procédure rappelée en annexe de ce travail (Annexe A): 
	\end{itemize}
	
	\bigskip
	\begin{equation}
	C_{6}^{ij} = \frac{3}{\pi}\int_{0}^{\infty} \alpha^{i} (i\omega) \alpha^{j} (i\omega) d\omega
	\end{equation}
	\bigskip
	
	Les coefficients d’ordre supérieur sont quant à eux évalués à l’aide des formules de récursivité établies par Starkschall et Gordon \cite{starkschall1972error} : 
	
		\begin{equation}{C}_{8}^{ij} = 3{C}_{6}^{ij}\sqrt {{Q}_{i}{Q}_{j}}\end{equation} avec
		\begin{equation}Q_{i} = s\sqrt{Z^{i}} \frac{\langle r^{4}\rangle r^{i}}{\langle r^{2}\rangle r^{i}}\end{equation}
	
	
	
		\begin{equation} {C}_{10}^{ij}=\frac {49}{40} \frac{{\left({C}_{8}^{ij} \right)}^{2}}{{C}_{6}^{ij}} \end{equation} et
		\begin{equation} {C}_{n+4}^{ij}={C}_{n-2}^{ij}{\left(\frac{ {C}_{n+2}^{ij} }{ {C}_{n}^{ij} }\right)}^{3} \end{equation}
	
	
	En plus de fournir un potentiel aux longues distances plus précis, les potentiels DFT-D3 fournissent également une séparation plus nette entre les termes de courte et de longue portée. 
	
	
	%%%%%%%%%%%%%%%%%%%%%%%%%%%%%%%%%%%%%%%%%%%%%%
	%%%%%%%%%%%%%%%%%%%%%%%%%%%%%%%%%%%%%%%%%%%%%%
	\subsubsection{La méthode de Tkatchenko-Scheffer (DFT-TS)}
	%%%%%%%%%%%%%%%%%%%%%%%%%%%%%%%%%%%%%%%%%%%%%%
	%%%%%%%%%%%%%%%%%%%%%%%%%%%%%%%%%%%%%%%%%%%%%%
	
	Cette méthode a été proposée en 2009 par Tkatchenko et Scheffler\cite{tkatchenko2009accurate}. De nombreux travaux, essentiellement dédiés à l’étude de structures cristallines, ont été menés en utilisant cette correction. A titre d'exemple, citons notamment Kronik et Tkatchenko\cite{kronik2014understanding}, qui ont étudié la structure du cristal d'Hemozoin, substance potentiellement responsable des fortes fièvres caractéristiques du paludisme. Les auteurs ont démontré la fiabilité de la méthode en calculant en particulier les paramètres de maille du cristal. D’autres cristaux ont été ainsi caractérisés, tels que le cristal de Brushite -- qui n’est autre qu’un phosphate de calcium hydraté -- au moyen de calculs de modes de vibration. Très récemment, Bu\u{c}ko \textit{et al}. \cite{buvcko2014extending} ont montré que cette méthode pouvait également servir à l’étude de systèmes ioniques, à condition d’apporter quelques modifications au calcul du volume effectif et des charges des atomes. 
	
	Dans cette méthode, l’expression de l'énergie de dispersion est identique à celle établie pour la méthode DFT-D2. La différence majeure réside dans le calcul des coefficients de dispersion et de la fonction de damping qui sont, dans cette approche, dépendants de la densité de charge. Par ailleurs, la méthode DFT-TS est, en toute rigueur, également capable d'inclure les variations des contributions atomiques intervenant dans le calcul de l’énergie d’interaction, au moyen d'une prise en compte des effets d'environnement. 
	
	Les coefficients de dispersion, la polarisabilité et les rayons atomiques sont calculés à partir des expressions suivantes :
	
	
		\begin{equation} \alpha_{i} = \nu_{i} \alpha_{i}^{free} \end{equation}
		\begin{equation} C_{6ii} = \nu_{i}^{2} C_{6ii}^{free} \end{equation}
		\begin{equation} R_{0i} = \left(\frac{\alpha_{i}}{\alpha_{i}^{free}}\right)^{\frac{1}{3}} R_{0i}^{free} \end{equation}

	
	\noindent expressions dans lesquelles $\alpha_{i}^{free}$ est tel que :
	
	\begin{equation}
	\alpha_{i}^{free} = \frac{V_{i}^{eff}}{V_{i}^{libre}}
	\end{equation}
	
	Le terme $\alpha$ représente le rapport entre le volume effectif occupé par un atome $i$ dans une molécule ou un solide (noté $V_{i}^{eff}$), et le volume de l’atome isolé (noté $V_{i}^{libre}$) obtenu à l’aide d’une partition de Hirschfeld de la densité \cite{buvcko2014extending} :
	
	\begin{equation}
	\nu_{i} = \frac{V_{i}^{eff}}{V_{i}^{libre}} = \frac{\int r^{3} w_{i}(r)n(r)d^{3}r}{\int r^{3} n_{i}^{free} (r)d^{3}r}
	\end{equation}
	
	où $n_{i}^{free}$ est la densité sphérique moyenne de l’atome isolé $i$ et où $w_{i}(r)$ représente le poids de Hirschfeld, définit à partir des densités atomiques libres : 
	
	\begin{equation}
	w_{i}(r)= \frac{n_{i}^{free}(r)}{\sum_{j=1}^{N} n_{j}^{free}(r)}
	\end{equation}
	\bigskip
	
	Les paramètres $\alpha_{i}^{free}$, $C_{6ii}^{free}$ et $R_{0i}^{free}$ sont tabulés pour tous les éléments des premières périodes de la classification périodique, à l'exception des lanthanides.
	
	
	%%%%%%%%%%%%%%%%%%%%%%%%%%%%%%%%%%%%%%%%%%%%%%
	%%%%%%%%%%%%%%%%%%%%%%%%%%%%%%%%%%%%%%%%%%%%%%
	\subsubsection{Self-consistent screening de la méthode de Tkatchenko-Scheffer (DFT-TS-SCS)}
	%%%%%%%%%%%%%%%%%%%%%%%%%%%%%%%%%%%%%%%%%%%%%%
	%%%%%%%%%%%%%%%%%%%%%%%%%%%%%%%%%%%%%%%%%%%%%%
	
	En l'absence de champ électrostatique, les interactions entre deux systèmes sont décrites en fonction de leurs moments permanents et des polarisabilités statiques. Pour un champ périodique de pulsation $\omega$, les interactions sont modifiées et les polarisabilités dépendent de cette pulsation. L'étude des forces de dispersion à longue distance nécessite alors la connaissance d'intégrales qui tiennent compte de l'ensemble des fréquences $\omega$. Pour des raisons mathématiques évidentes, les intégrations de fonctions discontinues restent complexes. Il convient alors de se ramener à la connaissance de la variation des polarisabilités dépendantes des fréquences imaginaires de chacun des deux systèmes pris isolement, puisque ces fonctions sont continues (certains détails figurent en Annexe de ce travail). \\
	
	La variante DFT-TS-SCS de la méthode Tkatchenko-Scheffler (DFT-TS) a été proposée en 2012 par Tkatchenko \textit{et al}.\cite{tkatchenko2012accurate} avec pour objectif de tenir compte de la perturbation due à un champ oscillant de fréquence $\omega$. Par la méthode DFT-TS-SCS, la polarisabilité dynamique s'obtient à partir de la résolution de l’équation auto-cohérente :
	
	\begin{equation}
	\alpha_{i}^{SCS}(i \omega) = \alpha_{i}^{TS}(i \omega) - \alpha_{i}^{TS}(i \omega) \sum_{i\neq j} \tau_{ij} \alpha_{j}^{SCS}(i \omega)
	\end{equation} 
	
	dans laquelle $\alpha_{i}^{TS}(i \omega)$ représente la polarisabilité effective dynamique dépendante du champ, approximée par l’expression : 
	
	\begin{equation}
	\alpha_{i}^{TS}(i \omega) = \frac{\alpha_{i}^{TS}(0)}{1 + (\omega/\omega_{i})^{2}}
	\end{equation}
	
	La fréquence/pulsation du champ est estimée au moyen d’une formule empirique du type : 
	
	\begin{equation}
	\omega_{i} = \frac{4}{3} \frac{C_{6,ii}^{TS}}{(\alpha_{i}^{TS})^{2}}
	\end{equation}
	
	dans laquelle les termes statiques de polarisabilité $\alpha_{i}^{TS}$ et les coefficients de van der Waals d’ordre six $C_{6,ii}^{TS}$ sont estimés selon les équations présentées ci-avant lors de la présentation de la méthode DFT-TS.
	
	Finalement, les coefficients de dispersion d’ordre 6 utilisés dans l’expression de l’énergie de dispersion $E_{disp}$-D2 sont évalués au moyen de l’expression de Casimir-Polder :
	
	\begin{equation}
	C_{6}^{ij} = \frac{3}{\pi} \int_{0}^{\infty} \alpha_{i}^{SCS} (i\omega) \alpha_{i}^{SCS} (i\omega) d\omega
	\end{equation}
	
	et les rayons de van der Waals des atomes sont recalculés au moyen de l’expression : 	
	\begin{equation}
	R_{i}^{SCS} = \left(\frac{\alpha_{i}^{SCS}}{\alpha_{i}^{TS}}\right)^{1/3} R_{i}^{TS}
	\end{equation}
	
	\noindent qui fait intervenir les expressions des rayons atomiques, calculés par la méthode de Tkatchenko (DFT-TS).

	
	Tout comme dans l’approche DFT-D2, le paramètre empirique $s_{R}$ figurant dans l’expression de la fonction d’amortissement est généralement fixé à 0,97 (pour la fonctionnelle PBE) conformément aux recommandations de Bu\u{c}ko, Lebègue et al \cite{buvcko2013tkatchenko}.
	
	La méthode DFT-TS-SCS est disponible dans le code VASP (code que nous allons utiliser dans la thèse) grâce à Bu\u{c}ko, Lebègue \textit{et al}.\cite{buvcko2013tkatchenko}. Les auteurs ont étudié une large variété de solides (gaz nobles, cristaux ioniques et solides moléculaires), ainsi que des structures de type chaîne et des métaux. La méthode se révèle traduire avec une précision raisonnable les propriétés structurales et les énergies de cohésion, hormis pour les solides ioniques. La détermination des propriétés structurales, du volume de la maille, etc., sont autant de facteurs qu’il nous faut estimer avec le plus grand soin pour mener à bien les calculs de fréquences de vibration. Toutes les informations nécessaires à la comparaison des approches D2, D3, TS et TS-SCS seront données dans ce manuscrit dans le chapitre 6. Par ailleurs, les auteurs ont également montré que les effets dynamiques du champ périodique traités dans la méthode TS-SCS sont en général négligeables pour des systèmes en interaction faible, comme c'est le cas des atomes et des molécules neutres qui nous intéressent de prime abord dans ce travail.
		
