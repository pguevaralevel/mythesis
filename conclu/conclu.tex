	\newgeometry{textwidth=16cm}
	\chapter*{Conclusion Générale}
	\addcontentsline{toc}{chapter}{Conclusion Générale}
	\markboth{Conclusion Générale}{Conclusion Générale}
	\minitoc
	\restoregeometry
	
	
	
	Le cœur du présent travail de thèse consistait en une caractérisation théorique des spectres IR et Raman d'une série d'hétérocycles aromatiques définis comme modèles moléculaires en raison de leur représentativité de la diversité des systèmes asphalténiques. Ces derniers étant un enjeu conséquent pour l'industrie du pétrole, un effort de recherche se concentre en effet sur la compréhension de leur structure chimique et des interactions qui en découlent, responsables des propriétés physico-chimiques observées (agrégation en premier lieu, mais également précipitation et floculation).\\ 
	
	Pour répondre au double objectif de la résolution de la nature chimique des asphaltènes, d'une part, et des interactions à l'origine du phénomène d'agrégation, d'autre part, J. Shaw et K. Michaelian ont proposé un schéma de travail reposant sur la spectroscopie infra-rouge. Ce travail de thèse s'inscrit dans la mise en œuvre de leur méthodologie, en fournissant des données théoriques fiables à même de guider les expérimentateurs dans l'interprétation et la caractérisation de leurs spectres. L'idée générale de nos travaux est de caractériser les signatures vibrationnelles des modes inter et intra-moléculaires, de même que leurs modifications sous l'effet de l'environnement ou encore de la position et du nombre d'hétéroéléments. Pour peu que l'on parvienne à dégager de nos calculs des tendances, sinon des conclusions, quant à la fréquence et à la position des signatures vibrationnelles de ces modèles, en fonction de l'environnement et de la composition chimique, ces tendances serviront de guide à l'expérimentateur pour l'analyse des spectres réels.\\  
	
	La mise en place d'une stratégie calculatoire à même de répondre à ces problématiques s'est révélée ardue, du fait de la dimension des systèmes étudiés et de la nécessité d'incorporer à nos calculs des interactions longue portée, encore mal décrites par les méthodes de type DFT. Une part conséquente du présent travail a donc été consacrée au choix des outils méthodologiques les plus adaptés, ainsi qu'au choix d'approximations judicieuses, permettant de diminuer le coût calculatoire sans entamer la qualité des simulations. 
	Sur les sept chapitres regroupés dans ce manuscrit, trois sont ainsi consacrés au détail des méthodes et approximations employées, tandis qu'un chapitre à part entière présente les résultats de calculs préliminaires, menés sur des molécules test, dans le but de démontrer la pertinence de la stratégie calculatoire mise en place et son aptitude à traduire, avec une précision satisfaisante, les propriétés ciblées. 
	La part importante de ces chapitres préliminaires se veut représentative de l'effort méthodologique réalisé tout au long de ce travail de thèse, essentiel pour répondre au sujet ambitieux que représente la structure chimique des asphaltènes. Ces réflexions méthodologiques s'étendent à toutes les échelles de calcul envisagées, puisque les problématiques poursuivies sont multiples. Rappelons qu'il s'agit de déterminer l'influence 
	\begin{itemize}
	\item du nombre et de la position des hétéroéléments,
	\item de l'environnement (état gazeux, liquide ou solide),
	\item et des interactions à l'origine des processus d'agrégation
	\end{itemize}
sur les modes de vibration des systèmes asphalténiques.\\
 
	Par soucis de clarté, nous avons fait le choix d'associer à chacune de ces problématiques un chapitre spécifique, mais il est bien évident que les résultats obtenus, bien qu'à différents niveaux calculatoires (moléculaire, solide ou dynamique moléculaire), demeurent intimement liés. 
	Ainsi, notre chapitre cinq se concentrait sur l'influence des hétéroatomes, et présentait donc des résultats de calculs moléculaires. Dans cette partie, nos calculs portaient sur une série de six motifs moléculaires, contenant ou non une proportion d'hétéroéléments (N, S ou O), d'abord étudiés isolément puis sous forme de dimères (de sorte à simuler l'effet de l'interaction). Il est à noter que nos résultats se révèlent en bon accord avec les spectres expérimentaux, démontrant une fois encore la pertinence de notre prise en compte des anharmonicités électrique et mécanique pour la REVS. Toutefois, nos calculs ont révélé que les signatures vibrationnelles caractéristiques des modes intermoléculaires associés aux dimères se trouvaient aux bas nombre d'ondes (0-150 cm$^{-1}$), soit dans une zone traditionnellement mal décrites par les calculs moléculaires. Ces conclusions se déduisent de la confrontation des spectres vibrationnels des monomères et des dimères, qui montrent, pour les systèmes contenant de l'oxygène, du soufre et du carbone, une correspondance parfaite pour l'ensemble des bandes, à l'exception des bandes induites par les interactions intermoléculaires, que l'on voit apparaître aux bas nombre d'ondes sur les spectres des dimères. Le cas des motifs contenant de l'azote se révèle plus spécifique, puisque monomères et dimères présentent des différences dans la signature des modes de vibration intramoléculaires dans la région en-dessous de 600 cm$^{-1}$. De fait, les systèmes contenant de l'azote peuvent être plus aisément caractérisés en tant que systèmes isolés ou en tant que systèmes en interaction, au travers de l'analyse de deux zones spectrales distinctes. Pour l'ensemble des autres systèmes, la conclusion quant à la présence ou l'absence d'interactions se déduit d'une analyse aux bas nombres d'onde. 	
Du point de vue structural, et non plus strictement vibrationnel, les configurations envisagées pour les dimères modèles étudiés révèlent qu'il n'existe pas d'orientation privilégiée des hétéroéléments, de sorte que l'interaction prépondérante demeure le $\pi$-stacking.\\ 	

	
	Le passage de l'état gazeux, simulé au travers de calculs moléculaires, à l'état solide, présenté dans le chapitre six, avait pour objectif de comprendre l'effet de l'environnement en caractérisant la modification des signatures des modes de vibration précédemment analysés. Les modes de réseau calculés, par la prise en compte d'un environnement plus représentatif des systèmes réels, ont ainsi été comparés aux modes de vibration intermoléculaires présentés au chapitre cinq. Cette comparaison révèle que, pour l'ensemble des systèmes aromatiques considérés, une bande active caractéristique des interactions intermoléculaires se retrouve systématiquement aux alentours de 100 cm$^{-1}$, quels que soient la taille du système, la nature du noyau aromatique (benzénique ou non) ou encore le type d'hétéroélément présent.\\ 

 
 
 	Enfin, les simulations en dynamique moléculaire menées dans le but d'analyser l'influence des hétéroéléments et des proportions de mélanges des espèces modèles sur le comportement de nano-agrégats, ont fait l'objet d'une analyse au chapitre sept. De ces modélisations, il ressort que les noyaux polyaromatiques, au cœur de la structure chimique des asphaltènes, jouent un rôle prépondérant dans l'agrégation de ces systèmes. La position des hétéroatomes au sein des modèles se révèle influer sur les interactions de $\pi$-stacking à l'origine du phénomène. Toutefois, bien que la proportion d'hétéroatomes soit conséquente au sein des asphaltènes -- ce qui porte à penser qu'ils possèdent une influence remarquable dans les processus de nano-agrégation --, ce sont la taille et la nature des ramifications qui jouent véritablement un rôle décisif dans la description du phénomène. \\
	
	
	En conclusion générale, les modes de vibration caractérisés tout au long de ce travail permettent de fournir aux expérimentateurs des critères d'analyse pour l'interprétation des spectres IR et Raman. 
	Le résultat principal de la présente étude réside dans l'interprétation des modes de vibration intermoléculaires, caractéristiques des interactions de type $\pi-\pi$ et $\pi-H$, identifiables aux bas nombres d'onde (soit aux alentours de 100 cm$^{-1}$). La présence, sur les spectres expérimentaux, de bande(s) dans cette zone révèle sans ambiguité l'existence de systèmes en interaction. Au-delà de ce résultat, au demeurant directement utilisable par les spectroscopistes, l'ensemble de nos calculs représente une avancée vers la compréhension de la structure et des propriétés des asphaltènes. Notamment, la méthodologie calculatoire poursuivie, de par sa fiabilité, ainsi que les différentes échelles de calcul utilisées, font de cette exploration un véritable travail de développement. Pour la rendre complète, la finalisation de cette étude devra toutefois passer par :
	
	\begin{itemize}
	\item le calcul de spectres au bas nombre d'ondes, sur des systèmes pour lesquels on ne dispose pas de données de référence, dans le but de valider nos conclusions
	\item une analyse, par exemple, des composantes principales (ACP) des spectres vibrationnels, afin d'étudier la distribution des échantillons (technique déjà employée en archéologie et en pharmacologie), problématique cruciale dès que l'on dispose d'un trop grand nombre de données à analyser et/ou a hierarchiser comme dans le cas des asphaltènes. 
	\end{itemize}













 	
	
	
	
	
	

	
	
	