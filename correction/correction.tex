\newgeometry{textwidth=16cm}
\chapter[Corrections à la dispersion]{Corrections à la Dispersion}
\minitoc
\restoregeometry

\newpage

	\section*{Introduction}
\spacing{1.5}
	
	En mécanique quantique les molécules sont décryptes comme un assemblage de noyaux et d'électrons attaché aux effets classiques et quantiques. Même en considérant les noyaux comme fixes, le comportement des électrons constituent un problème complique à N corps. Pour la résolution de l'équation de Schrödinger la théorie de la fonctionnelle de la densité établie une alternative intéressante pour aborder ce problème, grâce à la description du système par sa densité électronique\cite{adamo2014decrire}.
	
	\bigskip
	La DFT a eu un grand succès dans l'étude de systèmes dans lesquels la taille fait impossible un calcul ab initio Post Hartree- Fock, pour la compréhension parmi différents sujets en sciences et technologie.\cite{}
	\bigskip
	
	Cependant cette méthode présente déficiences au moment de traduire la dispersion dans les systèmes et les interactions du van der Waals, voire le théorème Hohenberg- Kohn prédit que si nous connaissons la fonctionnelle correcte, nous pourrions obtenir l’énergie à l’état fondamental en arrivant à son point minimum, donc la difficulté provient des fonctionnelles approximatifs. Local et semi local fonctionnelles  ne peuvent pas reproduire le comportement asymptotique de $1/R^{6}$, comme cela affecte les molécules où il y a une considérable superposition de la densité est inconnue. Il existe une corrélation claire de l'incapacité de décrire les interactions de Van der Waals avec la fonctionnelle d'échange GGA dans la région de faible densité et gradient de densité réduite x. 
	
	
	L'échec de LDA et GGA pour donner l'énergie correcte pour deux densités fixes a longue distances a conduit à des corrections. 
	
	
	\section{DFT-D3}
	
	Les proprietes que presente cette correction par rapport aux versiones precedents sont les suivantes :
	\bigskip 
	\begin{itemize}
		\item Elle est moins empirical, les parametres les plus importants sont calculés par KS-DFT.
		\item L'approche est asympotiquement correcte pour les fonctionnelles de la densité pour les systemes limités (molecules) ou non metallic infinte systemes.
		\item Elle fournit une description pour les elements plus revelants (Z=1-94).
		\item Les coefficients de la dispersion des atoms paires et leur cutoff radio sont calculés. 
	\end{itemize}
	\bigskip
	
	L'énergie totale DFT-D3 est donnée par la relation:	
	\begin{equation}
	E_{DFT-D3} = E_{KS-DFT} - E_{disp}
	\end{equation} 
	\bigskip
	
	Où $E_{KS-DFT}$ est l'énergie KS usuel obtenué par la fonctionnelle de la densité choisie, et $E_{disp}$ est la correction à la dispersion qui est represente comme une somme de l'énergie de deux ou trois corps.
	
	\begin{equation}
	E_{disp} = E ^{(2)} + E^{(3)}
	\end{equation}
	
	Le plus important terme de deux corps expression est :
	\begin{equation}
	E^{(2)} = \sum_{AB} \sum_{n=6,8,10...} s_{n}\frac{C_{n}^{AB}}{r^{n}_{AB}} f_{d,n} (r_{AB})
	\end{equation}
	Le premier somme est sur tous les atomes paires du systèmes. $C_{n}^{AB}$ indique la moyenne de nème ordre des coefficients de la dispersion pour les atomes paires AB, et $r_{AB}$ est la distance internucléaire entre eux. Global $s_{n}$ ou facteur d'écaillage est ajusté seulement pour $n>6$ pour assurer l'exactitude asymptotique qui est remplie quand $C_{6}^{AB}$ est exacte. 
	\bigskip
	
	$C_{6}$ terme n'est pas largement écaillage, les plus hauts termes sont nécessaires afin d'adapter le potentiel pour la fonctionnelle de la densité utilisée dans sa région moyenne. Il a été constaté que les termes $n>8$ fassent la methode legerement instable dans certains situations, pour cela il a été fait une troncature à $n=8$.
	\bigskip
	
	Pour eviter les singularités dans petits $r_{AB}$ et double comptage des effets de correlation pour distances intermediares, la fonction damping $f_{d,n}$ est employée pour determiner le range de la correction de la dispersion.
	\bigskip
	
	Il est utilisé une variante proposée par Chai et Head-Gordon\cite{chai2008long} laquelle revele être numeriquemente stable et practique égalmente pour les termes d'ordre superieur de la dispersion. l'expression de cette fonction est donnée par :
	
	\begin{equation}
	f_{d,n} (r_{AB}) = \frac{1}{1+ 6(r_{AB}/(s_{r,n}R_{0}^{AB}))^{-\alpha_{n}}}
	\end{equation}
	\bigskip
	
	Où $s_{r,n}$ est le facteur d'écaillage dependant du radio de cutoff $R_{0}^{AB}$. Il remplace $s_{6}$ terme en DFT-D2. 
	\bigskip
	
	Compare au potentiel DFT-D2 le nouveau est moins contraignant pour les petites distances
	mais plus attrayante dans la region typique des interactions de van der waals. Il fournit une séparation plus nette entre le court terme et les effets de dispersion à longue portée.
	
	\subsection{Les coefficients de la dispersion}
	
	
	Au lieu d'utiliser une formule d'interpolation dérivée de manière empirique comme dans DFT-D2, les coefficients de la dispersion sont maintenant calculés par DFT dépendant du temps (TD-DFT), employant des relations de récursion connues pour les termes d'ordre supérieur des multipoles. Le point du depart est la formule de Casimir- Polde\cite{kaplan2006intermolecular} 
	\bigskip
	\begin{equation}
	C_{6}^{AB} = \frac{3}{\pi}\int_{0}^{\infty} \alpha^{A} (i\omega) \alpha^{B} (i\omega) d\omega
	\end{equation}
	\bigskip
	
	Où $\alpha(i\omega)$ est la moyenne du dipole de polarisabilité à la frequence imaginaire $\omega$. Les coefficientes d'ordre superieur sont calculés recursivement par :
	\bigskip
	\begin{equation}
	C_{8}^{AB} = 3C_{6}^{AB} \sqrt{Q^{A}Q^{B}}
	\end{equation}
	
	et \begin{equation}
	C_{10}^{AB} =\frac{49}{40} \frac{(C_{8}^{AB})^{2}}{C_{6}^{AB}}
	\end{equation}
	
	Avec
	\begin{equation}
	Q^{A} = s_{42}\sqrt{Z^{A}} \frac{\langle r^{4}\rangle r^{A}}{\langle r^{2}\rangle r^{A}}
	\end{equation}
	
	\bigskip
	\subsection{Le terme de Trois Corps}
	\bigskip
	
	La partie à long terme de l'interaction entre trois atomes à l'état fondamental ne correspond pas exactement aux énergies d'interaction prises par paires. Au meilleur de nos connaissances, nous ne sommes pas conscients de toute considération de cet effet dans une structure modele. Le premier terme non additif de dispersion dérivé du
	la théorie des perturbations de troisième ordre pour trois atomes ABC est :
	
	\begin{equation}
	E^{ABC} = \frac{C_{9}^{ABC}(3\cos\theta_{a}\cos\theta_{b}\cos\theta_{c}+ 1)}{(r_{AB} r_{BC} r_{CA})^{3}}
	\end{equation}
	
	\bigskip
	\section{La méthode Tkatchenko-Scheffler (DFT-TS)}
	
	Elle a été propose en 2009 par Tkatchenko et Scheffler\cite{tkatchenko2009accurate}. Nombreux études des structures cristallines ont été menés en utilisant cette correction. Par exemple Kronik et Tkatchenko\cite{kronik2014understanding} étudient le cristal d'Hemozoin d'importance biologique, lequel présent des interactions sur les liaisons hydrogènes. Pour vérifier l'exactitude de la méthode ils sont calculés les paramètres de maille et les modes de vibration du Brushite et le phosphate de calcium hydraté. Bu\u{c}ko et col\cite{buvcko2014extending} ont montre que la méthode peut être utilisée dans systèmes ioniques en changeant le calcul de volume effectif et les charges des atomes par la version itératif de la partition de Hisrfeld.
	
	Dans cette méthode la expression pour l'énergie de dispersion est officiellement identique à la méthode DFT-D2, la plus important différence cependant c'est que les coefficients de dispersion et la fonction damping sont dépendants de la densité de charge. La méthode DFT-TS est capable de prendre en compte les variation à la contribution des interactions de van der waals des atomes dû à leur environnement. 
	
	En effet les coefficient de la dispersion, la polarisabilité et les radios atomiques sont calculés à partir des valeurs des atomes libres selon les expressions suivantes :
	
	\begin{equation}
	\alpha_{i} = \nu_{i} \alpha_{i}^{free}
	\end{equation}
	
	\begin{equation}
	C_{6ii} = \nu_{i}^{2} C_{6ii}^{free}
	\end{equation}
	
	\begin{equation}
	R_{0i} = \left(\frac{\alpha_{i}}{\alpha_{i}^{free}}\right)^{\frac{1}{3}} R_{0i}^{free}
	\end{equation}
	
	Les paramètres $\alpha_{i}^{free}$, $C_{6ii}^{free}$ et $R_{0i}^{free}$ sont tabulé pour tous les éléments des premiers 6 lignes du tableau périodique des éléments, sauf pou les lanthanides. $\nu_{i}$ représente le volume effectif et il est calculé en utilisant la partition de Hirshfeld de la densité des électrons. 
	
	\begin{equation}
	\nu_{i} = \frac{\int r^{3} w_{i}(r)n(r)d^{3}r}{\int n_{i}^{free} (r)d^{3}r}
	\end{equation}
	
	Où n(r) est la densité totale électronique, $n_{i}^{free}$ est la moyenne de la densité électronique sphérique des électrons pour les espèces atomiques neutres libres i. $w_{i}(r)$ est le poids de Hisrfeld et il se definit par les densités atomiques libres.
	
	\begin{equation}
	w_{i}(r)= \frac{n_{i}^{free}(r)}{\sum_{j=1}^{N} n_{j}^{free}(r)}
	\end{equation}
	\bigskip
	
	L'energie de dispersion su systeme AB est egal : 
	
	\begin{equation}
	E_{dis}= -\frac{1}{2} \sum_{A=1}^{N} \sum_{B=1}^{N} \sum_{L} \frac{C_{6AB}}{R^{6}_{AB,L}} f_{damp}(R_{AB,L})
	\end{equation}
	\bigskip
	
	\subsection{Self-consistent screening dans la méthode Tkatchenko-Scheffler}
	
	Ce variant de la méthode Tkatchenko-Scheffler a été proposé en 2012 par Tkatchenko et col\cite{tkatchenko2012accurate} avec l'objectif d'inclure le dépistage des effets à longe portée des polarisabilités effectifs des atomes. Grâce à l'utilisation de l'équation self-consistent screening d'électrodynamique classique. 
	
	Cela conduit à une description clair de polarisation et dépolarisation pour le tenseur de polarisabilité des molécules et solides. 
	
	Dans cette méthode la fréquence dépendante de la polarisabilité s'obtient la équation self-consistent :
	
	\begin{equation}
	\alpha_{i^{SCS}}(\omega) = \alpha_{i}(\omega) - \alpha_{i}(\omega) \sum_{i\neq j} \tau_{ij} \alpha_{j}^{SCS}(\omega)
	\end{equation} 
	
	où $\tau_{ij}$ est le tenseur d'interaction dipôle- dipôle et $\alpha_{i}(\omega)$ est la fréquence effectif dépendant de la polarisabilité, avec l'expression approximatif : 
	
	\begin{equation}
	\alpha_{i} (\omega) = \frac{\alpha_{i}}{1 + (\omega/\omega_{i})^{2}}
	\end{equation}
	
	avec la fréquence d'excitation moyenne caractéristique $\omega_{i} = \frac{4}{3} \frac{C_{6ii}}{(\alpha_{i})^{2}}$
	\bigskip
	
	Les coefficients de la dispersion sont calcules par l'intégral de Casimir- Polder :
	
	\begin{equation}
	C_{6ii} = \frac{3}{\pi} \int_{0}^{\infty} \alpha_{i}^{SCS} (\omega) \alpha_{i}^{SCS} (\omega) d\omega
	\end{equation}
	
	Les radios atomiques de van der waals sont calcules par renormalisation du radio obtenu par la méthode de Tkatchenko (DFT-TS). 
	
	\begin{equation}
	R_{0i}^{SCS} = \left(\frac{\alpha_{i}^{SCS}}{\alpha_{i}}\right)^{1/3} R_{0i}
	\end{equation}
	
	L'energie de dispersion est évaluée en utilisant la même équation DFT-TS, avec les paramètres corrigés $C_{6ii}^{SCS}$, $\alpha_{i}^{SCS}$ et $R_{0i}^{SCS}$
	
	Cette methode est disponible sur le code VASP grace à Bu\u{c}ko et col\cite{buvcko2013tkatchenko} qui étudiaient une large gamme de solides comme gaz nobles, cristals ioniques et moléculaires, structures type chaines et métaux. La méthode conduit à des valeurs raisonnablement précises des propriétés structurelles et de cohésion pour divers types de matières solides. Cependant, l'analyse des résultats a montré que les performances ne sont pas aussi bons pour tous les systèmes et il y a certains types de solides où l'approche échoue définitivement. C'est ainsi que pour les cristaux moléculaires le volume calculé est tout à fait exact, l'énergie de cohésion est exacte pour l'azote, mais surestimée dans une certaine mesure pour les autres systèmes. Ils ont montré que les effets de dépistage sont assez petites pour les systèmes constitués par interactions faibles des atomes neutres et molécules. La méthode est moins performante dans les solides ioniques.