\newgeometry{textwidth=16cm}
	\chapter*{Introduction}
	\addcontentsline{toc}{chapter}{Introduction}
	\markboth{Introduction}{Introduction}
	\minitoc
	\restoregeometry
	
	
	Les asphaltènes représentent une fraction du pétrole brut, définie sur la seule base des propriétés de solubilité de ce dernier. Ainsi, appartiennent à la famille des asphaltènes les composés chimiques du pétrole solubles dans les solvants aromatiques, tels que le toluène, mais insolubles dans les $n$-alcanes, tels le $n$-pentane ou le $n$-hexane. Cette définition, limitée à une unique propriété physique, laisse comprendre implicitement que la nature chimique des asphaltènes, c'est-à-dire leur composition aussi bien que le type et la forme des différentes structures moléculaires qui les constituent, reste mal définie. En somme, ces systèmes chimiques sont pareils à un puzzle dont on ne possèderait que l'emballage. Les propriétés macroscopiques -- la solubilité -- correspondent à l'image du puzzle présentée sur la boîte, mais le nombre de pièces, de même que leurs formes et la manière dont elles s'assemblent, nous sont inconnus. Tel est le défi auquel font face depuis de nombreuses années les chercheurs, aidés de leur palette de techniques expérimentales et de méthodes théoriques. L'enjeu est de taille, tant cette famille de composés pose de problèmes, aux divers stades de l'extraction, de la production et du raffinage du pétrole. Dans l'industrie pétrolière, le seul mot \og asphaltène \fg{} est invariablement associé aux phénomènes d'agrégation caractéristiques de ces systèmes, responsables entre autres de l'obstruction des pipelines. \\
	
	Au fil des années, diverses techniques de caractérisation ont permis de lever le voile sur une partie des propriétés chimiques des asphaltènes. Il est désormais admis que les asphaltènes sont formés d'un ensemble de structures moléculaires composées de noyaux aromatiques poly-condensés, ramifiés par des chaînes alkyles. La majeure partie des propriétés physiques macroscopiques de ces systèmes, notamment leur propension à l'agrégation et leur comportement en solubilité, découlent toutefois de ce qu'ils présentent un taux significatif d'hétéroatomes (N, S et O) et de métaux (V et Ni), à l'origine d'interactions intermoléculaires caractéristiques, comme les liaisons hydrogène, les interactions acido-basiques, la formation de complexes métalliques, les phénomènes de \og $\pi-\pi$ stacking \fg{} et de \og $H-\pi$ stacking \fg{}. Comme il sera détaillé dans le \textbf{premier chapitre} du présent travail, l'ensemble des techniques de caractérisation potentiellement utilisables sur ces fractions les plus lourdes du pétrole brut ont été employées à maintes reprises, au gré des améliorations technologiques, pour tenter de percer à jour la structure chimique intime des asphaltènes. Bien que ces travaux aient pour certains offert des avancées remarquables, nombreuses sont les études -- expérimentales comme théoriques -- dont les conclusions \textit{a priori} contradictoires suscitent la controverse. Si l'on a pu jusqu'à présent identifier quelques-uns des atomes à l'origine des interactions responsables des phénomènes parasites observés aux différentes étapes de la production du pétrole, l'état actuel des connaissances ne nous permet pas de donner une définition structurale précise pour ces systèmes. \\
	
	Le présent travail de thèse prend naissance dans l'idée originale proposée en 2011 par J. Shaw et K. Michaelian, d'employer les spectroscopies infra-rouge (IR) et Raman -- en tant que techniques de caractérisation courantes -- à la base d'un \og protocole \fg{} innovant de caractérisation moléculaire des asphaltènes. Dans le projet proposé par ces chercheurs, la spectroscopie IR sert de pivot, et sa simplicité de mise en œuvre, en même temps que les principes physiques sur lesquels elle repose, permettent de la coupler aisément à une large variété de techniques expérimentales.\\
	
	En première étape à ce \og protocole \fg, J. Shaw et K. Michaelian proposent d'employer des analyses IR et Raman à la constitution d'une large base de données, dont le but serait de faciliter l'identification des motifs moléculaires constitutifs des asphaltènes. 
	Dans une seconde étape, le couplage des techniques d'IR (lointain et de photo-acoustique) permettrait de caractériser la présence -- ou non -- d'interactions de type $\pi-\pi$. Cet approfondissement présenterait alors le double intérêt de vérifier la validité des familles de molécules précédemment identifiées, et d'associer à ces interactions une fréquence et une intensité, deux paramètres physiques caractéristiques. \\
	
	Pour répondre à ces attentes, le soutien des méthodes de la chimie quantique pour la caractérisation des propriétés vibrationnelles apparaît crucial, et c'est dans ce cadre que s'inscrit ce travail de thèse. En aidant à l'identification des signatures caractéristiques des différentes familles de composés et des modes inter-moléculaires qui leur sont associés, la chimie calculatoire offre un guide précieux aux expérimentateurs, et contribue à la construction des deux étapes du schéma proposé par J. Shaw et K. Michaelian.\\
	
	Au vu de la diversité des asphaltènes, du point de vue de la nature chimique et de la taille, la première partie de ce travail de thèse a consisté en une réflexion en termes de stratégie calculatoire et méthodologique, le but étant d'accéder aux données vibrationnelles calculées dans les hypothèses harmonique et anharmonique. Ce travail préliminaire, véritable clé de voûte de notre investigation théorique, sera détaillé dans le \textbf{second chapitre} du présent manuscrit.\\
	
	Au-delà de la stratégie calculatoire, la nature méconnue des asphaltènes et leur mélange au sein des fractions lourdes du pétrole posent de nombreux problèmes de mise en œuvre. 
	Tout d'abord, la diversité des composés chimiques constitutifs de ces systèmes nous a contraint à faire le choix de molécules modèles judicieuses, aptes à représenter les différentes classes de composés postulées comme présentes au sein des asphaltènes. 
	Par ailleurs, le comportement en mélange de ces différentes familles de composés induit, dans le cadre des modélisations, deux problématiques majeures, que sont : 
	\begin{itemize}
	\item la prise en compte des effets de l'environnement chimique d'un système, de sorte à apporter une description théorique aussi proche que possible des conditions de mesure
	\item la traduction, en termes de description quantique, de l'ensemble des interactions identifiées entre ces systèmes, et notamment les interactions longue portée.
	\end{itemize}
	
	La taille des systèmes à l'étude a imposé le recours aux méthodes de la théorie de la fonctionnelle de la densité (DFT). Toutefois, il est bien connu que le talon d'Achille de ces méthodes demeure la description des forces de dispersion, pour laquelle les avancées méthodologiques sont encore lentes et parfois mises en défaut. La nécessité, dans la présente étude, de traduire les effets d'interactions de type longue portée, nous impose de présenter, dans le \textbf{troisième chapitre}, les solutions proposées jusqu'à ce jour pour palier certains des défauts de la DFT dans ce domaine. Cette partie abordera notamment la notion de fonctionnelle hybride et détaillera le principe des approches employées dans ce travail, qui ajoutent \textit{ad hoc} à un calcul Kohn-Sham (KS) usuel des fonctions semi-empiriques. Ces approches sont reconnues aujourd'hui comme étant les plus aptes à rendre compte des effets de dispersion dans le cadre de l'utilisation des méthodes de la DFT. \\
	
	À la suite de ces trois chapitres, indispensables pour établir la méthodologie suivie pour répondre à notre problématique, nous reviendrons dans une quatrième partie sur la mise en œuvre pratique de notre stratégie calculatoire. Celle-ci présentera donc les calculs préliminaires menés sur des molécules test (à savoir des composés thiophéniques), pour lesquelles nous disposions d'une grande quantité d'informations moléculaires de référence, afin d'évaluer la fiabilité des approximations auxquelles nous avons eu recours. En effet, comme il aura été explicité plus avant, le traitement quantique de ces systèmes de grande dimension, qui repose sur la résolution de l'équation vibrationnelle de Schrödinger (REVS) dans la double hypothèse des anharmonicités électrique et mécanique, requiert de réduire la taille de l'espace actif des configurations vibrationnelles. De fait, dans l'objectif de parvenir à un ratio acceptable temps \textit{versus} précision des calculs réalisés, nous nous sommes employés à la construction de bases réduites de modes normaux. L'emploi de ces bases réduites, déduites des couplages observés entre les modes inter-moléculaires et les modes de torsion angulaire et de stretching des liaisons C-H, sera donc corroboré et justifié par les résultats présentés au \textbf{chapitre 4}.  \\
	
	Les trois derniers chapitres de ce manuscrit présentent l'ensemble des résultats obtenus par le biais des stratégies calculatoires mises en place, en gardant à l'esprit que notre travail se veut traiter des différentes problématiques inhérentes aux asphaltènes, que sont : leur nature chimique, et particulièrement le rôle des hétéroéléments, l'influence de l'environnement et les interactions à l'origine de l'agrégation caractéristique de ces systèmes.\\ 
	
	C'est pourquoi le \textbf{chapitre 5} se propose de décrire les résultats des calculs moléculaires réalisés dans le but d'élucider le rôle singulier des hétéroatomes dans les signatures vibrationnelles de ces systèmes. Ces calculs, qui emploient les outils méthodologiques développés pour la REVS, ont pour double objectif : 
	\begin{itemize}
	\item de déterminer la signature propre de molécules modèles isolées
	\item d'analyser les modifications respectives de ces signatures sous l'effet des interactions inter-moléculaires, simulées sur des dimères. 
	\end{itemize}

	Afin d'assurer une analyse complète, nous avons mené nos calculs sur une collection de six motifs moléculaires, pour les trois hétéroéléments (N, S et O), ainsi que sur des motifs identiques pris pour références, ne comportant que des atomes de carbone.\\


	Le \textbf{chapitre 6} se concentre quant à lui sur l'effet de l'environnement, simulé par des calculs à l'état solide. Ce dernier constitue en effet l'état le plus proche des conditions réelles, pour lequel nous possédons des données expérimentales de référence aux très bas nombre d'onde obtenues en photo-acoustique sur des systèmes cristallins. Le passage de l'état moléculaire à l'état solide induit bien évidemment de profondes modifications dans la description des modes de vibration intermoléculaires. Nos calculs visent ainsi à caractériser explicitement ces modes en identifiant des singularités dans les signatures spectrales, avec pour objectif final de fournir aux expérimentateurs des caractéristiques précises leur permettant d'identifier ou non la présence de systèmes associés. \\ 
	
	Enfin, le \textbf{chapitre 7} de ce manuscrit développe les conclusions de travaux menés dans le sens d'une meilleure compréhension des phénomènes d'agrégation des asphaltènes. Les calculs réalisés dans cet objectif se placent donc à une toute autre échelle et permettent d'inscrire le présent travail dans une étude complète de type \og bottom-up \fg menée au laboratoire. Seront ainsi présentées dans cette partie des simulations en dynamique moléculaire, dont l'objectif est de comprendre le rôle des interactions intermoléculaires dans la nano-agrégation (les échelles supérieurs étant la micro agrégation et  la floculation toutes deux responsables des désagréments observés par les industriels à chaque fois que des asphaltènes sont présents dans les matrices de pétrole). Ces simulations seront de deux types, chacune présentant un objectif distinct : 
	\begin{itemize}
	\item dans un premier temps, nous considèrerons des systèmes modèles des asphaltènes, dans le but de comprendre l'influence de la position des hétéroéléments au sein de la moléculle. En effet, suivant leur position, différentes sortes d'interactions spécifiques se créent (liaisons hydrogène, interactions $\pi-\pi$ ou $\pi-H$, \textit{etc.})
	\item dans un second temps, nous nous baserons sur des systèmes moléculaires mis en évidence très récemment par le biais de la spectroscopie XPS, et pouvant être considérés \textit{a priori} comme plus réalistes. L'intérêt est ici d'étudier le comportement des agrégats suivant les proportions dans lesquelles sont mélangés ces systèmes et suivant la nature et la teneur des hétéroatomes qui les constituent.   
	\end{itemize} 




 









 	
	
	
	
	

		
	
	
	