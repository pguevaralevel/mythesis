\newgeometry{textwidth=16cm}
	\chapter*{Résumé}
	\markboth{Résumé}{Résumé}
	\pagenumbering{Roman}
	\minitoc
	\restoregeometry
	
	
	Les asphaltènes, constituants indésirables et méconnus des fractions lourdes du pétrole brut, représentent depuis de nombreuses années un enjeu de recherche majeur. Le défi principal posé par cette famille de composés porte sur leur caractérisation moléculaire précise, seule à même d’expliquer les phénomènes d’agrégation parasites qu’ils engendrent, à l’origine notamment des dépôts solides observés dans l’industrie à tous les stades de la production du pétrole. En effet, bien qu’il soit admis que ces molécules présentent une structure polyaromatique, organisée autour d’un noyau d’hétérocycles condensés faisant intervenir des hétéroéléments aussi bien que des groupements aliphatiques et naphténiques, leur description structurale demeure imprécise et sujette à débat. De fait, depuis la découverte des asphaltènes à la fin du XIXème siècle, la description de ces composés repose sur leurs propriétés de solubilité, seule caractéristique apte à les identifier sans ambiguïté au sein des mélanges complexes que constituent les fractions lourdes du pétrole. 
	À ce jour, la quasi-totalité des techniques de caractérisation dont disposent les physico-chimistes ont été employées avec plus ou moins de succès pour tenter de résoudre la problématique de la structure moléculaire des asphaltènes. En dépit des moyens mis en œuvre, les résultats fournis par ces méthodes, incomplets et parfois en apparence contradictoires, n’ont toujours pas permis de lever le voile sur la nature chimique de ces molécules.\\
	 
	Dans cette optique et considérant les écueils passés, le couplage expérience-théorie nous est apparu comme une approche de choix pour l’étude des asphaltènes. Techniques majeures pour l’analyse de systèmes complexes, les spectroscopies IR et Raman ont été employées dans ce travail, à l’appui d’investigations théoriques poussées permettant d’aider à l’identification des bandes caractéristiques. Notre approche, basée sur la mise en œuvre d’un protocole pour l’identification des modes de vibration intermoléculaires, parmi les modes intramoléculaires, sert ainsi de base à la caractérisation de systèmes en interaction.\\
	 
	Pour ce faire, une série d’hétérocycles aromatiques (comprenant des atomes de soufre, d’oxygène ou d’azote), ainsi que leurs analogues carbonés, ont été préalablement étudiés au niveau moléculaire par une approche de type DFT ($\omega$B97x-D/6-311++G**) menée dans l’hypothèse anharmonique, complétée par une étude approfondie à l’état solide par deux méthodes de DFT incluant des corrections sur le terme de dispersion. Ces calculs, réalisés dans l’objectif d’offrir aux expérimentateurs un guide d’interprétation des spectres IR, ont été complétés d’une étude en dynamique moléculaire basée sur des résultats de mesures XPS reportées très récemment dans la littérature. Cette analyse, bien que préliminaire, vise à modéliser le phénomène d’agrégation des asphaltènes et complète efficacement l’étude entreprise au niveau DFT.\\
	 
	Au regard de l’objectif affiché de notre protocole, les calculs réalisés au préalable sur les dimères d’hétérocycles révèlent que les modes de vibration intermoléculaires associés aux systèmes asphalténiques apparaissent aux alentours de 100-150 cm$^{-1}$. La position de ces modes est caractéristique, que l’on considère un monomère ou un dimère, pour les systèmes comprenant des atomes de soufre ou d’oxygène, de même que pour les molécules constituées exclusivement d’atomes de carbone. Le cas des molécules comportant des atomes d’azote est quelque peu différent puisque la région spectrale située en-deçà de 600 cm$^{-1}$ diffère suivant que l’on considère un monomère ou un dimère, permettant de conclure si le système est ou non en interaction.\\
	 
	Toutefois, la zone spectrale associée aux modes de vibration intermoléculaires, calculée pour se situer aux bas nombres d’onde, pose des problèmes de fiabilité du point de vue des outils théoriques employés. La prise en compte de l’environnement du système au travers de calculs à l’état solide, en employant les corrections de dispersion offertes par les méthodes de DFT les plus récentes, garanti la validité de nos conclusions en permettant une caractérisation fiable des modes de vibration de phonons, même aux bas nombres d’onde.\\ 
	
	Par la mise en œuvre de ce protocole innovant, couplant expérience et théorie pour l’étude des asphaltènes, le présent travail de thèse dévoile le rôle prépondérant des noyaux polyaromatiques dans le processus d’agrégation, de même que l’influence de l’hétéroatome et de la taille, comme de la nature, des ramifications sur la signature des modes de vibration intermoléculaires. 